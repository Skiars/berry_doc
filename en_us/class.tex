\chapter {Object Oriented Function}

For optimization considerations, Berry did not consider simple types as objects. These simple types include \texttt{nil} types, numeric types, boolean types, and string types. But Berry provides classes to implement the object mechanism. Among Berry's basic data types, \texttt{list}, \texttt{map} and \texttt{range} are class objects. An object is a collection containing data and methods, where data is composed of some variables, and methods are functions. The type of an object is called a class, and the entity of an object is called an instance.

\section {Class and instance}

\subsection {Class declaration}

To use a class, you must first declare it. The declaration of a class starts with the keyword \texttt{class}. The member variables and methods of the class must be specified in the declaration. This is an example of declaring a class:
\begin{lstlisting}[language=berry, numbers=none]
class person
    var name, age
    def init(name, age)
        self.name = name
        self.age = age
    end
    def tostring()
        return'name: '+ str(self.name) +', age:' + str(self.age)
    end
end
\end{lstlisting}

Class member variables are declared with keyword \texttt{var}, while member methods are declared with keyword \texttt{def}. Currently, Berry does not support initializing member variables at the time of definition, so the initialization of member variables should be done by the constructor. The properties of the class cannot be modified after the declaration is completed, so the class is a read-only object \footnote {This design is to ensure that the class can be statically constructed in the C language when the interpreter is implemented and the \texttt{const} property can be used Modified to save RAM}.

Berry's class does not support access restrictions, and all properties of the class are visible to the outside. In native classes, you can use some tricks to make properties invisible to Berry code (usually let the member name start with "\text{.}"). You can use some conventions to restrict access to the members of the class, such as the convention that the attributes starting with an underscore are private attributes. This convention does not have any use at the grammatical level, but is conducive to the logical structure of the code.

\subsection {Instantiate}

The class itself is just an abstract description. Taking cars as an example, I know the concept of cars, and when we really want to use cars, we need real cars. The use of classes is similar. We will not only use this abstract description, but need to produce a concrete object based on this description. This process is called \textbf{Instantiation of the class}, abbreviated as instantiation, and the concrete object produced by instantiation is called \textbf{Instance}. The class itself does not have data, and instantiation produces an instance based on the information described by the class and gives the instance specific data.

\subsection {Method and \texttt{self} Parameters}

Class methods are essentially functions. Unlike ordinary functions, methods implicitly pass in a \texttt{self} parameter, and \text{self} is always the first parameter, which stores a reference to the current instance. Due to the existence of \texttt{self} parameters, the number of parameters of the method will be one more than the number of parameters defined in the declaration. Here we use a simple example to demonstrate:
\begin{lstlisting}[language=berry, numbers=none]
class Test
    def method()
        return self
    end
end
object = Test()
print(object)
print(object.method())
\end{lstlisting}
This example defines a \texttt{Test} class, which has a \texttt{method} method, which returns its \texttt{self} parameter. The last two lines in the routine print the value of the instance \texttt{object} of the \texttt{Test} class and the return value of the method \texttt{method} respectively. The running result of this example is \footnote{Because the instance objects are dynamically allocated and their memory addresses are random, the result of the reader running this code may be different from here. }
\begin{lstlisting}[numbers=none]
<instance: 00E880D4>
<instance: 00E880D4>
\end{lstlisting}
It can be seen that the \texttt{self} parameter of the method and the name of the use instance (\texttt{object} in the example) both represent the same object, and they are both instance references. Use \texttt{self} to access the members or attributes of the instance in the method.

\subsection {Constructor and Destructor}

\subsubsection {Constructor}The constructor of the class is the \texttt{init} method. The constructor is called when the class is instantiated. Therefore, the constructor is generally used for member initialization, for example:
\begin{lstlisting}[language=berry, numbers=none]
class Test
    var a
    def init()
        self.a ='this is a test'
    end
end
\end{lstlisting}
The constructor in this example initializes the \texttt{a} member of the \texttt{Test} class to the string \texttt{'this is a test'}. If we instantiate the class, we can get the value of member \texttt{a}:
\begin{lstlisting}[language=berry, numbers=none]
print(Test().a) # this is a test
\end{lstlisting}

\subsubsection {Destructor}

The destructor of the class is the \texttt{deinit} method. The destructor is called when the instance is destroyed. The destructor is generally used to complete some cleanup work. Because the garbage collection mechanism automatically releases the memory of useless objects, there is no need to release the memory in the destructor (and there is no way to release the memory in the destructor). In most cases, there is no need to use a destructor, unless a certain class requires certain processing when it is destroyed. A typical example is that a file object must close the file when it is destroyed.

\section {Class inheritance}

Berry only supports single inheritance, that is, a class can only have one base class, and the base class uses the operator \texttt{:} to declare:
\begin{lstlisting}[language=berry, numbers=none]
class Test: Base
    ...
end
\end{lstlisting}
Here the \texttt{Test} class inherits from the \texttt{Base} class. The subclass will inherit all the methods and properties of the base class, and you can override them in the subclass. This mechanism is called \textbf{Overload}. Under normal circumstances, we will only overload methods, not properties.

The inheritance mechanism of the Berry class is relatively simple. Subclasses will contain references to the base class, and instance objects are similar. When instantiating a class with a base class, multiple objects are actually generated. These objects will be chained together according to the inheritance relationship, and finally we will get the instance object at the end of the inheritance chain.

\section {Method Overload}

\textbf{Overload} means that the subclass and the base class use the same name method, and the subclass method will override the mechanism of the base class method. To be precise, member variables can also be overloaded, but this overloading has no meaning. Method overloading is divided into ordinary method overloading and operator overloading.

\subsection {Common method overload}

\subsection {Operator Overloading}You can use the operator overloading of the class to make the instance support the operation of the built-in operator. For example, for a class overloaded with the addition operator, we can use the addition operator to perform operations on the instance. An overloaded operator is a method with a special name, and the overloaded function form of a binary operator is
\begin{algorithm}
    \texttt{def }$\bm{operator}$\texttt{(}$\bm{other}$\texttt{)}\\
    \qquad $\bm{block}$ \ \
    \texttt{end}
\end{algorithm}\vspace{-0.6em}\\
$\bm{operator}$ is an overloaded binary operator. The left operand of the binary operator is the \texttt{self} object, and the right operand is the value of the parameter $\bm{other}$. The overloaded function form of the unary operator is
\begin{algorithm}
    \texttt{def }$\bm{operator}$\texttt{()}\\
    \qquad $\bm{block}$ \\
    \texttt{end}
\end{algorithm}\vspace{-0.6em}\\
$\bm{operator}$ is an overloaded unary operator. To distinguish it from the subtraction operator, the unary minus sign is written as \texttt{-*} when overloaded. Operator overloaded functions should have a return value, because the default \texttt{nil} return value is usually not the expected result. Let's take an integer class as an example to illustrate the use of operator overloading. First define the \texttt{integer} class:
\begin{lstlisting}[language=berry]
class integer
    var value
    def init(v)
        self.value = v
    end
    def +(other)
        return integer(self.value + other.value)
    end
    def *(other)
        return integer(self.value * other.value)
    end
    def -*()
        return integer(-self.value)
    end
    def tostring(other)
        return str(self.value)
    end
end
\end{lstlisting}
The \texttt{integer} class overloads the plus, multiplication, and symbolic operators, and the \texttt{tostring} method is to make the instance use the \texttt{print} function to output the result. We can use a simple line of code to test the operator overloading function of the class:
\begin{lstlisting}[language=berry, numbers=none]
integer(1) + integer(2) * -integer(3) # -5
\end{lstlisting}
The result of this line of code is an instance of \texttt{integer}. The value of the \texttt{value} member of this instance is \texttt{-5}, which is the same as the result of the same four arithmetic operations on integers.

Logical operators cannot be overloaded directly. If you need an instance to support logical operations, you must implement the \texttt{tobool} method. The method has no parameters and the return value must be of Boolean type. The logic operation of the instance is actually realized by converting the instance into a Boolean value, so the logic operation of the instance is completely in line with the nature of the general logic operation. The subscript operator is not directly overloaded, but is implemented by the methods \texttt{item} and \texttt{setitem}. \texttt{item} The method is used for subscript reading, its first parameter is the subscript value, and the return value is the result of the subscript operation; \texttt{setitem} is used for subscript writing, and its first parameter is the subscript Value, the second parameter is the value to be written, this method does not use the return value.

The overloaded operator can be assigned any meaning, even not satisfying the usual properties of operators. Considering the versatility of the code and the difficulty of understanding, it is not recommended that users give overloaded operators a function far from the general meaning.

\subsubsection {Overload of compound assignment operator}

The compound assignment operator cannot be directly overloaded, but we can achieve the purpose of "overloading" the compound assignment operator by overloading the binary operator corresponding to the compound assignment operator. For example, after overloading the ``\texttt{+}'' operator, you can use the ``\texttt{+=}'' operator for instances of related classes. It is worth noting that the use of compound assignment operations on the instance will cause the variables of the bound instance to lose their reference to the instance.
\begin{lstlisting}[language=berry]
class integer
    var value
    def init(x)
        self.value = x
    end
    def +(other)
        return integer(self.value + other.value)
    end
end
a = integer(4) # a: <instance: 0x55edff400a78>
a += integer(5) # a: <instance: 0x55edff4011b8>
print(a.value) # 9
\end{lstlisting}After the 11th line of code is executed, the instance bound in the variable \texttt{a} has actually changed. This line of code is equivalent to \texttt{a = integer(4) + integer(5)}. If the binary operator of the class overload does not modify the state of the instance, then the corresponding compound assignment operator will not modify any instance (it may generate new instances).

\section{Instance}

\textbf{Instance} is an object generated after class instantiation. A class can be instantiated multiple times to generate different instances. Berry instances are referenced by the class they belong to and the corresponding data fields. All instances of a class will refer to this class, but the data fields of these instances are independent of each other.

\subsection {Access base class object}

The built-in function \texttt{super} is used to access base class objects.