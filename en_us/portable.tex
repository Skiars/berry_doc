\chapter {Porting Guide}

\section {Configuration File}

The configuration header file of the Berry interpreter is \textsl{berry\_conf.h}. This file includes a batch of macros for configuration and defines some platform-related content.

\subsection{\textsl{berry\_conf.h} File}

\subsubsection{Configuration Macro Switch}

The configuration macros introduced in this section are usually used for compiling switches of some source codes. For the convenience of description, we call this macro "macro switch". For the macro switch, "on" refers to setting the macro switch to a non-zero value, and "off" refers to setting the value of the macro switch to \texttt{0}.

Some macro switches have multiple states, not just "on" or "off". These macro switches are generally used for configurations with multiple options. There are also some configuration macros that are not macro switches. No matter what the value of these macros is, there will be no code and therefore will not participate in compilation. These macros are generally used to configure the quantity value.

\ffititle{BE\_DEBUG} \label{section::BE_DEBUG}

This macro switch is used to turn on or off the debugging function of the interpreter itself. When the value of \texttt{BE\_DEBUG} is \texttt{0}, debugging is turned off, otherwise it will be turned on. The debugging function mentioned here refers to the debugging of the interpreter, not the debugging function of the Berry program. The default value of \texttt{BE\_DEBUG} is \texttt{0}. If you use \textsl{Makefile} that comes with the interpreter project to compile, this macro switch will be turned on automatically when you use the \texttt{make debug} command.

When this macro is opened, some assertions will be turned on, and an error message will be output when the interpreter encounters an error that the assertion can catch. You can open \texttt{BE\_DEBUG} when debugging the interpreter, and close it when compiling the release.

\ffititle{BE\_SINGLE\_FLOAT}

This macro switch configures the floating point type used by the \texttt{breal} type. When the value of the macro is \texttt{0}, the \texttt{double} type will be used to implement \texttt{breal}, otherwise the \texttt{float} type will be used to implement \texttt{breal}. This macro switch can be turned on in some environments with low performance or memory configuration. In the default implementation, this macro switch is turned off.

\ffititle{BE\_INTGER\_TYPE}

This macro configures the implementation of the \texttt{bint} type. When the value of the macro is \texttt{0}, the \texttt{int} type will be used to implement \texttt{bint}, when the value is \texttt{1}, the \texttt{long} type will be used to implement \texttt{bint}, and when the value is \texttt{2}, $64$ will be used ] Bit signed integer type (\texttt{\_\_int64} under Windows, \texttt{long long} on other platforms) implements \texttt{bint}. The default value of this macro is \texttt{2}. If you want to reduce memory usage, you can set this macro to \texttt{0} or \texttt{1} to enable $32$-bit integer type.

\ffititle{BE\_RUNTIME\_DEBUG\_INFO}

This macro is used to configure runtime debugging information of Berry code. It has 3 available values: set to \texttt{0} to turn off the output of the file name and line number of the runtime debugging information, and set to \texttt{1} to output the file name and line number in the runtime debugging information, set to \texttt{2} When using \texttt{uint16\_t} (16-bit integer) type to store the row number information. Its default value is \texttt{1}.

Setting this macro to \texttt{0} will not store the file name and line number information, so the memory consumption is minimal. When set to \texttt{2}, it consumes less memory, but if the program is too long, \texttt{uint16\_t} will overflow.

\ffititle{BE\_USE\_PRECOMPILED\_OBJECT}

This macro switch configures the function of constructing objects at compile time. Turning on this macro means that constructing objects at compile time is enabled. This macro is turned on by default. When this macro is turned on, the native objects in the standard library will be generated at compile time, and when this macro is turned off, the objects in the standard library will be built at runtime.

\texttt{be\_regfunc} and \texttt{be\_regclass} functions will be affected by this macro. The built-in object table cannot be modified when using compile-time object construction. At this time, these two functions cannot register objects in the built-in scope, but register objects in the global scope.

The objects constructed during the compile time are stored together with the code and will not occupy RAM (or the readable and writable area in the memory) resources. The construction technology during the compile time can also reduce the startup time of the interpreter, so it is recommended to open this macro. Please refer to section \ref{section::precompiled_build} for more details on compile-time construction techniques.

\ffititle{BE\_STACK\_TOTAL\_MAX}

This macro defines the maximum Berry stack capacity, which refers to the number of Berry objects. When the Berry code uses more than this amount of stack, it will stop executing the program and return an error message. The default value of this macro is \texttt{2000}, which can be modified according to the memory capacity of the system.

This value does not affect the memory usage of the Berry stack, because the capacity of the Berry stack is dynamically adjusted, so no matter how much it is set to, it cannot help reduce memory usage. Its main function is to terminate execution when the Berry program consumes too much stack. These programs are very likely to be incorrect, for example, recursive function calls without return conditions will continue to consume the stack.

\ffititle{BE\_STACK\_FREE\_MIN}

This macro defines the minimum available space in the Berry stack, and its default value is \texttt{10}. The native function may push a value into the Berry stack. At this time, the stack will not automatically grow, so make sure that there is enough space in the stack for the native function to use. It is not recommended to modify this value, but to use the \texttt{be\_stack\_require} function where more stack space is really needed.

In order to detect stack overflow errors when debugging the interpreter, you can open the \texttt{BE\_DEBUG} macro (section \ref{section::BE_DEBUG}).

\ffititle{BE\_STR\_HASH\_CACHE}When this macro is opened, the short string object will save the hash value of the string to improve the running speed, but the size of each string object will increase by 4 bytes. This macro is turned off by default, and the current tests have not found that opening this macro will bring significant improvement.

\ffititle{BE\_USE\_STRING\_MODULE}

This macro switch is used to enable or disable the \texttt{string} module, which is turned on by default.

\ffititle{BE\_USE\_JSON\_MODULE}

This macro switch is used to enable or disable the \texttt{json} module, which is turned on by default.

\ffititle{BE\_USE\_MATH\_MODULE}

This macro switch is used to enable or disable the \texttt{math} module, which is turned on by default.

\ffititle{BE\_USE\_TIME\_MODULE}

This macro switch is used to enable or disable the \texttt{time} module, which is turned on by default.

\ffititle{BE\_USE\_OS\_MODULE}

This macro switch is used to enable or disable the \texttt{os} module, which is turned on by default.

\ffititle{BE\_EXPLICIT\_ABORT}

This macro determines the \texttt{abort} function used internally by the Berry interpreter. By default or when the macro is not defined, the \texttt{abort} function in the C standard library will be used. This macro is defined as \texttt{abort} by default. If the user needs to explicitly specify the \texttt{abort} function used by the interpreter, then replace the macro definition with the function required by the user. This function should be in the same form as the \texttt{abort} function declaration in the standard library.

\ffititle{BE\_EXPLICIT\_EXIT}

This macro determines the \texttt{exit} function used internally by the Berry interpreter. By default or when the macro is not defined, the \texttt{exit} function in the C standard library will be used. This macro is defined as \texttt{exit} by default. If the user needs to explicitly specify the \texttt{exit} function used by the interpreter, then replace the macro definition with the function required by the user. This function should be in the same form as the \texttt{exit} function declaration in the standard library.

\ffititle{BE\_EXPLICIT\_MALLOC}

This macro determines the \texttt{malloc} function used internally by the Berry interpreter. By default or when the macro is not defined, the \texttt{malloc} function in the C standard library will be used. This macro is defined as \texttt{malloc} by default. If the user needs to explicitly specify the function \texttt{malloc} used by the interpreter, then replace the macro definition with the function required by the user. This function should be in the same form as the \texttt{malloc} function declaration in the standard library.

\ffititle{BE\_EXPLICIT\_FREE}

This macro determines the \texttt{free} function used internally by the Berry interpreter. By default or when the macro is not defined, the \texttt{free} function in the C standard library will be used. This macro is defined as \texttt{free} by default. If the user needs to explicitly specify the \texttt{free} function used by the interpreter, then replace the macro definition with the function required by the user. This function should be in the same form as the \texttt{free} function declaration in the standard library.

\ffititle{BE\_EXPLICIT\_REALLOC}

This macro determines the \texttt{realloc} function used internally by the Berry interpreter. By default or when the macro is not defined, the \texttt{realloc} function in the C standard library will be used. This macro is defined as \texttt{realloc} by default. If the user needs to explicitly specify the \texttt{realloc} function used by the interpreter, then replace the macro definition with the function required by the user. This function should be in the same form as the \texttt{realloc} function declaration in the standard library.

\ffititle{be\_assert}

This macro is used to define the implementation of the assertion function. By default, the \texttt{assert} function in the C standard library is used to implement the assertion. If the target system is inconvenient to use the \texttt{assert()} function in the standard library to make an assertion, you can modify the definition of the \texttt{be\_assert} macro. A correct assertion function should use the following declaration:

\begin{lstlisting}[language=c, numbers=none]
void assert(int condition);
\end{lstlisting}

Among them, \texttt{condition} is the assertion condition. If the condition is not met, an error message will be output and the program will be terminated. Of course, the "assert" function is usually implemented using a macro.

\section{\textsl{berry\_port.c} File}

This file implements the low-level IO functions of the Berry interpreter, including standard input and output and file system support. The \textsl{berry\_port.c} file in the \textsl{default} directory contains a set of portable IO support. File operations and standard input and output are implemented using APIs in the C standard library. Path and folder operations support both Windows and POSIX standard APIs. This file also implements a set of FatFs-based IO operation functions for users to use directly. If you need to use the Berry interpreter in other environments, then these functions must be implemented separately (may only need to be implemented partially).

This section will introduce the functions of the functions implemented in the \textsl{berry\_port.c} file and guide users to implement their own version.

\ffititle{be\_writebuffer}

\begin{lstlisting}[language=c, numbers=none]
void be_writebuffer(const char *buffer, size_t length);
\end{lstlisting}

Output a piece of data to the standard output device, the parameter \texttt{buffer} is the first address of the output data block, and \texttt{length} is the length of the output data block. This function outputs to the \texttt{stdout} file by default. Inside the interpreter, this function is usually used as a character stream output, not a binary stream.

\texttt{be\_writebuffer} Functions are very versatile and must be implemented.

\ffititle{be\_readstring}\begin{lstlisting}[language=c, numbers=none]
char* be_readstring(char *buffer, size_t size);
\end{lstlisting}

Input a piece of data from the standard input device, and read at most one row of data each time this function is called. The parameter \texttt{buffer} is the data buffer passed in by the caller, and the capacity of the buffer is \texttt{size}. This function will stop reading and return when the buffer capacity is used up, otherwise it will return when a newline character or end of file character is read. If the function executes successfully, it will directly use the \texttt{buffer} parameter as the return value, otherwise it will return \texttt{NULL}.

This function will add the read line breaks to the read data, and each time the \texttt{be\_readstring} function is called, it will continue to read from the current position. This function is only called in the implementation of the native function \texttt{input}, and the \texttt{be\_readstring} function may not be implemented when it is not necessary.

\ffititle{be\_fopen}

\begin{lstlisting}[language=c, numbers=none]
void* be_fopen(const char *filename, const char *modes);
\end{lstlisting}

To open a file, \texttt{filename} is the name of the file to be opened, and \texttt{modes} is the opening method. The function will return a file handle or a pointer to the file operation structure. The usage of this function is similar to the \texttt{fopen} function in the C standard library. The file name is a C-style string (ending with a \texttt{\textbackslash 0} character), and the pattern should at least support the following conditions:

\begin{itemize}
    \item \texttt{r}, \texttt{rt}: To open a text file in read-only mode, the file must exist.
    \item \texttt{r+}, \texttt{rt+}: Open a text file in read-write mode, and create a new file if the file does not exist.
    \item \texttt{rb}: Open a binary file in read-only mode, the file must exist.
    \item \texttt{rb+}: Open a binary file in read-write mode, and create a new file if the file does not exist.
    \item \texttt{w}, \texttt{wt}: Create and open a text file in write-only mode, and the existing file will be deleted.
    \item \texttt{w+}, \texttt{wt+}: Create and open a text file in read-write mode, and the existing file will be deleted.
    \item \texttt{wb}: Create and open a binary file in write-only mode, and the existing file will be deleted.
    \item \texttt{wb+}: Create and open a binary file in read-write mode, and the existing file will be deleted.
\end{itemize}

By default, the \texttt{fopen} function in the C standard library is used to implement \texttt{be\_fopen}. If you use other methods to achieve, you should ensure that the above operating modes can be achieved. If no file operations are required, this function can be left blank. The file operations here include all scenarios such as using the \texttt{open} function in the script, loading the script from a file (using the \texttt{be\_loadfile} function), etc.

\ffititle{be\_fclose}

\begin{lstlisting}[language=c, numbers=none]
int be_fclose(void *hfile);
\end{lstlisting}

Close a file, \texttt{hfile} is the closed file handle. The function of this function is similar to the function \texttt{fclose} in the C standard library.

\ffititle{be\_fwrite}

\begin{lstlisting}[language=c, numbers=none]
size_t be_fwrite(void *hfile, const void *buffer, size_t length);
\end{lstlisting}

Write a piece of data to the specified file. Parameter \texttt{hfile} is the file handle to be written, \texttt{buffer} is the pointer of the data to be written, \texttt{length} is the number of data to be written (in bytes).

\ffititle{be\_fread}

\begin{lstlisting}[language=c, numbers=none]
size_t be_fread(void *hfile, void *buffer, size_t length);
\end{lstlisting}

Read a piece of data from the specified file. The parameter \texttt{hfile} is the file handle to be read, \texttt{buffer} is the pointer to the read buffer, and \texttt{length} is the number of bytes to be read.

\ffititle{be\_fgets}

\begin{lstlisting}[language=c, numbers=none]
char* be_fgets(void *hfile, void *buffer, int size);
\end{lstlisting}

Read a line from the file, similar to the \texttt{fgets} function in the C standard library. Parameter \texttt{hfile} is the file handle to be read, \texttt{buffer} is the pointer of the read buffer, and \texttt{size} is the capacity of the read buffer. This function will return when \texttt{size - 1} bytes, newline characters and end of file characters are read, and the return value is \texttt{buffer}.

\ffititle{be\_fseek}\begin{lstlisting}[language=c, numbers=none]
int be_fseek(void *hfile, long offset);
\end{lstlisting}

Set the position of the file read and write pointer. The parameter \texttt{hfile} is the file handle to be operated, and \texttt{offset} is the value to be set.

\ffititle{be\_ftell}

\begin{lstlisting}[language=c, numbers=none]
long int be_ftell(void *hfile);
\end{lstlisting}

Get the current read and write pointer of the file, the parameter \texttt{hfile} is the handle of the file to be operated, and the return value of this function is the read and write pointer of the file.

\ffititle{be\_fflush}

\begin{lstlisting}[language=c, numbers=none]
long int be_fflush(void *hfile);
\end{lstlisting}

Write the data in the file buffer to the file. Parameter \texttt{hfile} is the file to be operated.

\ffititle{be\_fsize}

\begin{lstlisting}[language=c, numbers=none]
size_t be_fsize(void *hfile);
\end{lstlisting}

Get the size of the file. Parameter \texttt{hfile} is the file to be operated.