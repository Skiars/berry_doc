\chapter{表达式}

\section{基础}

一个表达式(Statement)由一个到多个操作数和运算符组成,通过对表达式求值可以得到一个结果,这个结果被称为表达式的值。操作数可以是一个字面值、变量、函数调用或者是子表达式等。使用简单的表达式和运算符也可以组合成较为复杂的表达式。与四则运算类似,运算符的优先级会影响表达式的求值顺序,优先级越高的运算符,其表达式越先求值。

\subsection{运算符和表达式}

Berry提供一些一元运算符(Unary Operator)和二元运算符(Binary Operator)。例如逻辑与运算符 \texttt{\&\&} 就是一个二元运算符,而逻辑非运算符 \texttt{!} 是一个一元运算符。一些运算符既可以是一元运算符也可以是二元运算符,这种运算符的具体含义要通过上下文来决定。例如运算符 \texttt{-} 在表达式 \texttt{-1} 中是一元符号,但是在表达式 \texttt{1-2} 中则是二元的减号。

\subsubsection{运算符组合表达式}

二元运算符左右两边都可以是子表达式,因此可以使用二元运算符来组合表达式。一个较为复杂的表达式往往具有多种运算符和操作数。这时候,表达式中各个子表达式的求值顺序(order of evaluation)可能会影响到表达式的值。而运算符的优先级(precedence)和结合性(associativity)保证了表达式求值顺序的唯一性。例如一个组合表达式:
\begin{lstlisting}[language=berry, numbers=none]
1 + 10 / 2 * 3
\end{lstlisting}
日常使用的四则运算会先计算除法表达式 \texttt{10/2},然后计算乘法表达式,最后计算加法表达式。这是因为乘法和除法比加法具有更高的优先级。

\subsubsection{操作数类型}

在表达式的运算中,操作数可能和运算符具有不匹配的类型,另外二元运算符通常要求左右操作数是同一种类型。表达式 \texttt{'10'+10} 是错误的,你无法对一个字符串和一个整数做加法,而表达式 \texttt{-'b'} 的问题是不能对字符串取负值。有时候二元运算符两边的操作数类型不同但是却可以执行运算,例如将一个整数和一个实数相加时,整数对象会被转换为实数并与另一个实数对象相加。而逻辑与和逻辑或运算符允许运算符两边的操作数为任意类型,在逻辑表达式中它们总是会按照一定的规则转换为 \texttt{boolean} 类型。

另外一种情况是使用自定义类的时候可以重载运算符。本质上来说你可以任意解释这个运算符,那么它的操作数应该是什么类型也是由你决定了。

\subsection{优先级和结合性}

在使用多个运算符组合而成的复合表达式中,运算符的优先级和结合性决定了表达式的求值顺序。各个运算符的优先级和结合性在表\ref{tab::operator_list}中给出。

优先级指定了不同运算符之间的求值顺序,具有高优先级运算符的表达式会先被求值。例如,对表达式 \texttt{1+2*3} 的求值过程会先计算 \texttt{2*3} 的结果,然后计算加法表达式的结果。使用括号可以提升低优先级表达式的求值顺序,例如在表达式 \texttt{(1+2)*3} 求值中,先计算括号中表达式 \texttt{1+2} 的结果,然后计算括号外的乘法表达式。

结合性是指运算符两侧操作数的求值顺序,这里的操作数可能是子表达式。例如,在加法表达式 \texttt{expr1 + expr2} 中,先计算 \texttt{expr1} 的值再计算 \texttt{expr2} 的值,这是因为加法运算符是左结合的。

\begin{table}[htb]
    \centering
    \setlength{\tabcolsep}{3mm}
    \begin{tabular}{cclc} \Xhline{1pt}
        \makecell[c]{\textbf{优先级}} & \makecell[c]{\textbf{运算符}} & \makecell[c]{\textbf{说明}} & \makecell[c]{\textbf{结合性}} \\ \Xhline{1pt}
        1 & \texttt{()} & 分组符号 & - \\
        2 & \texttt{() [] .} & 函数调用,下标运算,域运算 & 左 \\
        3 & \texttt{- ! \textasciitilde} & 负号,逻辑非,位翻转 & 左 \\
        4 & \texttt{* / \%} & 乘法,除法,取余数 & 左 \\
        5 & \texttt{+ -} & 加法,减法 & 左 \\
        6 & \texttt{<< >>} & 左移,右移 & 左 \\
        7 & \texttt{\&} & 位与 & 左 \\
        8 & \texttt{\textasciicircum} & 位异或 & 左 \\
        9 & \texttt{|} & 位或 & 左 \\
        10 & \texttt{..} & 范围运算符 & 左 \\
        11 & \texttt{< <= > >=} & 小于,小于等于,大于,大于等于 & 左 \\
        12 & \texttt{== !=} & 等于,不等于 & 左 \\
        13 & \texttt{\&\&} & 逻辑与 & 左 \\
        14 & \texttt{||} & 逻辑或 & 左 \\
        15 & \texttt{=} & 赋值 & 右 \\
        \Xhline{1pt}
    \end{tabular}
    \caption{运算符列表}
    \label{tab::operator_list}
\end{table}

\subsubsection{使用括号提升优先级}

当我们需要低优先级较低的运算符先求值时可以使用括号。表达式求值过程中会先计算括号中表达式的值。也就是说,对于整个表达式来说,括号中的表达式相当于一个操作数,而不管括号中的表达式的构成。

\section{运算符}

\subsection{算术运算符}

算术运算符用于实现算术运算,这些运算符和我们平时使用的数学符号相似。Berry提供的算术运算符如表\ref{tab::arthmetic_operator}所示。

\begin{table}[htb]
    \centering
    \setlength{\tabcolsep}{10mm}
    \begin{tabular}{ccc} \Xhline{1pt}
        \makecell[c]{\textbf{运算符}} & \makecell[c]{\textbf{功能}} & \makecell[c]{\textbf{示例}} \\ \Xhline{1pt}
        \texttt{-} & 一元负号 & \texttt{- expr} \\
        \texttt{+} & 加号/字符串连接 & \texttt{expr + expr} \\
        \texttt{-} & 减号 & \texttt{expr - expr} \\
        \texttt{*} & 乘号 & \texttt{expr * expr} \\
        \texttt{/} & 除号 & \texttt{expr / expr} \\
        \texttt{\%} & 取余数 & \texttt{expr \% expr} \\
        \Xhline{1pt}
    \end{tabular}
    \caption{算术运算符}
    \label{tab::arthmetic_operator}
\end{table}

二元运算符 \texttt{+} 除了做为加号外还是字符串连接符,当该运算符的操作数为字符串时会执行字符串连接,把两个字符串连接成一条更长的字符串。准确地说,作为字符串连接符的 \texttt{+} 已经不属于算术运算符的范畴了。

二元运算符 \texttt{\%} 是取余数符号,它的操作数必须都是整数,取余运算的结果是左操作数除以右操作数后的余数,例如 \texttt{11\%4} 的结果是 \texttt{3} 。实数类型不能做整除,因此不支持取余。

通常情况下算术运算符不满足交换律。例如表达式 \texttt{2/4} 和 \texttt{4/2} 的值并不相同。

所有的算数运算符都可以在类中重载,重载后的运算符不一定局限于它们原本的功能设计,而是由程序员自己决定。

\subsection{关系运算符}

关系运算符用于比较操作数的大小关系。Berry支持的6种关系运算符在表\ref{tab::relop_operator}中给出。

\begin{table}[htb]
    \centering
    \setlength{\tabcolsep}{10mm}
    \begin{tabular}{ccc} \Xhline{1pt}
        \makecell[c]{\textbf{运算符}} & \makecell[c]{\textbf{功能}} & \makecell[c]{\textbf{示例}} \\ \Xhline{1pt}
        \texttt{<} & 小于 & \texttt{expr < expr} \\
        \texttt{<=} & 小于等于 & \texttt{expr <= expr} \\
        \texttt{==} & 等于 & \texttt{expr == expr} \\
        \texttt{!=} & 不等于 & \texttt{expr != expr} \\
        \texttt{>=} & 大于等于 & \texttt{expr >= expr} \\
        \texttt{>} & 大于 & \texttt{-expr} \\
        \Xhline{1pt}
    \end{tabular}
    \caption{关系运算符}
    \label{tab::relop_operator}
\end{table}

通过比较操作数的大小关系或判断操作数是否相等,对关系表达式求值会产生一个布尔类型的结果。当关系满足时,关系表达式的值为 \texttt{true},否则为 \texttt{false}。关系运算符 \texttt{==} 和 \texttt{!=} 可以使用任何类型的操作数,且允许左右两边的操作数具有不同的类型。其他关系运算符允许使用以下几种操作数的组合:\vspace{-0.5em}
\begin{gather*}
    \bm{integer} \quad relop \quad \bm{integer} \\
    \bm{real} \quad relop \quad \bm{real} \\
    \bm{integer} \quad relop \quad \bm{real} \\
    \bm{real} \quad relop \quad \bm{integer} \\
    \bm{string} \quad relop \quad \bm{string}
\end{gather*}

关系运算中,等号 \texttt{==} 和不等号 \texttt{!=} 满足交换律。如果左右操作数是同一类型或都是数值类型(整数和实数)则根据操作数的值来判断操作数是否相等,否则认为操作数不相等。等于和不等于是互逆运算:若 \texttt{a==b} 为真,则 \texttt{a!=b} 为假,反之也成立。其他关系运算符不满足交换律,但具有下列性质:\texttt{<} 和 \texttt{>=} 是互逆运算,\texttt{>} 和 \texttt{<=} 是互逆运算。关系运算要求操作数必须是相同类型,否则是错误的表达式。

实例可以重载运算符为方法。如果重载关系运算符,程序需要自行保证以上性质。

关系运算符中,\texttt{==} 和 \texttt{!=} 运算符比 \texttt{<}、\texttt{<=}、\texttt{>} 和 \texttt{>=} 的要求更为宽松,后者只允许在相同的类型之间进行比较。在实际的程序开发中,相等或者不等的判定通常比大小的判定要简单,有些操作对象可能不能判断大小而只能判断相等或不等,布尔类型就是如此。

\subsection{逻辑运算符}

逻辑运算符分为逻辑与、逻辑或和逻辑非3种。如表\ref{tab::logic_operator}所示。

\begin{table}[htb]
    \centering
    \setlength{\tabcolsep}{10mm}
    \begin{tabular}{ccc} \Xhline{1pt}
        \makecell[c]{\textbf{运算符}} & \makecell[c]{\textbf{功能}} & \makecell[c]{\textbf{示例}} \\ \Xhline{1pt}
        \texttt{\&\&} & 逻辑与 & \texttt{expr \&\& expr} \\
        \texttt{||} & 逻辑或 & \texttt{expr || expr} \\
        \texttt{!} & 逻辑非 & \texttt{!expr} \\
        \Xhline{1pt}
    \end{tabular}
    \caption{逻辑运算符}
    \label{tab::logic_operator}
\end{table}

对于逻辑与运算符,当两个操作数的值都为 \texttt{true} 时,逻辑表达式的值为 \texttt{true},否则为 \texttt{false}。

对于逻辑或运算符,当两个操作数的值都为 \texttt{false} 时,逻辑表达式的值为 \texttt{false},否则为 \texttt{true}。

逻辑非运算符的作用是翻转操作数的逻辑状态。当操作数的值为 \texttt{true} 时,逻辑表达式的值为 \texttt{false},否则值为 \texttt{true}。

逻辑运算符要求操作数是布尔类型,如果操作数不是布尔类型则会进行转换。转换规则见\ref{section:type_bool}节。

逻辑运算使用了一种称为\textbf{短路求值}(short-circuit evaluation)的求值策略。这种求值策略为:对于逻辑与运算符,当且仅当左操作数为真时才会对又操作数求值;对于逻辑或运算符来说,当且仅当左操作数为假时才会对右操作数求值。短路求值的性质导致逻辑表达式中的代码可能不会全部运行。

\subsection{位运算符}

位运算符能实现一些二进制位的操作,位运算只能对整数类型使用。位运算符的的详细信息如表\ref{tab::bitwise_operator}所示。位运算是指直接对整数进行二进制位的操作。逻辑运算都可以扩展为位运算,以逻辑与为例,我们可以在每个二进制位上执行该运算以实现按位与,例如 $110_b\ {\rm AND}\ 011_b = 010_b$。位运算还支持移位运算,该运算会在二进制的基础上对数字进行移动。

\begin{table}[htb]
    \centering
    \setlength{\tabcolsep}{10mm}
    \begin{tabular}{ccc} \Xhline{1pt}
        \makecell[c]{\textbf{运算符}} & \makecell[c]{\textbf{功能}} & \makecell[c]{\textbf{示例}} \\ \Xhline{1pt}
        \texttt{\textasciitilde} & 按位翻转 & \texttt{\textasciitilde expr} \\
        \texttt{\&} & 按位与 & \texttt{expr \& expr} \\
        \texttt{|} & 按位或 & \texttt{expr | expr} \\
        \texttt{\textasciicircum} & 按位异或 & \texttt{expr \textasciicircum\ expr} \\
        \texttt{<<} & 左移 & \texttt{expr << expr} \\
        \texttt{>>} & 右移 & \texttt{expr >> expr} \\
        \Xhline{1pt}
    \end{tabular}
    \caption{位运算符}
    \label{tab::bitwise_operator}
\end{table}

尽管只能用于整数,位运算依然用途广泛。位运算可以实现很多优化技巧,在很多算法中,使用位运算能节省不少代码。例如判断一个数 \texttt{n} 是否是 2 的整次幂,我么可以依据 \texttt{n \& (n - 1)} 的结果是否是 \texttt{0} 来判断。在一些执行效率比较高的语中,移位运算还可用来优化乘除法(在脚本语言中通常不会有明显的效果)。

按位与运算符``\texttt{\&}''是二元运算符,它将两个整数操作数进行二进制位的与运算:只有操作数对应的二进制位都为 \texttt{1} 时,结果的该位才为 \texttt{1}。例如 $1110_b\ \&\ 0111_b = 0110_b$。

按位或运算符``\texttt{|}''是二元运算符,它将两个整数操作数进行二进制位的或运算:只有操作数对应的二进制位都为 \texttt{0} 时,结果的该位才为 \texttt{0}。例如 $1000_b\ |\ 0001_b = 1001_b$。

按位异或运算符``\texttt{\textasciicircum}''是二元运算符,它将两个整数操作数进行二进制位的异或运算:操作数对应的二进制位不同时,结果的该位值为 \texttt{1}。例如 $1100_b\ \hat{}\ 0101_b = 1001_b$。

向左移位运算符``\texttt{<<}''是二元运算符,它在二进制的基础上把左操作数向左移动由右操作数指定的位数。例如 $00001010_b \ll 3 = 01010000_b$。

向右移位运算符``\texttt{>>}''是二元运算符,它在二进制的基础上把左操作数向右移动由右操作数指定的位数。例如 $10100000_b \gg 3 = 00010100_b$。

按位翻转运算符``\textasciitilde''是一元运算符,表达式的结果是把操作数的每个二进制位的值进行翻转。例如 $\mathtt{\sim}10100011_b = 01011100_b$。

以下是一些使用位运算的例子,通常我们不会直接使用二进制,例子中的结果都已经转换成了常用的进制。
\begin{lstlisting}[language=berry, numbers=none]
1 << 1      # 2
168 >> 4    # 10
456 & 127   # 72
456 | 127   # 511
0xA5 ^ 0x5A # 255
~2          # -3
\end{lstlisting}

\subsection{赋值运算符} \label{section::assign_operator}

赋值运算符 \texttt{=} 仅出现在赋值表达式中,其左操作数必须是可写对象。赋值表达式没有结果,因此不能使用连续的赋值运算。这是一些赋值表达式的例子:
\begin{lstlisting}[language=berry, numbers=none]
a = 45    b = 'string'    c = 0
\end{lstlisting}
而下列赋值表达式都是非法的:
\begin{lstlisting}[language=berry, numbers=none]
1 = 5           # Trying to assign a constant 1
a = b = 0       # Continuous assignment
\end{lstlisting}

赋值运算会将右侧操作数的值传递左侧操作数,对于 \texttt{nil}、整数、实数和布尔类型,赋值运算是会传递对象的值,而其他类型的赋值运算传递对象的引用。

\subsection{域运算符和下标运算符}

域运算符 \texttt{.} 用于访问对象的一个属性或者成员,你可以对模块和实例这两种类型来使用域运算符:
\begin{lstlisting}[language=berry, numbers=none]
l = list[]
l.append('item 0')
s = l.item(0)       # 'item 0'
\end{lstlisting}

下标运算符 \texttt{[]} 用于访问对象的元素,例如
\begin{lstlisting}[language=berry, numbers=none]
l[2] = 10   # Read by index
n = l[2]    # Write by index
\end{lstlisting}

支持使用下标读取的类要实现 \texttt{item} 方法,支持使用下标写入的类要实现 \texttt{setitem} 方法。标准容器中的 map 和 list 实现了这两个方法,因此它们支持使用下标运算符读取和写入。字符串支持下标读取,但是不支持通过下标写入(字符串是只读的值):
\begin{lstlisting}[language=berry, numbers=none]
'string'[2]         # 'r'
'string'[2] = 'a'   # error: value 'string' does not support index assignment
\end{lstlisting}

目前字符串支持使用整数下标,且下标范围不能超过字符串的长度。
