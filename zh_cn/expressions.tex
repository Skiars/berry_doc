\chapter{表达式}

\section{基础}

一个表达式(Statement)由一个到多个操作数和运算符组成,通过对表达式求值可以得到一个结果,这个结果被称为表达式的值。操作数可以是一个字面值、变量、函数调用或者是子表达式等。使用简单的表达式和运算符也可以组合成较为复杂的表达式。与四则运算类似,运算符的优先级会影响表达式的求值顺序,优先级越高的运算符,其表达式越先求值。

\subsection{运算符和表达式}

Berry提供一些一元运算符(Unary Operator)和二元运算符(Binary Operator)。例如逻辑与运算符 \texttt{\&\&} 就是一个二元运算符,而逻辑非运算符 \texttt{!} 是一个一元运算符。一些运算符既可以是一元运算符也可以是二元运算符,这种运算符的具体含义要通过上下文来决定。例如运算符 \texttt{-} 在表达式 \texttt{-1} 中是一元符号,但是在表达式 \texttt{1-2} 中则是二元的减号。

\subsubsection{运算符组合表达式}

二元运算符左右两边都可以是子表达式,因此可以使用二元运算符来组合表达式。一个较为复杂的表达式往往具有多种运算符和操作数。这时候,表达式中各个子表达式的求值顺序(order of evaluation)可能会影响到表达式的值。而运算符的优先级(precedence)和结合性(associativity)保证了表达式求值顺序的唯一性。例如一个组合表达式:
\begin{lstlisting}[language=berry, numbers=none]
1 + 10 / 2 * 3
\end{lstlisting}
日常使用的四则运算会先计算除法表达式 \texttt{10/2},然后计算乘法表达式,最后计算加法表达式。这是因为乘法和除法比加法具有更高的优先级。

\subsubsection{操作数类型}

在表达式的运算中,操作数可能和运算符具有不匹配的类型,另外二元运算符通常要求左右操作数是同一种类型。表达式 \texttt{'10'+10} 是错误的,你无法对一个字符串和一个整数做加法,而表达式 \texttt{-'b'} 的问题是不能对字符串取负值。有时候二元运算符两边的操作数类型不同但是却可以执行运算,例如将一个整数和一个实数相加时,整数对象会被转换为实数并与另一个实数对象相加。而逻辑与和逻辑或运算符允许运算符两边的操作数为任意类型,在逻辑表达式中它们总是会按照一定的规则转换为 \texttt{boolean} 类型。

另外一种情况是使用自定义类的时候可以重载运算符。本质上来说你可以任意解释这个运算符,那么它的操作数应该是什么类型也是由你决定了。

\subsection{优先级和结合性}

在使用多个运算符组合而成的复合表达式中,运算符的优先级和结合性决定了表达式的求值顺序。各个运算符的优先级和结合性在表\ref{tab::operator_list}中给出。

优先级指定了不同运算符之间的求值顺序,具有高优先级运算符的表达式会先被求值。例如,对表达式 \texttt{1+2*3} 的求值过程会先计算 \texttt{2*3} 的结果,然后计算加法表达式的结果。使用括号可以提升低优先级表达式的求值顺序,例如在表达式 \texttt{(1+2)*3} 求值中,先计算括号中表达式 \texttt{1+2} 的结果,然后计算括号外的乘法表达式。

结合性是指运算符两侧操作数的求值顺序,这里的操作数可能是子表达式。例如,在加法表达式 \texttt{expr1 + expr2} 中,先计算 \texttt{expr1} 的值再计算 \texttt{expr2} 的值,这是因为加法运算符是左结合的。

\begin{table}[htb]
    \centering
    \setlength{\tabcolsep}{3mm}
    \begin{tabular}{cclc} \toprule
        \textbf{优先级} & \textbf{运算符} & \textbf{说明} & \textbf{结合性} \\ \midrule
        1 & \texttt{()} & 分组符号 & - \\
        2 & \texttt{() [] .} & 函数调用,下标运算,域运算 & 左 \\
        3 & \texttt{- ! \textasciitilde} & 负号,逻辑非,位翻转 & 左 \\
        4 & \texttt{* / \%} & 乘法,除法,取余数 & 左 \\
        5 & \texttt{+ -} & 加法,减法 & 左 \\
        6 & \texttt{<< >>} & 左移,右移 & 左 \\
        7 & \texttt{\&} & 位与 & 左 \\
        8 & \texttt{\textasciicircum} & 位异或 & 左 \\
        9 & \texttt{|} & 位或 & 左 \\
        10 & \texttt{..} & 连接运算符 & 左 \\
        11 & \texttt{< <= > >=} & 小于,小于等于,大于,大于等于 & 左 \\
        12 & \texttt{== !=} & 等于,不等于 & 左 \\
        13 & \texttt{\&\&} & 逻辑与 & 左 \\
        14 & \texttt{||} & 逻辑或 & 左 \\
        15 & \texttt{? :} & 条件运算符 & 右 \\
        16 & \makecell{\texttt{= += -= *= /= \%=} \\
                       \texttt{\&= |= \textasciicircum= <<= >>=}} & 赋值 & 右 \\
        \bottomrule
    \end{tabular}
    \caption{运算符列表}
    \label{tab::operator_list}
\end{table}

\subsubsection{使用括号提升优先级}

当我们需要低优先级较低的运算符先求值时可以使用括号。表达式求值过程中会先计算括号中表达式的值。也就是说,对于整个表达式来说,括号中的表达式相当于一个操作数,而不管括号中的表达式的构成。

\section{运算符}

\subsection{算术运算符}

算术运算符用于实现算术运算,这些运算符和我们平时使用的数学符号相似。Berry提供的算术运算符如表\ref{tab::arthmetic_operator}所示。

\begin{table}[htb]
    \centering
    \setlength{\tabcolsep}{10mm}
    \begin{tabular}{clc} \toprule
        \textbf{运算符} & \textbf{说明} & \textbf{示例} \\ \midrule
        \texttt{-} & 一元负号 & \texttt{- expr} \\
        \texttt{+} & 加号/字符串连接 & \texttt{expr + expr} \\
        \texttt{-} & 减号 & \texttt{expr - expr} \\
        \texttt{*} & 乘号 & \texttt{expr * expr} \\
        \texttt{/} & 除号 & \texttt{expr / expr} \\
        \texttt{\%} & 取余数 & \texttt{expr \% expr} \\
        \bottomrule
    \end{tabular}
    \caption{算术运算符}
    \label{tab::arthmetic_operator}
\end{table}

二元运算符 \texttt{+} 除了做为加号外还是字符串连接符,当该运算符的操作数为字符串时会执行字符串连接,把两个字符串连接成一条更长的字符串。准确地说,作为字符串连接符的 \texttt{+} 已经不属于算术运算符的范畴了。

二元运算符 \texttt{\%} 是取余数符号,它的操作数必须都是整数,取余运算的结果是左操作数除以右操作数后的余数,例如 \texttt{11\%4} 的结果是 \texttt{3} 。实数类型不能做整除,因此不支持取余。

通常情况下算术运算符不满足交换律。例如表达式 \texttt{2/4} 和 \texttt{4/2} 的值并不相同。

所有的算数运算符都可以在类中重载,重载后的运算符不一定局限于它们原本的功能设计,而是由程序员自己决定。

\subsection{关系运算符}

关系运算符用于比较操作数的大小关系。Berry支持的6种关系运算符在表\ref{tab::relop_operator}中给出。

\begin{table}[htb]
    \centering
    \setlength{\tabcolsep}{10mm}
    \begin{tabular}{clc} \toprule
        \textbf{运算符} & \textbf{说明} & \textbf{示例} \\ \midrule
        \texttt{<} & 小于 & \texttt{expr < expr} \\
        \texttt{<=} & 小于等于 & \texttt{expr <= expr} \\
        \texttt{==} & 等于 & \texttt{expr == expr} \\
        \texttt{!=} & 不等于 & \texttt{expr != expr} \\
        \texttt{>=} & 大于等于 & \texttt{expr >= expr} \\
        \texttt{>} & 大于 & \texttt{-expr} \\
        \bottomrule
    \end{tabular}
    \caption{关系运算符}
    \label{tab::relop_operator}
\end{table}

通过比较操作数的大小关系或判断操作数是否相等,对关系表达式求值会产生一个布尔类型的结果。当关系满足时,关系表达式的值为 \texttt{true},否则为 \texttt{false}。关系运算符 \texttt{==} 和 \texttt{!=} 可以使用任何类型的操作数,且允许左右两边的操作数具有不同的类型。其他关系运算符允许使用以下几种操作数的组合:\vspace{-0.5em}
\begin{gather*}
    \bm{integer} \quad relop \quad \bm{integer} \\
    \bm{real} \quad relop \quad \bm{real} \\
    \bm{integer} \quad relop \quad \bm{real} \\
    \bm{real} \quad relop \quad \bm{integer} \\
    \bm{string} \quad relop \quad \bm{string}
\end{gather*}

关系运算中,等号 \texttt{==} 和不等号 \texttt{!=} 满足交换律。如果左右操作数是同一类型或都是数值类型(整数和实数)则根据操作数的值来判断操作数是否相等,否则认为操作数不相等。等于和不等于是互逆运算:若 \texttt{a==b} 为真,则 \texttt{a!=b} 为假,反之也成立。其他关系运算符不满足交换律,但具有下列性质:\texttt{<} 和 \texttt{>=} 是互逆运算,\texttt{>} 和 \texttt{<=} 是互逆运算。关系运算要求操作数必须是相同类型,否则是错误的表达式。

实例可以重载运算符为方法。如果重载关系运算符,程序需要自行保证以上性质。

关系运算符中,\texttt{==} 和 \texttt{!=} 运算符比 \texttt{<}、\texttt{<=}、\texttt{>} 和 \texttt{>=} 的要求更为宽松,后者只允许在相同的类型之间进行比较。在实际的程序开发中,相等或者不等的判定通常比大小的判定要简单,有些操作对象可能不能判断大小而只能判断相等或不等,布尔类型就是如此。

\subsection{逻辑运算符}

逻辑运算符分为逻辑与、逻辑或和逻辑非3种。如表\ref{tab::logic_operator}所示。

\begin{table}[htb]
    \centering
    \setlength{\tabcolsep}{10mm}
    \begin{tabular}{ccc} \toprule
        \textbf{运算符} & \textbf{说明} & \textbf{示例} \\ \midrule
        \texttt{\&\&} & 逻辑与 & \texttt{expr \&\& expr} \\
        \texttt{||} & 逻辑或 & \texttt{expr || expr} \\
        \texttt{!} & 逻辑非 & \texttt{!expr} \\
        \bottomrule
    \end{tabular}
    \caption{逻辑运算符}
    \label{tab::logic_operator}
\end{table}

对于逻辑与运算符,当两个操作数的值都为 \texttt{true} 时,逻辑表达式的值为 \texttt{true},否则为 \texttt{false}。

对于逻辑或运算符,当两个操作数的值都为 \texttt{false} 时,逻辑表达式的值为 \texttt{false},否则为 \texttt{true}。

逻辑非运算符的作用是翻转操作数的逻辑状态。当操作数的值为 \texttt{true} 时,逻辑表达式的值为 \texttt{false},否则值为 \texttt{true}。

逻辑运算符要求操作数是布尔类型,如果操作数不是布尔类型则会进行转换。转换规则见\ref{section::type_bool}节。

逻辑运算使用了一种称为\textbf{短路求值}(short-circuit evaluation)的求值策略。这种求值策略为:对于逻辑与运算符,当且仅当左操作数为真时才会对又操作数求值;对于逻辑或运算符来说,当且仅当左操作数为假时才会对右操作数求值。短路求值的性质导致逻辑表达式中的代码可能不会全部运行。

\subsection{位运算符}

位运算符能实现一些二进制位的操作,位运算只能对整数类型使用。位运算符的的详细信息如表\ref{tab::bitwise_operator}所示。位运算是指直接对整数进行二进制位的操作。逻辑运算都可以扩展为位运算,以逻辑与为例,我们可以在每个二进制位上执行该运算以实现按位与,例如 $110_b\ {\rm AND}\ 011_b = 010_b$。位运算还支持移位运算,该运算会在二进制的基础上对数字进行移动。

\begin{table}[htb]
    \centering
    \setlength{\tabcolsep}{10mm}
    \begin{tabular}{clc} \toprule
        \textbf{运算符} & \makecell[c]{\textbf{说明}} & \textbf{示例} \\ \midrule
        \texttt{\textasciitilde} & 按位翻转 & \texttt{\textasciitilde expr} \\
        \texttt{\&} & 按位与 & \texttt{expr \& expr} \\
        \texttt{|} & 按位或 & \texttt{expr | expr} \\
        \texttt{\textasciicircum} & 按位异或 & \texttt{expr \textasciicircum\ expr} \\
        \texttt{<<} & 左移 & \texttt{expr << expr} \\
        \texttt{>>} & 右移 & \texttt{expr >> expr} \\
        \bottomrule
    \end{tabular}
    \caption{位运算符}
    \label{tab::bitwise_operator}
\end{table}

尽管只能用于整数,位运算依然用途广泛。位运算可以实现很多优化技巧,在很多算法中,使用位运算能节省不少代码。例如判断一个数 \texttt{n} 是否是 2 的整次幂,我么可以依据 \texttt{n \& (n - 1)} 的结果是否是 \texttt{0} 来判断。在一些执行效率比较高的语中,移位运算还可用来优化乘除法(在脚本语言中通常不会有明显的效果)。

按位与运算符``\texttt{\&}''是二元运算符,它将两个整数操作数进行二进制位的与运算:只有操作数对应的二进制位都为 \texttt{1} 时,结果的该位才为 \texttt{1}。例如 $1110_b\ \&\ 0111_b = 0110_b$。

按位或运算符``\texttt{|}''是二元运算符,它将两个整数操作数进行二进制位的或运算:只有操作数对应的二进制位都为 \texttt{0} 时,结果的该位才为 \texttt{0}。例如 $1000_b\ |\ 0001_b = 1001_b$。

按位异或运算符``\texttt{\textasciicircum}''是二元运算符,它将两个整数操作数进行二进制位的异或运算:操作数对应的二进制位不同时,结果的该位值为 \texttt{1}。例如 $1100_b\ \hat{}\ 0101_b = 1001_b$。

向左移位运算符``\texttt{<<}''是二元运算符,它在二进制的基础上把左操作数向左移动由右操作数指定的位数。例如 $00001010_b \ll 3 = 01010000_b$。

向右移位运算符``\texttt{>>}''是二元运算符,它在二进制的基础上把左操作数向右移动由右操作数指定的位数。例如 $10100000_b \gg 3 = 00010100_b$。

按位翻转运算符``\textasciitilde''是一元运算符,表达式的结果是把操作数的每个二进制位的值进行翻转。例如 $\mathtt{\sim}10100011_b = 01011100_b$。

以下是一些使用位运算的例子,通常我们不会直接使用二进制,例子中的结果都已经转换成了常用的进制。
\begin{lstlisting}[language=berry, numbers=none]
1 << 1      # 2
168 >> 4    # 10
456 & 127   # 72
456 | 127   # 511
0xA5 ^ 0x5A # 255
~2          # -3
\end{lstlisting}

\subsection{赋值运算符} \label{section::assign_operator}

赋值运算符仅出现在赋值表达式中,该运算符的做操作数必须是一个可写的对象。赋值表达式没有结果,因此不能使用连续的赋值运算。

\subsubsection{简单赋值运算符}

简单赋值运算符 \texttt{=} 可以用于变量的赋值,如果左操作数变量没有定义则会定义该变量。赋值运算符用于将右操作数的值与左操作数进行绑定,该过程也称为“赋值”。因此左操作数不能是常量,也不能是任何不能写入的对象。这是一些合法的赋值表达式:
\begin{lstlisting}[language=berry, numbers=none]
a = 45    b = 'string'    c = 0
\end{lstlisting}
而以下赋值表达式是错误的:
\begin{lstlisting}[language=berry, numbers=none]
1 = 5           # Trying to assign a constant 1
a = b = 0       # Continuous assignment
\end{lstlisting}

将 \texttt{nil}、整数、实数和布尔类型 赋值给变量时,对象的值会传递给左操作数,然而对于其他类型,赋值运算只是把对象的引用传递给左操作数。由于字符串、函数和类类型是只读的,所有传引用不会有副作用,但是对于实例类型则要格外小心。

\subsubsection{复合赋值运算符}

复合赋值运算符是将二元运算符与赋值运算符组合在一起构成的运算符,它们是对简单赋值运算符的实用扩充,复合赋值运算符可以使一些表达式的书写得到简化。表\ref{tab::compound_assign}列出了所有的复合赋值运算符

\begin{table}[htb]
    \centering
    \setlength{\tabcolsep}{10mm}
    \begin{tabular}{clcl} \toprule
        \textbf{运算符} & \textbf{说明} & \textbf{运算符} & \textbf{说明} \\ \midrule
        \texttt{+=} & 加法赋值 & \texttt{-=} & 减法赋值 \\
        \texttt{*=} & 乘法赋值 & \texttt{/=} & 初法赋值 \\
        \texttt{\%=} & 取余赋值 & \texttt{\&=} & 按位与赋值 \\
        \texttt{|=} & 按位或赋值 & \texttt{\textasciicircum=} & 按位异或赋值 \\
        \texttt{<<=} & 左移赋值 & \texttt{>>=} & 右移赋值 \\
        \bottomrule
    \end{tabular}
    \caption{位运算符}
    \label{tab::compound_assign}
\end{table}

复合赋值表达式会对左操作数和右操作数执行复合赋值运算符对应的二元运算,然后将结果赋值给左操作数。以 \texttt{+=} 为例,表达式 \texttt{a += b} 与 \texttt{a = a + b} 等效。复合赋值运算符也是赋值运算符,因此具有较低的优先级,复合赋值运算符对应的二元运算符总在右操作数之后求值,故类似 \texttt{a *= 1 + 2} 这样的表达式应该等效为 \texttt{a = a * (1 + 2)}。

与简单赋值运算符不同,复合赋值运算符的左操作数要参与求值,因此复合赋值表达式没有定义变量的功能。赋值运算符本身不能在类中重载,用户只能重载复合赋值运算符对应的二元运算符,这也保证了复合赋值运算符总是会符合赋值运算的基本特征。

\subsection{域运算符和下标运算符}

域运算符 \texttt{.} 用于访问对象的一个属性或者成员,你可以对模块和实例这两种类型来使用域运算符:
\begin{lstlisting}[language=berry, numbers=none]
l = list[]
l.push('item 0')
s = l.item(0)       # 'item 0'
\end{lstlisting}

下标运算符 \texttt{[]} 用于访问对象的元素,例如
\begin{lstlisting}[language=berry, numbers=none]
l[2] = 10   # Read by index
n = l[2]    # Write by index
\end{lstlisting}

支持使用下标读取的类要实现 \texttt{item} 方法,支持使用下标写入的类要实现 \texttt{setitem} 方法。标准容器中的 map 和 list 实现了这两个方法,因此它们支持使用下标运算符读取和写入。字符串支持下标读取,但是不支持通过下标写入(字符串是只读的值):
\begin{lstlisting}[language=berry, numbers=none]
'string'[2]         # 'r'
'string'[2] = 'a'   # error: value 'string' does not support index assignment
\end{lstlisting}

目前字符串支持使用整数下标,且下标范围不能超过字符串的长度。

\subsection{条件运算符}

条件运算符(\texttt{? :})的作用类似于 \textbf{if else} 语句,不过前者可以在表达式中使用。条件运算符的使用形式为:\vspace{-0.5em}
\begin{gather*}
    cond\ \bm{?}\ expr1\ \bm{:}\ expr2
\end{gather*}

$\bm{cond}$ 是用于判断条件的表达式,条件运算符的求值过程是:首先求出 $\bm{cond}$ 的值,如果条件为真就对 $\bm{expr1}$ 求值并返回该值,否则对 $\bm{expr2}$ 求值并返回该值。$\bm{expr1}$ 和 $\bm{expr2}$ 可以具有不同的类型,因此下面这种写法是正确的:
\begin{lstlisting}[language=berry, numbers=none]
result = scope < 6 ? 'bad' : scope
\end{lstlisting}

这个表达式首先判断 \texttt{scope} 是否小于 \texttt{6},如果是则返回 \texttt{bad},否则返回 \texttt{scope} 的值。无论条件表达式的条件为何,都只会执行 $\bm{expr1}$ 或者 $\bm{expr2}$ 中的一个,类似于逻辑与和逻辑或运算的短路特性。

\subsubsection{嵌套条件运算符}

可以在一个条件运算符中嵌套另一个条件运算符,即条件表达式可以作为另一个条件表达式的 $\bm{cond}$ 或者 $\bm{expr}$。例如,使用条件表达式将分数分为优秀(excellent),好(good)和差(bad)三档:
\begin{lstlisting}[language=berry, numbers=none]
result = scope >= 9 ? 'excellent' : scope >= 6 ? 'good' : 'bad'
\end{lstlisting}

第一个条件检查分数是否不低于 \texttt{9} 分,如果是则执行 \texttt{?} 后的分支,返回 \texttt{'excellent'};否则执行 \texttt{:} 后的分支,这个分支也是一个条件表达式,它的条件检查分数是否不低于 \texttt{6},如果是则返回 \texttt{'good'},否则返回 \texttt{'bad'}。

条件运算符满足右结合性,因此要先求出分支表达式的值才能得到条件表达式的值。因此在嵌套条件表达式中,会先对被嵌套的条件表达式求值,然后才对外层的条件表达式求值。

\subsubsection{条件运算符的优先级}

由于条件表达式的优先级非常低(仅次于赋值运算符),在很多时候需要在条件表达式外加上括号。例如将条件表达式作为算术表达式的操作数时,括号会对结果产生不同的影响:
\begin{lstlisting}[language=berry, numbers=none]
result = 10 * (sign < 0 ? -1 : 1) # the result is -10 or 10
result = 10 * sign < 0 ? -1 : 1   # the result is -1 or 1
\end{lstlisting}
第一个表达式的结果是正确的,而第二个表达式将 \texttt{10 * sign < 0} 作为条件进行判断,与条件表达式作为乘法右操作数的期望不符合。

\subsection{连接运算符}

\subsubsection{\texttt{+} 运算符}

当左右操作数都是字符串时,\texttt{+} 运算符用于连接这两个字符串,得到的新字符串即表达式的值。因此该运算符常用于字符串的连接:
\begin{lstlisting}[language=berry, numbers=none]
result = 'abc' + '123' # the result is 'abc123'
\end{lstlisting}

\texttt{+} 运算符还可以用来连接两个 list 实例:
\begin{lstlisting}[language=berry, numbers=none]
result = [1, 2] + [3, 4] # the result is [1, 2, 3, 4]
\end{lstlisting}
和 \texttt{list.push} 方法不同,\texttt{+} 运算符会把两个 list 合并为一个更大的 list 对象,左操作数的元素在结果 list 的头部,而右操作数的元素在结果 list 的尾部。

\subsubsection{\texttt{..} 运算符}

\texttt{..} 是一个比较特殊的运算符。如果左操作数是一个字符串,则表达式的行为是将左右操作数拼接为一个新的字符串(如果右操作数不是字符串则自动转换):
\begin{lstlisting}[language=berry, numbers=none]
result = 'abc' + 123 # the result is 'abc123'
\end{lstlisting}
将一个字符串和非字符串值拼接时常使用 \texttt{..} 运算符。

如果左操作数是一个 list 实例,\texttt{..} 运算符会将右操作数追加到该 list 的尾部,然后将此 list 作为表达式的值:
\begin{lstlisting}[language=berry, numbers=none]
result = [1, 2] .. 3 # the result is [1, 2, 3]
\end{lstlisting}
这个过程会直接修改左操作数,这和 \texttt{list} 的 \texttt{push} 方法非常相似。list 的连接运算可以链式执行:
\begin{lstlisting}[language=berry, numbers=none]
result = [1, 2] .. 3 .. 4 # the result is [1, 2, 3, 4]
\end{lstlisting}
该过程中所有的值都会被追加到最左边 list 对象中。

如果左右操作数都是整数,使用 \texttt{..} 运算符可以得到一个整数范围对象:
\begin{lstlisting}[language=berry, numbers=none]
result = 1 .. 10 # the result is (1..10)
\end{lstlisting}
该对象用于表示一个整数闭区间,其中左操作数为下限,右操作数为上限。这种整数范围对象常用于迭代。
