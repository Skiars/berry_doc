\chapter*{序\quad 言}

\pagestyle{empty}
\thispagestyle{empty}

几年前我曾尝试过将Lua脚本语言移植到STM32F4单片机,后来又在ESP8266上体验过一款基于Lua的固件:NodeMCU,这些经历使我体验到使用脚本开发的便利。后来我又接触了一些脚本语言,例如Python、JavaScript、Basic以及MATLAB等。目前看来,只有极少数语言是适合移植到单片机平台的。以前我比较关注Lua,因为它是一款定位非常精简的嵌入式脚本语言,其设计目标就是嵌入到宿主程序中使用。然而对于单片机而言,Lua解释器可能还不够小,我无法在存储器比较小的32位单片机上运行它。为此,我开始阅读Lua代码并在此基础上开发自己的脚本语言------Berry。

这是一款超轻量级的嵌入式脚本语言,它还是一款多范式的动态语言。支持面向对象、面向过程和函数式(支持比较少)风格。这款语言的很多方面都参考了Lua,不过它的语法还借鉴了其他一些语言的设计。如果读者已经了解一门高级语言,Berry的语法应该会非常容易掌握:该语言只有一些简单的规则,并且有着非常自然的作用域设计。

我考虑的主要应用场景是性能较低的嵌入式系统,这些系统的内存可能非常小,因此要运行一个功能全面的脚本语言十分困难。这意味着我们可能不得不做出取舍,Berry也许只会提供最常用而基础的核心功能,而其他的不必要的特性只作为可选的模块,这必然导致语言的标准库过小,甚至语言本身也会有不确定的设计(例如浮点数和整数的实现方式等)。这些折衷带来的收益则是较大的优化空间,而坏处显然是语言标准的不统一。

Berry的解释器参考了Lua解释器的实现,解释器主要分为编译器和虚拟机两部分。Berry的编译器是一种一趟式编译器生成字节码,这种方案不生成抽象语法树,因此比较节省内存。而虚拟机则是寄存器式的,一般来说寄存器式虚拟机比堆栈式虚拟机的效率要高一些。除了实现语言特性外,我们还希望优化解释器的内存占用和运行效率。目前,Berry解释器性能指标并不比主流的语言差,但是内存占用则非常小。

直到后来我才了解到MicroPython项目:一个为单片机设计的精简版Python解释器。如今Python十分火热,这款为单片机设计的Python解释器也非常受欢迎。与目前比较成熟的技术相比,Berry是一门新的语言,没有足够的用户基础,它的优势是语法易于掌握,以及可能在资源占用和性能方面占有优势。

如果你需要移植Berry解释器,需要保证你使用的编译器提供对C99标准的支持(我先前完全遵守C89,后来的一些优化工作使我放弃了这个决定)。目前大部分编译器都提供对对C99的支持,ARM处理器开发中常见ARMCC(KEIL MDK)、ICC(IAR)以及GCC等编译器也都支持C99。

本文档介绍Berry的语法规则、标准库等设施,最后会指导读者去移植并扩展Berry。本文档不阐述解释器的实现机制,有时间的话也许会在其他的文档中说明。

作者水平有限,加上行笔匆忙,文中如有纰漏或者错误,望读者不吝斧正。

\rightline{官文亮}

\rightline{2019年4月}
