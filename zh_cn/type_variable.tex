\chapter{类型和变量}

\textbf{类型}是数据的一种属性,它定义了数据的含义以及可以对数据执行的操作。Berry中的类型主要分为类类型和简单类型(整数、浮点数、字符串等)。而类提供了用户自定义类型的机制,且类类型比简单类型提供更复杂的操作。

\section{内建类型}

Berry内置了一些简单类型或者类类型,这些类型被称为内建类型。所有可以使用字面值来表示的数据类型都属于内建类型。内建类型也是十分常用的类型。

\subsection{Nil}

Nil类型即空类型,它表示对象具有一个无效值,也可以说对象没有有意义的值。这是一个十分特殊的类型,尽管我们可能会说一个变量为\texttt{nil},但实际上nil类型没有值,因此这里说的其实是该变量的类型为nil(而不是值)。

一个变量在赋值之前的默认值就是\texttt{nil}。该类型可以用于逻辑运算,在逻辑运算中,\texttt{nil}等价于\texttt{false}。

\subsection{整数类型}

整数类型(integer)表示有符号的整数,简称整数。该类型表示所整数的比特位数取决于具体的实现,在32位平台中通常是一个32bit。Integer是一个算术类型,它支持所有算术运算。在逻辑运算中,整数\texttt{0}等价于\texttt{false},非\texttt{0}整数等价于\texttt{true}。整数可以比较大小,因此可以用于关系运算。

\subsection{实数类型}

实数类型(real),准确地说是浮点类型。实数类型通常实现为单精度浮点数或者双精度浮点数。相比于整数类型,浮点类型有更高的精度和更大的取值范围,因此该类型更适用于数学计算。需要注意的是,实数类型实际上是浮点数,因此还是存在精度问题,例如将两个\texttt{real}类型的数值进行相等比较是不推荐的。在逻辑运算中,\texttt{0.0}等价于\texttt{false},其他值等价于\texttt{true}。实数类型可以比较大小,能用于关系运算。

\subsection{布尔类型}

布尔类型(boolean)用来表示逻辑上量的真假。该类型有两个值:\texttt{true}和\texttt{false}。布尔类型只能比较等于和不等,而不能比较大小。

\subsection{字符串}

字符串(string)是一段字符构成的串。从存储上来讲,Berry把字符串分为长字符串和短字符串两种。相同的短字符串在内存中只有一个实例,并且所有的短字符串链接在一个哈希表中,这种设计有利于提高字符串相等比较的性能,并且可以减少内存使用。由于长字符串的使用频率较低,而哈希运算的开销却不少,因此没有将它们链接到哈希表中,,故内存中可能有多个相同的实例。另外,只要字符串对象被创建并初始化,那么它们将不再可以被修改,因此对字符串的编辑操作实际上会产生新的字符串,而不会修改旧的串。

Berry并不关心字符的格式或者编码,例如字符串\texttt{'abc'}实际上就是字符\texttt{'a'},\texttt{'b'}和\texttt{'c'}的ASCII码排列而成。因此,若字符串中存在宽字符(字符长度大于1字节)则无法直接统计字符串中字符的个数,使用\texttt{length()}函数实际上只能得到字符串的字节数。另外,为了方便和C语言交互,Berry的字符串总是以\texttt{'\textbackslash 0'}字符结束,这个特性对于Berry程序来说是透明的。

字符串类型可以比较大小,因此可以用于关系运算。

\subsection{函数}

函数(function)是一段具有名字的代码,它一般用于实现特定的功能。在Berry中函数实际上是一个大类,包括闭包、原生函数、原生闭包等几个子类型。不过,函数的所有子类型都有相同的操作,因此我们不需要关心``函数''具体是什么子类型。函数也是非常常用的类型,为了代码可以重复利用,我们通常把具有特定功能的对端代码做成一个函数,然后通过``函数调用''来运行这一段代码。

而函数本身可以当成一个只读的对象,而这种对象的类型就是``函数''。这意味着可以把函数赋值给一个变量并通过变量来调用。

函数类型只能比较等于和不等,而不能比较大小。

\subsection{类}

类(class)由一些成员变量和方法函数组成。类用于实现面向对象特性。类是一个抽象的对象并且是只读的。

类只能比较等于和不等,而不能比较大小。

\subsection{实例}

实例(instance)是一个类实例化之后的对象,实例是一个具体的对象,其成员变量是可读写的。

实例的运算支持要看该实例的运算符重载情况来决定。

\subsection{List}

可变数组。可以存储任何类型。

\subsection{Map}

用于存储键值对。值可以是任何类型。

\subsection{Range}

表示一个整数区间。通常用于范围迭代。

\section{变量}
