\chapter{类型和变量}

\textbf{类型}是数据的一种属性,它定义了数据的含义以及可以对数据执行的操作。Berry中的类型主要分为类类型和简单类型(整数、浮点数、字符串等)。而类提供了用户自定义类型的机制,且类类型比简单类型提供更复杂的操作。

\section{内建类型}

Berry内置了一些简单类型或者类类型,这些类型被称为内建类型。所有可以使用字面值来表示的数据类型都属于内建类型。内建类型也是十分常用的类型。

\subsection{Nil}

Nil类型即空类型,它表示对象具有一个无效值,也可以说对象没有有意义的值。这是一个十分特殊的类型,尽管我们可能会说一个变量为\texttt{nil},但实际上nil类型没有值,因此这里说的其实是该变量的类型为nil(而不是值)。

一个变量在赋值之前的默认值就是\texttt{nil}。该类型可以用于逻辑运算,在逻辑运算中,\texttt{nil}等价于\texttt{false}。

\subsection{Integer}

Integer类型表示有符号的整数,简称整数。该类型表示所整数的比特位数取决于具体的实现,在32位平台中通常是一个32bit。Integer是一个算术类型,它支持所有算术运算。在逻辑运算中,整数\texttt{0}等价于\texttt{false},非\texttt{0}整数等价于\texttt{true}。整数可以比较大小,因此可以用于关系运算。

\subsection{Real}

实数。具体实现可以选择选用单精度浮点数或者双精度浮点数。

\subsection{Boolean}

值可以是\texttt{true}或者\texttt{false}。

\subsection{String}

字符串是一个由字符组成的串。字符串是只读的。

\subsection{Function}

函数是一段具有名字的代码,它一般用于实现特定的功能。在Berry中函数实际上是一个大类,不过在通常情况下我们不用关心一个``函数''具体是什么子类型。

\subsection{Class}

由一些成员变量和方法函数组成。类用于实现面向对象特性。类是一个抽象的对象并且是只读的。

\subsection{Instance}

一个类实例化之后的对象,实例是一个具体的对象,其成员变量是可读写的。

\subsection{List}

可变数组。可以存储任何类型。

\subsection{Map}

用于存储键值对。值可以是任何类型。

\subsection{Range}

表示一个整数区间。通常用于范围迭代。
