\chapter{移植指南}

\section{配置文件}

Berry 解释器的配置头文件为 \textsl{berry\_conf.h}。此文件中包括一批用于配置的宏,并定义了一些平台相关的内容。

\subsection{\textsl{berry\_conf.h} 文件}

\subsubsection{配置宏开关}

本节介绍的配置宏通常用于某些源代码的编译开关,为了方便描述,我们把这种宏称“宏开关”。对于宏开关而言,“打开”是指将宏开关设置为非零值,而“关闭”是指将宏开关的值设置为 \texttt{0}。

有一些宏开关具有多种状态,而不仅仅是“打开”或者“关闭”,这些宏开关一般用于有多种选择的配置。还有一些配置宏不是宏开关,无论这些宏的值是多少,都不会有代码因此不参与编译,这些宏一般用于配置数量值。

\ffititle{BE\_DEBUG} \label{section::BE_DEBUG}

这个宏开关用于开启或关闭解释器本身的调试功能,\texttt{BE\_DEBUG} 的值为 \texttt{0} 时关闭调试,否则会打开调试。此处所说的调试功能是指对解释器的调试,而不是对 Berry 程序的调试功能。\texttt{BE\_DEBUG}的默认值为 \texttt{0}。如果你使用解释器项目自带的 \textsl{Makefile} 来编译,使用 \texttt{make debug} 命令时会自动打开这个宏开关。

打开这个宏时会开启一些断言,当解释器出现断言可以捕获的错误时将会输出错误信息。在调试解释器的时候可以打开 \texttt{BE\_DEBUG},编译发行版时则要关闭它。

\ffititle{BE\_SINGLE\_FLOAT}

这个宏开关配置 \texttt{breal} 类型使用的浮点类型。当宏的值为 \texttt{0} 时将使用 \texttt{double} 类型来实现 \texttt{breal},否则使用 \texttt{float} 类型实现 \texttt{breal}。在一些对性能或内存配置较低的环境中可以打开这个宏开关,默认实现中这个宏开关是关闭的。

\ffititle{BE\_INTGER\_TYPE}

这个宏配置 \texttt{bint} 类型的实现方式。当宏的值为 \texttt{0} 时将使用 \texttt{int} 类型来实现 \texttt{bint},值为 \texttt{1} 的时候将使用 \texttt{long} 类型来实现 \texttt{bint},值为 \texttt{2} 时将使用 $64$ 位有符号整数类型(Windows 下是 \texttt{\_\_int64},其他平台为 \texttt{long long})实现 \texttt{bint}。该宏的默认值为 \texttt{2}。如果想减少内存占用可以讲这个宏设置为 \texttt{0} 或者 \texttt{1} 来启用$32$位整数类型。

\ffititle{BE\_RUNTIME\_DEBUG\_INFO}

这个宏用于配置 Berry 代码的运行时调试信息。它有 3 个可用的值:设置为 \texttt{0} 会关闭运行时调试信息的文件名和行号输出,设置为 \texttt{1} 时会在运行时调试信息中输出文件名和行号,设置为 \texttt{2} 时将使用 \texttt{uint16\_t}(16 位整数)类型存储行号信息。它的默认值为 \texttt{1}。

将这个宏设置为 \texttt{0} 时不会因为存储文件名和行号信息,因此内存消耗最少。设置为 \texttt{2} 时消耗的内存也比较少,但是程序太长会使 \texttt{uint16\_t} 溢出。

\ffititle{BE\_USE\_PRECOMPILED\_OBJECT}

这个宏开关配置编译期构造对象的功能,打开这个宏就表示启用编译期构造对象,这个宏默认打开。打开这个宏时,标准库中的原生对象将在编译期生成,关闭这个宏会在运行时构建标准库中的对象。

\texttt{be\_regfunc} 和 \texttt{be\_regclass} 函数会受到这个宏的影响。使用编译期对象构造时不能修改内建对象表,此时这两个函数不能向 built-in 作用域中注册对象,而是向 global 作用域中注册对象。

编译期构造的对象和代码存放在一起,不会占用 RAM (或是内存中可读写的区域)资源,编译期构造技术还能减少解释器的启动时间,因此建议打开这个宏。有关编译期构造技术的更多细节请参考\ref{section::precompiled_build}节。

\ffititle{BE\_STACK\_TOTAL\_MAX}

这个宏定义了最大的 Berry 堆栈容量,该容量是指 Berry 对象的数量。当 Berry 代码使用了超过此数量的堆栈时将会停止执行程序并返回错误信息。这个宏的默认值是 \texttt{2000},可以根据系统的内存容量修改这个值。

这个值不会影响 Berry 堆栈的内存使用量,因为 Berry 堆栈的容量是动态调节的,因此无论将它设置为多少都不能帮助减少内存使用。它的主要作用是在 Berry 程序因为消耗过多的堆栈时终止执行。这些程序很有可能是不正确的,例如没有返回条件的递归函数调用就会不断地消耗堆栈。

\ffititle{BE\_STACK\_FREE\_MIN}

这个宏定义了 Berry 堆栈最少的可用空间,其默认值为 \texttt{10}。原生函数可能会往 Berry 堆栈中压入值,此时堆栈不会自动增长,因此要保证堆栈中有足够的空位供原生函数使用。不建议修改这个值,而是在确实需要更多的栈空间的位置使用 \texttt{be\_stack\_require} 函数。

为了在调试解释器时检测栈溢出错误,可以打开 \texttt{BE\_DEBUG} 宏(\ref{section::BE_DEBUG}节)。

\ffititle{BE\_STR\_HASH\_CACHE}

这个宏打开时,短字符串对象会保存字符串的 Hash 值以提高运行速度,但是每个字符串对象的尺寸将增加 4 个字节。这个宏默认关闭,目前的测试也没有发现打开该宏会带来显著提升。

\ffititle{BE\_USE\_STRING\_MODULE}

这个宏开关用于启用或关闭 \texttt{string} 模块,默认打开。

\ffititle{BE\_USE\_JSON\_MODULE}

这个宏开关用于启用或关闭 \texttt{json} 模块,默认打开。

\ffititle{BE\_USE\_MATH\_MODULE}

这个宏开关用于启用或关闭 \texttt{math} 模块,默认打开。

\ffititle{BE\_USE\_TIME\_MODULE}

这个宏开关用于启用或关闭 \texttt{time} 模块,默认打开。

\ffititle{BE\_USE\_OS\_MODULE}

这个宏开关用于启用或关闭 \texttt{os} 模块,默认打开。

\ffititle{BE\_EXPLICIT\_ABORT}

这个宏决定 Berry 解释器内部使用的 \texttt{abort} 函数,在默认情况下或者该宏未定义时将使用 C 标准库中的 \texttt{abort} 函数,该宏默认被定义为 \texttt{abort}。如果用户需要明确指定解释器所用的 \texttt{abort} 函数,则将该宏定义替换为用户所需的函数,此函数要和标准库中的 \texttt{abort} 函数声明形式相同。

\ffititle{BE\_EXPLICIT\_EXIT}

这个宏决定 Berry 解释器内部使用的 \texttt{exit} 函数,在默认情况下或者该宏未定义时将使用 C 标准库中的 \texttt{exit} 函数,该宏默认被定义为 \texttt{exit}。如果用户需要明确指定解释器所用的 \texttt{exit} 函数,则将该宏定义替换为用户所需的函数,此函数要和标准库中的 \texttt{exit} 函数声明形式相同。

\ffititle{BE\_EXPLICIT\_MALLOC}

这个宏决定 Berry 解释器内部使用的 \texttt{malloc} 函数,在默认情况下或者该宏未定义时将使用 C 标准库中的 \texttt{malloc} 函数,该宏默认被定义为 \texttt{malloc}。如果用户需要明确指定解释器所用的 \texttt{malloc} 函数,则将该宏定义替换为用户所需的函数,此函数要和标准库中的 \texttt{malloc} 函数声明形式相同。

\ffititle{BE\_EXPLICIT\_FREE}

这个宏决定 Berry 解释器内部使用的 \texttt{free} 函数,在默认情况下或者该宏未定义时将使用 C 标准库中的 \texttt{free} 函数,该宏默认被定义为 \texttt{free}。如果用户需要明确指定解释器所用的 \texttt{free} 函数,则将该宏定义替换为用户所需的函数,此函数要和标准库中的 \texttt{free} 函数声明形式相同。

\ffititle{BE\_EXPLICIT\_REALLOC}

这个宏决定 Berry 解释器内部使用的 \texttt{realloc} 函数,在默认情况下或者该宏未定义时将使用 C 标准库中的 \texttt{realloc} 函数,该宏默认被定义为 \texttt{realloc}。如果用户需要明确指定解释器所用的 \texttt{realloc} 函数,则将该宏定义替换为用户所需的函数,此函数要和标准库中的 \texttt{realloc} 函数声明形式相同。

\ffititle{be\_assert}

该宏用于定义断言函数的实现,默认情况下使用 C 标准库中的 \texttt{assert} 函数来实现断言。如果目标系统不方便使用标准库中的 \texttt{assert()} 函数进行断言则可以修改 \texttt{be\_assert} 宏的定义。一个正确的断言函数应该使用如下声明:

\begin{lstlisting}[language=c, numbers=none]
void assert(int condition);
\end{lstlisting}

其中,\texttt{condition} 是断言条件,该条件不满足时将会输出一条错误信息并终止程序运行。当然,“断言”函数通常使用一个宏来实现。

\section{\textsl{berry\_port.c} 文件}

这个文件实现了 Berry 解释器的底层 IO 函数,包括标准输入输出和文件系统的支持。\textsl{default} 目录下的 \textsl{berry\_port.c} 文件包含了一套可移植的 IO 支持,其中文件操作和标准输入输出使用 C 标准库中的 API 实现,路径和文件夹操作同时支持 Windows 和 POSIX 标准的 API。该文件还实现了一套基于 FatFs 的 IO 操作函数可供用户直接使用。如果需要在其他环境下使用 Berry 解释器,那么这些函数必须另行实现(可能只需要实现一部分)。

本节将介绍 \textsl{berry\_port.c} 文件中所实现函数的功能,并指导用户实现自己的版本。

\ffititle{be\_writebuffer}

\begin{lstlisting}[language=c, numbers=none]
void be_writebuffer(const char *buffer, size_t length);
\end{lstlisting}

该函数用于向标准输出设备输出一段数据,参数 \texttt{buffer} 为输出数据块的首地址,\texttt{length} 为输出数据块的长度。这个函数默认向 \texttt{stdout} 文件进行输出。在解释器内部,该函数通常被用作字符流输出,而不是二进制流。

\texttt{be\_writebuffer} 函数用途非常广泛,因此必须被实现。

\ffititle{be\_readstring}

\begin{lstlisting}[language=c, numbers=none]
char* be_readstring(char *buffer, size_t size);
\end{lstlisting}

从标准输入设备中输入一段数据,每调用一次该函数最多读取一行数据。参数 \texttt{buffer} 为调用方传入的数据缓冲区,该缓冲区的容量为 \texttt{size}。该函数会在缓冲区容量被用完时停止读取并返回,否则将在读取到换行符或者文件结束符时返回。如果函数执行成功,它将直接把 \texttt{buffer} 参数作为返回值,否则返回 \texttt{NULL}。

该函数会将读取到的换行符加入到读取数据中,每次调用 \texttt{be\_readstring} 函数都会从当前位置继续读取。该函数只在原生函数 \texttt{input} 的实现中被调用,在不必要等情况下可以不实现 \texttt{be\_readstring} 函数。

\ffititle{be\_fopen}

\begin{lstlisting}[language=c, numbers=none]
void* be_fopen(const char *filename, const char *modes);
\end{lstlisting}

打开一个文件,\texttt{filename} 为要打开的文件名,\texttt{modes} 为打开方式,函数将返回一个文件句柄或者文件操作结构的指针。该函数的使用方式和 C 标准库中的 \texttt{fopen} 函数相似。文件名是一个 C 风格的字符串(以 \texttt{\textbackslash 0} 字符结尾),而模式至少应该支持以下情况:

\begin{itemize}
    \item \texttt{r}、\texttt{rt}:以只读方式打开一个文本文件,文件必须存在。
    \item \texttt{r+}、\texttt{rt+}:以读写方式打开一个文本文件,如果文件不存在则创建一个新文件。
    \item \texttt{rb}:以只读方式打开一个二进制文件,文件必须存在。
    \item \texttt{rb+}:以读写方式打开一个二进制文件,如果文件不存在则创建一个新文件。
    \item \texttt{w}、\texttt{wt}:以只写方式新建并打开一个文本文件,已存在的文件将被删除。
    \item \texttt{w+}、\texttt{wt+}:以读写方式新建并打开一个文本文件,已存在的文件将被删除。
    \item \texttt{wb}:以只写方式新建并打开一个二进制文件,已存在的文件将被删除。
    \item \texttt{wb+}:以读写方式新建并打开一个二进制文件,已存在的文件将被删除。
\end{itemize}

默认情况下使用 C 标准库中的 \texttt{fopen} 函数来实现 \texttt{be\_fopen}。如果使用其他方式来实现,则应该保证以上操作模式能够被实现。如果不需要任何文件操作则此函数可以留空,这里的文件操作包括在脚本中使用 \texttt{open} 函数、从文件中加载脚本(使用 \texttt{be\_loadfile} 函数)等在内的所有情景。

\ffititle{be\_fclose}

\begin{lstlisting}[language=c, numbers=none]
int be_fclose(void *hfile);
\end{lstlisting}

关闭一个文件,\texttt{hfile} 为被关闭的文件句柄。该函数的功能类似于 C 标准库中的 \texttt{fclose} 函数。

\ffititle{be\_fwrite}

\begin{lstlisting}[language=c, numbers=none]
size_t be_fwrite(void *hfile, const void *buffer, size_t length);
\end{lstlisting}

向指定的文件中写入一段数据。参数 \texttt{hfile} 为待写入的文件句柄,\texttt{buffer} 为写入数据的指针,\texttt{length} 为待写入数据的数量(单位为字节)。

\ffititle{be\_fread}

\begin{lstlisting}[language=c, numbers=none]
size_t be_fread(void *hfile, void *buffer, size_t length);
\end{lstlisting}

从指定的文件中读取一段数据。参数 \texttt{hfile} 为待读取的文件句柄,\texttt{buffer} 为读取缓冲区的指针,\texttt{length} 为需要读取的字节数。

\ffititle{be\_fgets}

\begin{lstlisting}[language=c, numbers=none]
char* be_fgets(void *hfile, void *buffer, int size);
\end{lstlisting}

从文件中读取一行,类似于 C 标准库中的 \texttt{fgets} 函数。参数 \texttt{hfile} 为待读取的文件句柄,\texttt{buffer} 为读取缓冲区的指针,\texttt{size} 为读取缓冲区的容量。当读取到 \texttt{size - 1} 字节、换行符以及文件结束符时该函数会返回,返回值为 \texttt{buffer}。

\ffititle{be\_fseek}

\begin{lstlisting}[language=c, numbers=none]
int be_fseek(void *hfile, long offset);
\end{lstlisting}

设置文件读写指针的位置。参数 \texttt{hfile} 为待操作为文件句柄,\texttt{offset} 为需要设置的值。

\ffititle{be\_ftell}

\begin{lstlisting}[language=c, numbers=none]
long int be_ftell(void *hfile);
\end{lstlisting}

获得文件当前的读写指针,参数 \texttt{hfile} 为待操作文件的句柄,该函数的返回值为该文件的读写指针。

\ffititle{be\_fflush}

\begin{lstlisting}[language=c, numbers=none]
long int be_fflush(void *hfile);
\end{lstlisting}

将文件缓冲区中的数据写入到文件中。参数 \texttt{hfile} 为待操作的文件。

\ffititle{be\_fsize}

\begin{lstlisting}[language=c, numbers=none]
size_t be_fsize(void *hfile);
\end{lstlisting}

获取文件的大小。参数 \texttt{hfile} 为待操作的文件。
