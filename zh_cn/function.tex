\chapter{函数}

\textbf{函数}(function)是一种可以被外部代码调用的``子程序'',作为程序的一部分,函数本身也是一段代码。函数可以具有0到多个参数,并且会返回一个结果,这个结果称为函数的\textbf{返回值}。

在Berry中,函数是\textbf{第一类值}(first class value)。因此,除了调用函数以外,你还可以把函数作为值传递,例如将函数绑定到变量,将函数作为返回值等等。

\section{基本信息}

函数的使用包括函数的定义和调用两个部分。函数定义语句使用 \texttt{def} 关键字作为开头,函数定义是将函数体的代码打包并命名的过程,这个过程仅仅生成函数结构而不会执行函数。执行函数须使用\textbf{调用运算符}(call operator),该运算符是一对圆括号。调用运算符作用于一个结果是函数类型的表达式,传给函数的参数写在圆括号之内,多个参数之间使用逗号隔开。调用表达式的结果就是函数的返回值。

\subsection{函数定义}

\subsubsection{具名函数}

\textbf{具名函数}(named function)是定义时赋予了名字的函数,其定义语句由以下几部分构成:\texttt{def} 关键字、函数名、由0到多个参数(parameter)组成的列表以及函数体(function body),参数列表中的多个参数以逗号分隔,所有参数写在一对圆括号中。我们把函数定义时的参数称为\textbf{形参},而调用函数时的参数称为\textbf{实参}。函数定义的一般形式为:
\begin{algorithm}
    \texttt{def} $\bm{name}$ \texttt{(} $\bm{arguments}$ \texttt{)} \\
        \qquad $\bm{block}$ \\
    \texttt{end}
\end{algorithm}\vspace{-0.6em}\\
函数名$\bm{name}$是一个标识符;$\bm{arguments}$为形参列表;$\bm{block}$为函数体,如果函数体为空语句则函数被称为``空函数''。函数返回值语句包含在函数体中,如果$\bm{block}$中没有返回语句,函数默认会返回 \texttt{nil} 返回值。函数名实际上是绑定函数对象的变量名称,如果当前的作用域已经存在这个名称,定义函数相当于把函数对象绑定到这个变量。

下面的例子定义了一个名为 \texttt{add} 的函数,该函数的功能是求两个数的和并返回。
\begin{lstlisting}[language=berry, numbers=none]
def add(a, b)
    return a + b
end
\end{lstlisting}
\texttt{add} 函数具有两个参数 \texttt{a} 和 \texttt{b},两个被加数便通过这些参数传入函数进行计算。\texttt{return} 语句会返回计算的结果。

作为类属性的函数称为方法,这部分内容会在面向对象章节中说明。

\subsubsection{匿名函数}

与具名函数不同,\textbf{匿名函数}(anonymous function)没有名字,其定义表达式的形式为:
\begin{algorithm}
    \texttt{def} \texttt{(} $\bm{arguments}$ \texttt{)} \\
        \qquad $\bm{block}$ \\
    \texttt{end}
\end{algorithm}\vspace{-0.6em}\\
可以看出,与具名函数相比,匿名函数的定义中没有函数名$\bm{name}$。匿名函数的定义实质上是一个表达式,该表达式称为\textbf{函数字面值}。为了使用匿名函数,我们可以将函数字面值绑定到一个变量:
\begin{lstlisting}[language=berry, numbers=none]
add = def (a, b)
    return a + b
end
\end{lstlisting}
这段代码的功能和上一小节中 \texttt{add} 函数的功能完全相同。使用匿名函数可以方便地以字面值的形式传递函数值。与其他类型的字面量一样,函数字面量也是表达式的最小单元,因此从 \texttt{def} 关键字之间 \texttt{end} 之间是一个不可分割的整体。

\subsection{调用函数}

以 \texttt{add} 函数为例,调用该函数需要提供两个数值,通过调用函数可以得到两数之和:
\begin{lstlisting}[language=berry, numbers=none]
res = add(5, 3)
print(res)      # 8
\end{lstlisting}
我们把被调用的函数(例中的 \texttt{add} 函数)称为\textbf{被调函数},而调用被调函数的函数(例中为 \texttt{main} 函数)称为\textbf{主调函数}。函数调用的过程为:首先解释器会(隐式地)使用实参列表初始化被调函数的形参列表,同时暂停主调函数并保存其状态,接下来为被调函数创建环境并执行被调函数。

函数会在遇到 \texttt{return} 语句时结束执行并将返回值传递给主调函数。解释器会在被调函数返回后销毁被调函数的环境,然后恢复主调函数的环境并继续执行主调函数。函数的返回值也是函数调用表达式的结果。

下面的例子定义了一个函数 \texttt{square} 并把这个函数绑定到变量 \texttt{f},然后通过变量 \texttt{f} 来调用 \texttt{square} 函数函数。这种用法类似于C语言的函数指针。
\begin{lstlisting}[language=berry, numbers=none]
def square(n)
    return n * n
end
f = square
print(f(5))     # 25
\end{lstlisting}
需要注意的是,函数对象只是绑定到这些变量(参考\ref{section::assign_operator}节)并且不可修改,因此对函数名对应的变量重新赋值并不会使这个函数丢失:
\begin{lstlisting}[language=berry, numbers=none]
f = square
square = nil
print(f(5))     # 25
\end{lstlisting}
可以看到对 \texttt{square} 重新赋值后函数依然能正常调用。只有函数对象不再与任何变量绑定之后才会丢失,这类函数对象占用的资源将会被系统回收。

\subsubsection{前向调用}

函数的调用必须位于函数变量的作用域内,因此在函数定义之前通常不能调用。为了解决这个问题可以使用这种方法来折衷:
\begin{lstlisting}[language=berry]
var func1
def func2(x)
    return func1(x)
end
def func1(x)
    return x * x
end
print(func2(4))     # 16
\end{lstlisting}
在这个示例中, \texttt{func2} 调用了 \texttt{func1},而函数 \texttt{func1} 的定义却在 \texttt{func2} 之后。执行这段代码后,程序将会输出正确结果 \texttt{16}。这个例程利用了函数定义时不会被调用的机制,在定义 \texttt{func2} 之前先定义变量 \texttt{func1},这样可以保证编译时不会找不到符号 \texttt{func1}。然后我们在 \texttt{func2} 之后定义函数 \texttt{func1},这样会使用该函数来覆盖变量 \texttt{func1} 的值。最后一行 \texttt{print(func2(4))} 中调用函数 \texttt{func2} 时,变量 \texttt{func1} 已经是我们需要的函数,因此会输出正确结果。

\subsubsection{递归调用}

\textbf{递归函数}是指会直接或者间接调用自身的函数。递归是指一种将问题划分为同类子问题然后解决的策略。以阶乘为例,阶乘的递归定义为$0!=1, n!=n\cdot(n-1)!$,我们可以根据定义写出用于计算阶乘的递归函数:
\begin{lstlisting}[language=berry]
def fact(n)
    if (n == 0)
        return 1
    end
    return n * fact(n - 1)
end
\end{lstlisting}
以$5$的阶乘为例,手工计算5的阶乘的过程为:
\begin{equation*}
5! = 5 \times 4 \times 3 \times 2 \times 1 = 120
\end{equation*}
调用 \texttt{fact} 函数得到结果也是$120$:
\begin{lstlisting}[language=berry, numbers=none]
print(fact(5))  # 120
\end{lstlisting}

为了保证递归调用的深度有限(递归层次过深会耗尽栈空间),递归函数必须有一个结束条件。\texttt{fact} 函数定义中第2行的 \texttt{if} 语句用于结束条件的检测,当计算到 \texttt{n} 为 \texttt{0} 时递归过程结束。上述阶乘公式不适用于非整数参数,执行类似 \texttt{fact(5.1)} 的表达式将会因无法结束递归而发生栈溢出错误。

还有一种情况是 \texttt{间接递归},也就是函数不是由它自己调用而是由它调用的另一个函数(直接或间接)调用。间接递归时通常需要使用函数前向调用的技巧,以计算奇数和偶数的函数 \texttt{is\_odd} 和 \texttt{is\_even} 函数为例:
\begin{lstlisting}[language=berry]
var is_odd
def is_even(n)
    if (n == 0)
        return true
    end
    return is_odd(n - 1) 
end
def is_odd(n)
    if (n == 0)
        return false
    end
    return is_even(n - 1) 
end
\end{lstlisting}
这两个函数互相调用了对方。为了保证第6行调用 \texttt{is\_odd} 函数时作用域中有这个名称,在第1行定义了变量 \texttt{is\_odd}。

\subsubsection{匿名函数调用}

如果一个匿名函数只会被调用一次,最简单的办法就是在定义的时候调用,例如:
\begin{lstlisting}[language=berry, numbers=none]
res = def (a, b) return a + b end (1, 2) # 3
\end{lstlisting}
在这个例程中,我们在函数字面值后面直接使用调用表达式来调用函数。这种用法很适合于那种只会在一个位置进行调用的函数。

还可以把匿名函数绑定到变量之后调用:
\begin{lstlisting}[language=berry, numbers=none]
add = def (a, b) return a + b end
res = add(1, 2) # 3
\end{lstlisting}
这种用法与具名函数的调用类似,本质上都是对绑定了函数值的变量执行调用。需要注意的是,对匿名函数进行递归调用会比较困难,除非你使用前向调用的技巧。

\subsection{形参和实参}

函数在调用时会使用实参来初始化形参。通常情况下,实参和形参数量相等且位置一一对应,不过Berry也允许实参数量不等于形参:如果实参数量多余形参则多出的实参会被丢弃,如果实参数量少于形参则会把余下的形参初始化为 \texttt{nil}。

参数传递的过程与赋值运算相似。对于 \texttt{nil}、\texttt{boolean} 和数值类型,参数传递是值传递,而其他类型是传递引用。对于实例这种可写的传引用类型,在被调函数中修改它们也会修改主调函数中的对象。下面的例子展示了这个特性:
\begin{lstlisting}[language=berry]
var l = [], i = 0
def func(a, b)
    a.append(1)
    b = 'string'
end
func(l, i)
print(l, i)     # [1] 0
\end{lstlisting}
可以看到,调用函数 \texttt{func} 之后变量 \texttt{l} 的值发生了变化,而变量 \texttt{i} 的值没有变化。

\subsection{函数和局部变量}

函数体本身是一个作用域,因此在函数中定义的变量都是局部变量(参考\ref{section::scope_life}节)。和直接嵌套的块不同,每一次调用函数都会为局部变量分配空间。局部变量的空间在栈中分配,并且分配信息是在编译期确定的,因此这个过程非常快。当函数中嵌套有多层作用域时,解释器会为依据局部变量最多的作用域嵌套链来分配栈空间,而不是依据函数中局部变量的总数。

\subsection{\texttt{return} 语句}

\texttt{return} 语句用于返回函数的结果,也就是函数的返回值。Berry中的所有函数都具有返回值,但是你可以在函数体中不使用任何 \texttt{return} 语句,此时解释器会生成一条默认的 \texttt{return} 语句以保证函数的返回。\texttt{return} 语句有两种写法:
\begin{algorithm}
    \texttt{return} \\
    \texttt{return }$\bm{expression}$
\end{algorithm}\vspace{-0.6em}\\
第一种写法是只写出 \texttt{return} 关键字而不写要返回的表达式,这种情况下返回默认的 \texttt{nil} 值。第二种写法是在 \texttt{return} 关键字后面跟随表达式$\bm{expression}$,此时会把该表达式的值作为函数的返回值。程序执行到 \texttt{return} 语句时,当前运行的函数会结束执行并返回到调用该函数的代码中继续运行。

当使用单独的关键字 \texttt{return} 作为函数的返回语句时,容易引起二义性的问题,此时建议在 \texttt{return} 后面加分号来防止出现错误:
\begin{lstlisting}[language=berry, numbers=none]
def func()
    return;
    x = 1
end
\end{lstlisting}
在这个例子里,\texttt{return} 语句后的 \texttt{x = 1} 语句不会得到执行,因此是多余的。如果避免这种冗余的代码,\texttt{return} 语句后面通常会跟随 \texttt{end}、\texttt{else} 或者 \texttt{elif} 等关键字,这种情况即使使用单独的 \texttt{return} 语句时也不用担心出现歧义。

\section{闭包}

\subsection{基础概念}

前面已经提到,函数在 Berry 中是第一类值,你可以在任何地方定义函数,也可以把函数作为参数或者返回值进行传递。在函数中定义另一个函数时,嵌套定义的函数可以访问任何外层函数的局部变量。我们把函数中使用的``外层函数的局部变量''称为函数的\textbf{自由变量},广义的自由变量也包括全局变量,但是在 Berry 中没有这项规则。

\textbf{闭包}是将函数和\textbf{环境}绑定的一种技术。环境是一个映射,它将函数的每个自由变量与一个值相关联。在实现上,闭包会将函数原型与自有变量关联存储。函数原型在编译期生成,而环境是一个运行时的概念,因此闭包也是在运行时动态生成的。每个闭包都是在生成时将函数原型和环境进行绑定,例如在下面的这个例子中:
\begin{lstlisting}[language=berry]
def func(i)     # The outer function
    def foo()   # The inner function (closure)
        print(i)
    end
    foo()
end
\end{lstlisting}
内层函数 \texttt{foo} 是一个闭包,它有一个自由变量 \texttt{i},该变量是外层函数 \texttt{func} 的一个参数。在生成闭包 \texttt{foo} 时会将它的函数原型和包含了自由变量 \texttt{i} 的环境绑定,当变量 \texttt{foo} 离开作用域以后闭包会被销毁。通常,内层函数会作为外层函数的返回值,例如:
\begin{lstlisting}[language=berry]
def func(i)         # The outer function
    return def ()   # Return a closure (anonymous function)
        print(i)
        i = i + 1
    end
end
\end{lstlisting}
这里返回的闭包是一个匿名函数。当闭包被外层函数返回以后,外层函数的局部变量会被销毁,闭包将不能直接访问原来外层函数中的变量。系统会在自由变量被销毁时将自由变量的值复制到环境中。这些自由变量的生命周期和闭包一致,并且只能由闭包访问。返回的函数或闭包并不会自动执行,因此我们要调用 \texttt{func} 函数返回的闭包:
\begin{lstlisting}[language=berry]
f = func(0)
f()
\end{lstlisting}
这段代码会输出 \texttt{0},如果我们继续调用闭包 \texttt{f},将会得到输出 \texttt{1}, \texttt{2}, \texttt{3}\ldots\ 这可能不是很好理解:变量 \texttt{i} 在函数 \texttt{func} 返回后被销毁,而作为闭包 \texttt{f} 的自由变量,\texttt{i} 会保存到闭包的环境中,因此每次调用 \texttt{f} 都会使 \texttt{i} 的值加 $1$(\texttt{func} 函数定义的第 4 行)。

\subsubsection{闭包的用途}

闭包有很多用途,这里介绍几种常见的用途:

\paragraph{惰性求值}

闭包在被调用前不会做任何事情。

\paragraph{函数的私有通信}

可以让一些闭包共用自由变量,这些自由变量仅对这些闭包可见,通过改变这些自由变量的值来进行函数间的通信。这样做可以避免使用外部变量。

\paragraph{生成多个函数}

有时候我们可能需要使用多个函数,这些函数可能只是一些变量的取值有所不同。我们可以实现一个函数,然后将这些不同的变量作为函数参数。更好的办法是通过一个工厂函数来返回闭包,并且把这些可能不同的变量作为闭包的自由变量,这样在调用函数时不必总是要写那些参数,而且可以生成任意数量的同类函数。

\paragraph{模拟私有成员}

有些语言支持在对象中使用私有成员,而 Berry 的类不支持私有成员。我们可以使用闭包的自由变量来模拟私有成员。这种用途并不是设计闭包的本意,不过在现在,这种对闭包的``误用''已经十分常见。

\paragraph{缓存结果}

如果有一个运行非常耗时的函数,那么每次调用它都会花费很多时间。我们可以缓存这个函数的结果,调用函数前先在缓存中查找,如果找到了就返回缓存值,否则调用函数并更新缓存值。我们可以利用闭包来保存缓存值,这样不会将它暴露到外层作用域,而缓存的结果也会被被保留(直到闭包销毁)。

\subsection{绑定自由变量}

如果多个闭包绑定了同一个自由变量,所有闭包将始终公用这个自由变量。例如:
\begin{lstlisting}[language=berry]
def func(i)     # The outer function
    return [    # Return a closure list
        def ()  # The closure #1
            print("closure 1 log:", i)
            i = i + 1
        end,
        def ()  # The closure #2
            print("closure 2 log:", i)
            i = i + 1
        end
    ]
end
\end{lstlisting}
这个例子里的 \texttt{func} 函数将两个闭包通过一个列表来返回,这两个闭包公用自由变量 \texttt{i}。如果我们调用这些闭包:
\begin{lstlisting}[language=berry]
f = func(0)
f[0]() # closure 1 log: 0
f[1]() # closure 2 log: 1
\end{lstlisting}
可以看到,我们调用闭包 \texttt{f[0]} 时更新了自由变量 \texttt{i},而这个变化影响了调用闭包 \texttt{f[1]} 的结果。这是由于如果一个自由变量被多个闭包使用,该自由变量也只有一份副本,所有的闭包都有一个对该自由变量实体的引用。因此,任何对该自由变量的修改都对所有使用了该自由变量的闭包可见。

同理,在外层函数的局部变量没有销毁之前,修改自由变量的值也会影响到闭包:
\begin{lstlisting}[language=berry]
def func()
    i = 0
    def foo()
        print(i)
    end
    i = 1
    return foo
end
\end{lstlisting}
在这个例子中,我们在外层函数 \texttt{func} 返回之前把变量 \texttt{i}(它是闭包 \texttt{foo} 的自由变量)的值从 \texttt{0} 改成了 \texttt{1},那么我们在此后调用闭包 \texttt{foo} 时自由变量 \texttt{i} 的值也是 \texttt{1}:
\begin{lstlisting}[language=berry]
func()() # 1
\end{lstlisting}

\subsection{在循环中创建闭包}

在循环体中构造闭包时,你可能不希望闭包的自由变量跟着循环循环变量变化。我们先来看一个在 \texttt{while} 循环中创建闭包的例子:
\begin{lstlisting}[language=berry]
def func()
    l = [] i = 0
    while (i <= 2)
        l.append(def () print(i) end)
        i = i + 1
    end
    return l
end
\end{lstlisting}
这个例子中,我们在循环中构造闭包,并把这个闭包放在一个 \texttt{list} 中。很明显,当循环结束后,变量 \texttt{i} 的值将是 \texttt{3},而列表 \texttt{l} 中所有的闭包也是使用这个变量的引用。如果我们执行 \texttt{func} 返回的闭包将得到相同的的结果:
\begin{lstlisting}[language=berry]
res = func()
res[0]() # 3
res[1]() # 3
res[2]() # 3
\end{lstlisting}
如果我们希望每个闭包都引用不同的自由变量,我们可以再定义一层函数,然后用函数的参数绑定当前的循环变量:
\begin{lstlisting}[language=berry]
def func()
    l = [] i = 0
    while (i <= 2)
        l.append(def (n)
            return def () print(n) end
        end (i))
        i = i + 1
    end
    return l
end
\end{lstlisting}
为了帮助理解这段看起来很难理解的代码,我们重点说明第 4 到第 6 行的代码:
\begin{lstlisting}[language=berry]
def (n)
    return def ()
        print(n)
    end
end (i)
\end{lstlisting}
这里实际上是定义了一个匿名函数并且立即调用它,这个临时的匿名函数的作用是将循环变量 \texttt{i} 的值绑定到它的参数 \texttt{n},而变量 \texttt{n} 也是我们所需闭包的自由变量,这样在每次循环时所构造的闭包绑定的自由变量都不同。现在我们将得到期望的输出:
\begin{lstlisting}[language=berry]
res = func()
res[0]() # 0
res[1]() # 1
res[2]() # 2
\end{lstlisting}
还有一些办法可以解决循环变量作为自由变量的问题。稍微简单一点的办法是在循环体中定义一个临时变量:
\begin{lstlisting}[language=berry]
def func()
    l = [] i = 0
    while (i <= 2)
        temp = i
        l.append(def () print(temp) end)
        i = i + 1
    end
    return l
end
\end{lstlisting}
这里的 \texttt{temp} 就是临时变量,该变量的作用域在循环体中,所以每次循环都会重新定义。我们还可以使用 \texttt{for} 语句来解决问题:
\begin{lstlisting}[language=berry]
def func()
    l = []
    for (i : 0 .. 2)
        l.append(def () print(i) end)
    end
    return l
end
\end{lstlisting}
这可能是最简洁的办法。\texttt{for} 语句的迭代变量会在每次循环中创建,其原理和与前一个办法类似。
