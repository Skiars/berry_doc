\chapter{文法定义}

本章将给出与 Berry 有关的一些文法定义,我们使用\textbf{扩展的巴科斯范式}(Extended Backus Normal Form, EBNF)来定义或者表达文法。我们没有使用严格的 EBNF 语法来定义,而是做了很多简化,不过这些简化不会影响读者对文法的理解。

Berry 语言文法的 EBNF 定义如下:

\lstinputlisting[language=ebnf]{../ebnf/berry.ebnf}

标准的 EBNF 格式可以到相关的资料中查找,这里说明一下阅读上述文法需要注意的细节。在等号左边出现过的符号是非终结符,其他是终结符。使用引号 \texttt{'} 括起来的终结符是一个固定的串,它通常是语言的关键字或者操作符。有几个终结符不方便直接在 EBNF 中描述:\texttt{INTEGER} 表示整数字面值;\texttt{REAL} 表示实数字面值;\texttt{STRING} 表示字符串字面值;\texttt{ID} 表示标识符。这些终结符可以使用正则表达式定义:

\begin{itemize}
    \item \texttt{INTEGER}: \texttt{0x[a-fA-F0-9]+|\textbackslash d+}。
    \item \texttt{REAR}: \texttt{(\textbackslash d+\textbackslash.?|\textbackslash.\textbackslash d)\textbackslash d*([eE][+-]?\textbackslash d+)?}。
    \item \texttt{STRING}: \texttt{"(\textbackslash\textbackslash.|[\textasciicircum"])*"|'(\textbackslash\textbackslash.|[\textasciicircum'])*'}。
    \item \texttt{ID}: \texttt{[\_a-zA-Z]\textbackslash w*}
\end{itemize}

标准 EBNF 中顺序出现的符号使用逗号隔开,直观起见,我采用空格来实现逗号的功能。竖线符号``|''读作“或”,它表示其左右的模式只能匹配其中一个,或的优先级最低,例如文法 $a_0a_1|a_2$ 表示要么匹配式 $a_0a_1$ 要么匹配 $a_2$。方括号表示括号内部的子式被匹配0或1次,花括号表示内部的子式被匹配0到多次,圆括号则只有将内部的子式作为一个整体的功能。

以下是 Berry 标准库中 JSON 模块支持的 JSON 文法定义,EBNF 的用法依然遵守上文的约定:

\lstinputlisting[language=ebnf,morekeywords={string,number}]{../ebnf/json.ebnf}

非终结符 \texttt{string} 和 \texttt{number} 也可以使用正则表达式定义,\url{http://www.json.org} 给出了 JSON 的标准文法,其中也包括 \texttt{string} 和 \texttt{number} 的定义。Berry JSON 库对数字的支持与标准有所不同,标准的 JSON 数字必须以``-''或者数字``0-9''开头,而 Berry JSON 库还接受以小数点开头的数字。
