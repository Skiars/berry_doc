\chapter{文法定义}

本章将给出与 Berry 有关的一些文法定义,我们使用\textbf{扩展的巴科斯范式}(Extended Backus Normal Form, EBNF)来定义或者表达文法。我们没有使用严格的 EBNF 语法来定义,而是做了很多简化,不过这些简化不会影响读者对文法的理解。

Berry 语言文法的 EBNF 定义如下:

\lstinputlisting[language=ebnf]{../ebnf/grammar.ebnf}

\texttt{INTEGER} 是整数字面值,\texttt{REAL} 是实数字面值,\texttt{STRING} 是字符串字面值,\texttt{ID} 是标识符。这些词法元素可以使用正则表达式定义:

\begin{itemize}
    \item \texttt{INTEGER}: \texttt{0x[a-fA-F0-9]+|\textbackslash d+}。
    \item \texttt{REAR}: \texttt{(\textbackslash d+\textbackslash.?|\textbackslash.\textbackslash d)\textbackslash d*([eE][+-]?\textbackslash d+)?}。
    \item \texttt{STRING}: \texttt{"(\textbackslash\textbackslash.|[\textasciicircum"])*"|'(\textbackslash\textbackslash.|[\textasciicircum'])*'}。
    \item \texttt{ID}: \texttt{[\_a-zA-Z]\textbackslash w*}
\end{itemize}

以下是 Berry 标准库中 JSON 模块支持的 JSON 文法定义:

\lstinputlisting[language=ebnf,morekeywords={string,number}]{../ebnf/json.ebnf}

JSON 的 \texttt{string} 和 \texttt{number} 词法元素也可以使用正则表达式定义。
