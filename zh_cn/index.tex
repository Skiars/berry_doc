\documentclass[UTF8, oneside]{book}
    \usepackage{geometry}
    \usepackage{ctex}
    \usepackage{fancyhdr}
    \usepackage{makecell}
    \usepackage{multirow}
    \usepackage{listings}
    \usepackage{tikz}
    \usepackage{fontspec, xunicode, xltxtra}
    \usepackage{titlesec}
    \usepackage{float}
    \usepackage{caption}
    \usepackage{amsmath, amssymb}
    \usepackage[colorlinks, linkcolor=black, anchorcolor=black, citecolor=black]{hyperref}
    \usepackage[justification=centering]{caption}
    \usepackage{subcaption}
    \usepackage{color}
    \usepackage[super,square]{natbib}
    \usepackage{blindtext}
    \usepackage{bm}
    \usepackage{enumitem}
    \usepackage{graphicx}
    \usepackage{algorithm2e}
    \usepackage{ebgaramond}

    % 设置字体
    \setCJKmainfont[BoldFont=SimHei]{NSimSun}
    \setCJKmonofont{NSimSun}
    % 定义字体
    \setCJKfamilyfont{fzsh}{FZSongHei-B07S} % 方正宋黑
    \newcommand{\songhei}{\CJKfamily{fzsh}}
    % 定义字号
    \newcommand{\chuhao}{\fontsize{42pt}{54.6pt}\selectfont}
    \newcommand{\sanhao}{\fontsize{15.75pt}{20.5}\selectfont}

    % 页边距设置
    \geometry{left=3.18cm, right=3.18cm, top=2.54cm, bottom=2.54cm}

    % 设置标题和注解格式
    \titleformat{\chapter}{\centering\huge\bf\heiti}{\chinese{chapter}、}{0em}{}
    \titleformat{\section}{\Large\bf\heiti}{\thesection}{0.75em}{}
    \titleformat{\subsection}{\large\bf\heiti}{\thesubsection}{0.75em}{}
    \titlespacing*{\chapter} {0pt}{0pt}{20pt}
    %\captionsetup[figure]{name={图},labelsep=quad}
    \captionsetup[table]{name={表},labelsep=quad}
    % 标题样式重定义
    \renewcommand{\contentsname}{\centerline{\huge\bf\heiti 目\quad 录}}
    \renewcommand\bibname{\centerline{\normalsize\bf\heiti [参考文献]}}

    % 标题与上下文间隔
    \titlespacing*{\section}{0pt}{3ex}{1ex}
    \titlespacing*{\subsection}{0pt}{2ex}{1ex}
    \titlespacing*{\subsubsection}{0pt}{1ex}{0.5ex}

    % 列表行间距
    \setenumerate[1]{itemsep=1.5pt,partopsep=0pt,parsep=\parskip,topsep=5pt}
    \setitemize[1]{itemsep=1.5pt,partopsep=0pt,parsep=\parskip,topsep=5pt}
    \setdescription{itemsep=1.5pt,partopsep=0pt,parsep=\parskip,topsep=5pt}

    % 设置tikz
    \usetikzlibrary{arrows, automata}

    % 定义 PrintStyle 时使用打印样式
    % \def \PrintStyle {}

    % 设置代码样式
\ifx \PrintStyle \undefined
    \lstset{
        commentstyle=\small\color{green!50!black}\itshape,
        keywordstyle=\small\color{cyan!50!black}\bfseries,
        stringstyle=\small\color{purple},
        numberstyle=\small\ttfamily\color{gray},
    }
\else   % 打印用样式
    \lstset{
        commentstyle=\small\itshape,
        keywordstyle=\small\bfseries,
        numberstyle=\small\ttfamily,
    }
\fi
    % 通用代码样式
    \lstset{
        basicstyle=\small\ttfamily,
        columns=fullflexible,
        numbersep=1em,
        xleftmargin=\parindent,
        breakatwhitespace=false,
        numbers=left,
        captionpos=b,               % sets the caption-position to bottom
        breaklines=true,            % automatic line breaking only at whitespace
        keepspaces=true,
        showstringspaces=false,
        tabsize=4
    }

\begin{document}

    % 封面
    % 封面

%% temporary titles
% command to provide stretchy vertical space in proportion
\newcommand\nbvspace[1][3]{\vspace*{\stretch{#1}}}
% allow some slack to avoid under/overfull boxes
\newcommand\nbstretchyspace{\spaceskip0.5em plus 0.25em minus 0.25em}
% To improve spacing on titlepages
\newcommand{\nbtitlestretch}{\spaceskip0.6em}

% 长破折号
\newcommand{\cndash}{\raisebox{0.5mm}{------}}

\begin{titlepage}

    \title{\ebgaramond\Huge{\scshape The Berry Script Language\\Reference Manual}\\\Large{V0.0.7}}

    \author{官文亮}

    \maketitle

\end{titlepage}


    \chapter*{序\quad 言}

\pagestyle{empty}
\thispagestyle{empty}

几年以前我曾尝试过将Lua脚本语言移植到STM32F4单片机,后来我在ESP8266上体验过一款基于Lua的固件:NodeMCU,这些经历使我体验到了使用脚本开发的便利。后来我又接触了几款脚本语言,例如Python、JavaScript、Basic以及MATLAB等,在这些语言中只有少数是适合移植到单片机平台的。当时我比较关注Lua,因为它是一款定位非常精简的嵌入式脚本语言,其设计目标就是嵌入到宿主程序中使用。然而对于单片机而言Lua依然大了一点,我无法在存储器比较小的32位单片机上运行它。为此,我开始阅读Lua代码并在此基础上开发自己的脚本语言------Berry。

Berry是一款超轻量级的嵌入式脚本语言,它还是一款多范式的动态语言。支持面向对象、面向过程和函数式(支持比较少)风格。

Berry在很多地方都参考了Lua,但是它完全面向性能较低的嵌入式系统,这些系统可能只有几KB的RAM和不到64KB的ROM并且可能没有硬件浮点支持(即便支持通常也是单精度浮点而没有双精度浮点),因此要运行一个功能全面的脚本语言十分困难。为此,Berry不会试图去提供复杂的语法糖,而只是实现比较比较重要的功能。在精简设计的同时我尽可能满足功能的需求,使Berry成为一个功能全面的脚本语言。由于有Lua解释器作为模板,我的工作简单了很多,Berry的编译器和字节码参考了Lua的实现,这是一种对存储资源要求很低但是很高效的设计。

Berry解释器使用一个一趟式编译器把源代码编译成字节码,这种编译器不需要生成抽象语法树(AST)并且只需要读取一遍源代码就可以完成字节码的编译工作。字节码类似于机器语言,不过运行它的不是物理机器而是Berry虚拟机(VM)。Berry使用基于寄存器的虚拟机(还有一种基于堆栈的虚拟机),一般认为寄存器式的虚拟机要比堆栈式的虚拟机需要更少的字节码并且性能更好。当然,编译器和虚拟机对Berry源代码而言是透明的,本质上来说,一门语言并不依赖于解释器或者编译器的具体实现方式。

我是直到后来才了解到MicroPython项目,一个为单片机设计的Python解释器,该解释器十分精简,使得Python可以在单片机上运行。如今Python十分火热,而这款为单片机实现的Python解释器也非常受欢迎。而我没有使用现有语言的语法,这可能会影响到Berry的使用。不过Berry的语法十分简单,如果你有其他语言的基础,可能只要几个小时就能掌握。如果你需要移植Berry的解释器,需要保证你使用的编译器提供对C99标准的支持(我先前完全遵守C89,但是后来的一些优化工作使我放弃了这个决定),ARM处理器开发中常见ARMCC(KEIL MDK)、ICC(IAR)以及GCC等编译器都支持C99。

本文档介绍Berry的语法规则、标准库等设施,最后会指导读者去移植并扩展Berry。


    \clearpage

    \setcounter{page}{1} % 目录页码
    \pagenumbering{roman} % 罗马数字页码
    \setcounter{tocdepth}{2} % 目录深度
    \tableofcontents % 目录

    \newpage
    \setcounter{page}{1}    % 重置页码计数
    \pagestyle{fancy}
    \pagenumbering{arabic}  % 数字页码

    \chapter{基本信息}

\section{开始使用}

\subsection{获取解释器}

可以到Berry项目的GitHub页面(\url{https://github.com/gztss/berry})上获取Berry解释器的源代码。用户需要自行编译Berry的解释器,这些信息可以到项目主页的README.md文档中查看。

首先要要安装好GCC和git等工具,对于Linux和macOS系统,还要安装readline库。准备工作完成后使用\texttt{git}命令把解释器源代码从远程仓库克隆到本地:
\begin{lstlisting}[language=bash, numbers=none]
git clone https://github.com/gztss/berry.git
\end{lstlisting}
进入berry目录,使用\texttt{make}命令编译解释器:
\begin{lstlisting}[language=bash, numbers=none]
cd berry
make
\end{lstlisting}

现在应该能在berry目录下找到解释器的可执行文件(Windows下的文件名是``berry.exe'',在Linux和macOS下文件名是``berry''),你可以直接运行该可执行文件\footnote{在Windows中你可以直接双击运行可执行文件,在Linux或者macOS中通常要使用``终端''(Terminal)来运行。你也可以在Windows的``命令提示符''(cmd)窗口中运行解释器。具体的使用方法请参考README.md文档。}来启动解释器。在Linux或者macOS系统下可以使用命令\texttt{sudo make install}安装解释器,此后你可以在终端中输入\texttt{berry}来启动解释器。

\subsection{REPL环境}

直接启动解释器(在终端或命令窗口里输入\texttt{berry}而不带参数,或者在Windows中双击berry.exe)时会看到以下界面:
\begin{lstlisting}[language=berry, numbers=none]
Berry 0.0.4 (build in Feb  1 2019, 13:14:04)
[GCC 8.1.0] on Windows (default)
>
\end{lstlisting}
界面的前两行显示了Berry解释器的版本、编译时间、编译器和操作系统等信息,第三行的符号``\texttt{>}''叫做提示符,光标显示在提示符后面。

不带参数地启动解释器会进入REPL(Read Eval Print Loop)模式,也就是交互模式。在REPL模式下,我们可以输入源代码然后按下``Enter''键之后执行。

\subsubsection{Hello World程序}

以经典的``Hello World''程序为例,在REPL中输入\texttt{print('Hello World')}并执行,运行结果如下:
\begin{lstlisting}[language=berry, numbers=none]
Berry 0.0.4 (build in Feb  1 2019, 13:14:04)
[GCC 8.1.0] on Windows (default)
> print('Hello World')
Hello World
>
\end{lstlisting}
解释器输出了``\texttt{Hello World}''文本。这行代码通过调用\texttt{print}函数来实现对字符串\texttt{'Hello World'}的输出。在REPL中,如果表达式的返回值不是\texttt{nil}则会显示该返回值。例如输入表达式\texttt{1 + 2}会显示计算结果\texttt{3}:
\begin{lstlisting}[language=berry, numbers=none]
> 1 + 2
3
\end{lstlisting}

因此REPL下最简单的``Hello World''程序是直接输入字符串\texttt{'Hello World'}并运行:
\begin{lstlisting}[language=berry, numbers=none]
> 'Hello World'
Hello World
\end{lstlisting}

\subsubsection{REPL的更多用法}

你还可以把Berry解释器的交互模式当成一个科学计算器来使用,不过,一些数学函数不能直接使用,而要先使用\texttt{import math}语句来导入数学库,然后在使用数学库中的函数时要使用``\texttt{math.}''作为前缀,例如\texttt{sin}函数要写成\texttt{math.sin}:
\begin{lstlisting}[language=berry, numbers=none]
> import math
> math.pi
3.14159
> math.sin(math.pi / 2)
1
> math.sqrt(2)
1.41421
\end{lstlisting}

\subsection{脚本文件}

Berry的脚本文件是存储Berry代码的文件,脚本文件可以由解释器执行。通常情况下,脚本文件是扩展名为``.be''的文本文件。使用解释器执行脚本的命令是:
\begin{lstlisting}[language=bash, numbers=none]
berry script_file
\end{lstlisting}
\texttt{script\_file}是脚本文件的文件名。使用该命令会运行解释器执行\texttt{script\_file}脚本文件中的Berry代码,执行完毕后解释器会退出。

如果希望执行完脚本文件后解释器进入REPL模式可以在调用解释器的命令中加入\texttt{-i}参数:
\begin{lstlisting}[language=bash, numbers=none]
berry -i script_file
\end{lstlisting}
这条命令会先执行\texttt{script\_file}文件中的代码,然后进入REPL模式。

\section{词法}

在介绍Berry的语法之前,我们先来看一段简单的代码(你可以在REPL模式中运行这段代码):
\begin{lstlisting}[language=berry]
def func(x) # a function example
    return x + 1.5
end
print('func(10) =', func(10))
\end{lstlisting}

这段代码中定义了一个函数\texttt{func}并在后面调用了它。在了解这段代码怎样工作之前,我们先介绍Berry语言的语法元素。

以上代码中,语法元素的具体分类为:\texttt{def}、\texttt{return}和\texttt{end}是Berry语言的关键字;而第1行中的``\texttt{\# a function example}''被称为注释;\texttt{print}、\texttt{func}和\texttt{x}是一些标识符,它们通常用于表示一个变量;\texttt{1.5}和\texttt{10}这些数字被称为数值字面量,它们相当于日常生活中使用的数字;\texttt{'func(10) ='}是一个字符串字面量,他们大量用于需要表示文本的地方;\texttt{+}是一个加法运算符,这里使用加法运算符可以将变量\texttt{x}和数值\texttt{1.5}相加。

以上的分类实际上是从词法分析器的角度来做的。词法分析是Berry源代码解析的第一步,为了写出正确的源代码,我们先从最基础的词法开始介绍。

\subsection{注释}

注释是不会生成任何代码的一些文本,它们用于在源代码中做批注并给人们阅读,而编译器则不会解释它们的内容。Berry支持单行注释和跨行的块注释。单行注释从字符``\texttt{\#}'开始,直到换行字符结束。快注释从文本``\texttt{\#-}''开始,直到文本``\texttt{-\#}''结束。以下是使用注释的例子:
\begin{lstlisting}[language=berry, numbers=none]
# this is a line comment
#- this is a
   block comment
-#
\end{lstlisting}

和C语言类似,快注释不支持嵌套,以下代码将在第一个``\texttt{-\#}''文本处终止对注释的解析:
\begin{lstlisting}[language=berry, numbers=none]
#- some comments -# ... -#
\end{lstlisting}

\subsection{字面值}

字面值是编程时在源代码中直接写出的固定值。Berry的字面量有整数、实数、布尔量、字符串和nil。例如,数值\texttt{34}是一个整数字面值。

\subsubsection{数值字面值}

数值字面值包括\textbf{整数}(integer)字面值和\textbf{实数}(real)字面值。
\begin{lstlisting}[language=berry, numbers=none]
40      # integer
0x80    # hexadecimal literal (integer)
3.14    # real
1.1e-6  # real
\end{lstlisting}

数值字面值的写法和日常写法类似。Berry支持16进制的整数字面值,16进制字面值使用前缀\texttt{0x}或者\texttt{0X}开头,后面是一个16进制数。

\subsubsection{布尔字面值}

布尔值(boolean)用来表示逻辑状态中的真和假。你可以使用\texttt{true}和\texttt{false}这两个关键字来表示布尔字面值。

\subsubsection{字符串字面值}

字符串(string)是一段文本,它的字面值写法是使用一对\texttt{'}或\texttt{"}包围字符串文本:
\begin{lstlisting}[language=berry, numbers=none]
'this is a string'
"this is a string"
\end{lstlisting}

字符串字面值提供一些转义序列来表示不能直接输入的字符,转义序列以字符\texttt{'\textbackslash'}开始,然后紧跟一个特定的字符序列实现转义。Berry规定的转义序列有
\begin{table}[htb]
    \centering
    \setlength{\tabcolsep}{4mm}
    \begin{tabular}{clclcl} \Xhline{1pt}
        \makecell[c]{\textbf{转义字符}} & \makecell[l]{\textbf{意义}} & \makecell[c]{\textbf{转义字符}} & \makecell[l]{\textbf{意义}} & \makecell[c]{\textbf{转义字符}} & \makecell[l]{\textbf{意义}} \\ \Xhline{1pt}
        \texttt{\textbackslash a} & 响铃 & \texttt{\textbackslash b} & 退格符 & \texttt{\textbackslash f} & 换页符 \\
        \texttt{\textbackslash n} & 换行符 & \texttt{\textbackslash r} & 回车符 & \texttt{\textbackslash t} & 水平制表符 \\
        \texttt{\textbackslash v} & 垂直制表符 & \texttt{\textbackslash \textbackslash} & 反斜线 & \texttt{\textbackslash '} & 单引号 \\
        \texttt{\textbackslash "} & 双引号 & \texttt{\textbackslash ?} & 问号 & \texttt{\textbackslash 0} & 空字符 \\
        \Xhline{1pt}
    \end{tabular}
    \caption{转义字符序列}
    \label{tab::escape_character}
\end{table}

转义序列可以在字符串中使用,例如
\begin{lstlisting}[language=berry, numbers=none]
print('escape character LF\n\tnew line')
\end{lstlisting}
的运行结果是
\begin{lstlisting}[numbers=none]
escape character LF
        new line
\end{lstlisting}

还可以使用泛化的转义序列,其形式是\texttt{\textbackslash x}后跟2个十六进制数,或者是\texttt{\textbackslash}3个八进制数,使用这种转义序列可以表示任意字符。下面是使用ASCII字符集的一些例子:
\begin{lstlisting}[language=berry, numbers=none]
'\115' #- 'M' -#    '\x34' #- '4' -#    '\064' #- '4' -#
\end{lstlisting}

\subsubsection{Nil字面值}

Nil表示空值,其字面值使用关键字\texttt{nil}来表示。

\subsection{标识符} \label{section:identifier}

标识符(identifier)是由用户定义的名字,它由下划线或者字母作为开头,再由若干个下划线、字母或者数字的组合构成。和大多数语言类似,Berry是大小写敏感的,因此标识符\texttt{A}和标识符\texttt{a}会解析为两种标识符。
\begin{lstlisting}[language=berry, numbers=none]
a
TestVariable
Test_Var
_init
baseCass
_
\end{lstlisting}

\subsection{关键字}

Berry保留以下记号作为语言的关键字:
\begin{lstlisting}[language=berry, numbers=none]
    if          elif        else        while       for         def
    end         class       break       continue    return      true
    false       nil         var         do          import      as
\end{lstlisting}

关键字的具体使用方法会在相关的章节中介绍。注意,不能将关键字作为标识符使用,由于Berry是大小写敏感的,因此\texttt{If}可以用于标识符。

    \chapter{表达式}

\section{基础}

一个表达式(Statement)由一个到多个操作数和运算符组成,通过对表达式求值可以得到一个结果,这个结果被称为表达式的值。操作数可以是一个字面值、变量、函数调用或者是子表达式等。使用简单的表达式和运算符也可以组合成较为复杂的表达式。与四则运算类似,运算符的优先级会影响表达式的求值顺序,优先级越高的运算符,其表达式越先求值。

\subsection{运算符和表达式}

Berry提供一些一元运算符(Unary Operator)和二元运算符(Binary Operator)。例如逻辑与运算符 \texttt{\&\&} 就是一个二元运算符,而逻辑非运算符 \texttt{!} 是一个一元运算符。一些运算符既可以是一元运算符也可以是二元运算符,这种运算符的具体含义要通过上下文来决定。例如运算符 \texttt{-} 在表达式 \texttt{-1} 中是一元符号,但是在表达式 \texttt{1-2} 中则是二元的减号。

\subsubsection{运算符组合表达式}

二元运算符左右两边都可以是子表达式,因此可以使用二元运算符来组合表达式。一个较为复杂的表达式往往具有多种运算符和操作数。这时候,表达式中各个子表达式的求值顺序(order of evaluation)可能会影响到表达式的值。而运算符的优先级(precedence)和结合性(associativity)保证了表达式求值顺序的唯一性。例如一个组合表达式:
\begin{lstlisting}[language=berry, numbers=none]
1 + 10 / 2 * 3
\end{lstlisting}
日常使用的四则运算会先计算除法表达式 \texttt{10/2},然后计算乘法表达式,最后计算加法表达式。这是因为乘法和除法比加法具有更高的优先级。

\subsubsection{操作数类型}

在表达式的运算中,操作数可能和运算符具有不匹配的类型,另外二元运算符通常要求左右操作数是同一种类型。表达式 \texttt{'10'+10} 是错误的,你无法对一个字符串和一个整数做加法,而表达式 \texttt{-'b'} 的问题是不能对字符串取负值。有时候二元运算符两边的操作数类型不同但是却可以执行运算,例如将一个整数和一个实数相加时,整数对象会被转换为实数并与另一个实数对象相加。而逻辑与和逻辑或运算符允许运算符两边的操作数为任意类型,在逻辑表达式中它们总是会按照一定的规则转换为 \texttt{boolean} 类型。

另外一种情况是使用自定义类的时候可以重载运算符。本质上来说你可以任意解释这个运算符,那么它的操作数应该是什么类型也是由你决定了。

\subsection{优先级和结合性}

在使用多个运算符组合而成的复合表达式中,运算符的优先级和结合性决定了表达式的求值顺序。各个运算符的优先级和结合性在表\ref{tab::operator_list}中给出。

优先级指定了不同运算符之间的求值顺序,具有高优先级运算符的表达式会先被求值。例如,对表达式 \texttt{1+2*3} 的求值过程会先计算 \texttt{2*3} 的结果,然后计算加法表达式的结果。使用括号可以提升低优先级表达式的求值顺序,例如在表达式 \texttt{(1+2)*3} 求值中,先计算括号中表达式 \texttt{1+2} 的结果,然后计算括号外的乘法表达式。

结合性是指运算符两侧操作数的求值顺序,这里的操作数可能是子表达式。例如,在加法表达式 \texttt{expr1 + expr2} 中,先计算 \texttt{expr1} 的值再计算 \texttt{expr2} 的值,这是因为加法运算符是左结合的。

\begin{table}[htb]
    \centering
    \setlength{\tabcolsep}{3mm}
    \begin{tabular}{cclc} \Xhline{1pt}
        \makecell[c]{\textbf{优先级}} & \makecell[c]{\textbf{运算符}} & \makecell[c]{\textbf{说明}} & \makecell[c]{\textbf{结合性}} \\ \Xhline{1pt}
        1 & \texttt{()} & 分组符号 & - \\
        2 & \texttt{() [] .} & 函数调用,下标运算,域运算 & 左 \\
        3 & \texttt{- ! \textasciitilde} & 负号,逻辑非,位翻转 & 左 \\
        4 & \texttt{* / \%} & 乘法,除法,取余数 & 左 \\
        5 & \texttt{+ -} & 加法,减法 & 左 \\
        6 & \texttt{<< >>} & 左移,右移 & 左 \\
        7 & \texttt{\&} & 位与 & 左 \\
        8 & \texttt{\textasciicircum} & 位异或 & 左 \\
        9 & \texttt{|} & 位或 & 左 \\
        10 & \texttt{..} & 范围运算符 & 左 \\
        11 & \texttt{< <= > >=} & 小于,小于等于,大于,大于等于 & 左 \\
        12 & \texttt{== !=} & 等于,不等于 & 左 \\
        13 & \texttt{\&\&} & 逻辑与 & 左 \\
        14 & \texttt{||} & 逻辑或 & 左 \\
        15 & \texttt{=} & 赋值 & 右 \\
        \Xhline{1pt}
    \end{tabular}
    \caption{运算符列表}
    \label{tab::operator_list}
\end{table}

\subsubsection{使用括号提升优先级}

当我们需要低优先级较低的运算符先求值时可以使用括号。表达式求值过程中会先计算括号中表达式的值。也就是说,对于整个表达式来说,括号中的表达式相当于一个操作数,而不管括号中的表达式的构成。

\section{运算符}

\subsection{算术运算符}

算术运算符用于实现算术运算,这些运算符和我们平时使用的数学符号相似。Berry提供的算术运算符如表\ref{tab::arthmetic_operator}所示。

\begin{table}[htb]
    \centering
    \setlength{\tabcolsep}{10mm}
    \begin{tabular}{ccc} \Xhline{1pt}
        \makecell[c]{\textbf{运算符}} & \makecell[c]{\textbf{功能}} & \makecell[c]{\textbf{示例}} \\ \Xhline{1pt}
        \texttt{-} & 一元负号 & \texttt{- expr} \\
        \texttt{+} & 加号/字符串连接 & \texttt{expr + expr} \\
        \texttt{-} & 减号 & \texttt{expr - expr} \\
        \texttt{*} & 乘号 & \texttt{expr * expr} \\
        \texttt{/} & 除号 & \texttt{expr / expr} \\
        \texttt{\%} & 取余数 & \texttt{expr \% expr} \\
        \Xhline{1pt}
    \end{tabular}
    \caption{算术运算符}
    \label{tab::arthmetic_operator}
\end{table}

二元运算符 \texttt{+} 除了做为加号外还是字符串连接符,当该运算符的操作数为字符串时会执行字符串连接,把两个字符串连接成一条更长的字符串。准确地说,作为字符串连接符的 \texttt{+} 已经不属于算术运算符的范畴了。

二元运算符 \texttt{\%} 是取余数符号,它的操作数必须都是整数,取余运算的结果是左操作数除以右操作数后的余数,例如 \texttt{11\%4} 的结果是 \texttt{3} 。实数类型不能做整除,因此不支持取余。

通常情况下算术运算符不满足交换律。例如表达式 \texttt{2/4} 和 \texttt{4/2} 的值并不相同。

所有的算数运算符都可以在类中重载,重载后的运算符不一定局限于它们原本的功能设计,而是由程序员自己决定。

\subsection{关系运算符}

关系运算符用于比较操作数的大小关系。Berry支持的6种关系运算符在表\ref{tab::relop_operator}中给出。

\begin{table}[htb]
    \centering
    \setlength{\tabcolsep}{10mm}
    \begin{tabular}{ccc} \Xhline{1pt}
        \makecell[c]{\textbf{运算符}} & \makecell[c]{\textbf{功能}} & \makecell[c]{\textbf{示例}} \\ \Xhline{1pt}
        \texttt{<} & 小于 & \texttt{expr < expr} \\
        \texttt{<=} & 小于等于 & \texttt{expr <= expr} \\
        \texttt{==} & 等于 & \texttt{expr == expr} \\
        \texttt{!=} & 不等于 & \texttt{expr != expr} \\
        \texttt{>=} & 大于等于 & \texttt{expr >= expr} \\
        \texttt{>} & 大于 & \texttt{-expr} \\
        \Xhline{1pt}
    \end{tabular}
    \caption{关系运算符}
    \label{tab::relop_operator}
\end{table}

通过比较操作数的大小关系或判断操作数是否相等,对关系表达式求值会产生一个布尔类型的结果。当关系满足时,关系表达式的值为 \texttt{true},否则为 \texttt{false}。关系运算符 \texttt{==} 和 \texttt{!=} 可以使用任何类型的操作数,且允许左右两边的操作数具有不同的类型。其他关系运算符允许使用以下几种操作数的组合:\vspace{-0.5em}
\begin{gather*}
    \bm{integer} \quad relop \quad \bm{integer} \\
    \bm{real} \quad relop \quad \bm{real} \\
    \bm{integer} \quad relop \quad \bm{real} \\
    \bm{real} \quad relop \quad \bm{integer} \\
    \bm{string} \quad relop \quad \bm{string}
\end{gather*}

关系运算中,等号 \texttt{==} 和不等号 \texttt{!=} 满足交换律。如果左右操作数是同一类型或都是数值类型(整数和实数)则根据操作数的值来判断操作数是否相等,否则认为操作数不相等。等于和不等于是互逆运算:若 \texttt{a==b} 为真,则 \texttt{a!=b} 为假,反之也成立。其他关系运算符不满足交换律,但具有下列性质:\texttt{<} 和 \texttt{>=} 是互逆运算,\texttt{>} 和 \texttt{<=} 是互逆运算。关系运算要求操作数必须是相同类型,否则是错误的表达式。

实例可以重载运算符为方法。如果重载关系运算符,程序需要自行保证以上性质。

关系运算符中,\texttt{==} 和 \texttt{!=} 运算符比 \texttt{<}、\texttt{<=}、\texttt{>} 和 \texttt{>=} 的要求更为宽松,后者只允许在相同的类型之间进行比较。在实际的程序开发中,相等或者不等的判定通常比大小的判定要简单,有些操作对象可能不能判断大小而只能判断相等或不等,布尔类型就是如此。

\subsection{逻辑运算符}

逻辑运算符分为逻辑与、逻辑或和逻辑非3种。如表\ref{tab::logic_operator}所示。

\begin{table}[htb]
    \centering
    \setlength{\tabcolsep}{10mm}
    \begin{tabular}{ccc} \Xhline{1pt}
        \makecell[c]{\textbf{运算符}} & \makecell[c]{\textbf{功能}} & \makecell[c]{\textbf{示例}} \\ \Xhline{1pt}
        \texttt{\&\&} & 逻辑与 & \texttt{expr \&\& expr} \\
        \texttt{||} & 逻辑或 & \texttt{expr || expr} \\
        \texttt{!} & 逻辑非 & \texttt{!expr} \\
        \Xhline{1pt}
    \end{tabular}
    \caption{逻辑运算符}
    \label{tab::logic_operator}
\end{table}

对于逻辑与运算符,当两个操作数的值都为 \texttt{true} 时,逻辑表达式的值为 \texttt{true},否则为 \texttt{false}。

对于逻辑或运算符,当两个操作数的值都为 \texttt{false} 时,逻辑表达式的值为 \texttt{false},否则为 \texttt{true}。

逻辑非运算符的作用是翻转操作数的逻辑状态。当操作数的值为 \texttt{true} 时,逻辑表达式的值为 \texttt{false},否则值为 \texttt{true}。

逻辑运算符要求操作数是布尔类型,如果操作数不是布尔类型则会进行转换。转换规则见\ref{section:type_bool}节。

逻辑运算使用了一种称为\textbf{短路求值}(short-circuit evaluation)的求值策略。这种求值策略为:对于逻辑与运算符,当且仅当左操作数为真时才会对又操作数求值;对于逻辑或运算符来说,当且仅当左操作数为假时才会对右操作数求值。短路求值的性质导致逻辑表达式中的代码可能不会全部运行。

\subsection{位运算符}

位运算符能实现一些二进制位的操作,位运算只能对整数类型使用。位运算符的的详细信息如表\ref{tab::bitwise_operator}所示。目前 Berry 还没有支持位运算符。

\begin{table}[htb]
    \centering
    \setlength{\tabcolsep}{10mm}
    \begin{tabular}{ccc} \Xhline{1pt}
        \makecell[c]{\textbf{运算符}} & \makecell[c]{\textbf{功能}} & \makecell[c]{\textbf{示例}} \\ \Xhline{1pt}
        \texttt{\textasciitilde} & 按位翻转 & \texttt{\textasciitilde expr} \\
        \texttt{\&} & 按位与 & \texttt{expr \& expr} \\
        \texttt{|} & 按位或 & \texttt{expr | expr} \\
        \texttt{\textasciicircum} & 按位异或 & \texttt{expr \textasciicircum expr} \\
        \texttt{<<} & 左移 & \texttt{expr << expr} \\
        \texttt{>>} & 右移 & \texttt{expr >> expr} \\
        \Xhline{1pt}
    \end{tabular}
    \caption{关系运算符}
    \label{tab::bitwise_operator}
\end{table}

\subsection{赋值运算符} \label{section::assign_operator}

赋值运算符 \texttt{=} 仅出现在赋值表达式中,其左操作数必须是可写对象。赋值表达式没有结果,因此不能使用连续的赋值运算。这是一些赋值表达式的例子:
\begin{lstlisting}[language=berry, numbers=none]
a = 45    b = 'string'    c = 0
\end{lstlisting}
而下列赋值表达式都是非法的:
\begin{lstlisting}[language=berry, numbers=none]
1 = 5           # Trying to assign a constant 1
a = b = 0       # Continuous assignment
\end{lstlisting}

赋值运算会将右侧操作数的值传递左侧操作数,对于 \texttt{nil}、整数、实数和布尔类型,赋值运算是会传递对象的值,而其他类型的赋值运算传递对象的引用。

\subsection{域运算符和下标运算符}

域运算符 \texttt{.} 用于访问对象的一个属性或者成员,你可以对模块和实例这两种类型来使用域运算符:
\begin{lstlisting}[language=berry, numbers=none]
l = list[]
l.append('item 0')
s = l.item(0)       # 'item 0'
\end{lstlisting}

下标运算符 \texttt{[]} 用于访问对象的元素,例如
\begin{lstlisting}[language=berry, numbers=none]
l[2] = 10   # Read by index
n = l[2]    # Write by index
\end{lstlisting}

    \chapter {Statement}

Berry is an imperative programming language. This paradigm assumes that programs are executed step by step. Normally, Berry statements are executed sequentially, and this program structure is called sequential structure. Although the sequence structure is very basic, branch structures and loop structures are usually used in actual programs. Berry provides several control statements to realize this complex flow structure, such as conditional statements and iteration statements.

Except for line comments, carriage returns or line feeds ("\texttt{\textbackslash r}" and "\texttt{\textbackslash n}") are only used as blank characters, so statements can be written across lines. In addition, you can write multiple statements on the same line.

You can add a semicolon at the end of the statement to indicate the end of the statement, but the interpreter can usually split the statement automatically without using a semicolon. You can use semicolons to tell the interpreter how to parse the code for the code that will be ambiguous. However, it is better not to write ambiguous code.

\section {Simple sentence}

\subsection {Expression statement}

Expression statements are mainly statements composed of assignment expressions or function call expressions. Other expressions can also form sentences, but they have no meaning. For example, expression \texttt{1+2} is a sentence written alone, but it has no effect. The following routines give examples of expression statements and function statements:
\begin{lstlisting}[language=berry, numbers=none]
a = 1 # Assignment statement
print(a) # Call statement
\end{lstlisting}
Line 2 is a simple assignment statement that assigns the literal value \texttt{i} to the variable \texttt{a}. The statement in line 2 is a function call statement, which prints the value of variable \texttt{a} by calling the \texttt{print} function.

Cross-line expressions are written in the same way as single-line expressions, and no special line continuation symbols are required. E.g:
\begin{lstlisting}[language=berry, numbers=none]
a = 1 +
    func() # Wrap line
\end{lstlisting}
You can also write multiple expression statements on one line, and various types of statements can be written on one line. This example puts two expression statements on the same line:
\begin{lstlisting}[language=berry, numbers=none]
b = 1 c = 2 # Multiple statements
\end{lstlisting}

Sometimes the programmer wants to write two statements, but the interpreter may mistakenly think it is one statement. This problem is caused by the ambiguity in the process of grammatical analysis. Take this code as an example:
\begin{lstlisting}[language=berry, numbers=none]
a = c
(b) = 1 # Be regarded as a function call
\end{lstlisting}
Suppose the 4th and 5th lines are intended to be two expression sentences: \texttt{a = c} and \texttt{(b) = 1}, but the interpreter will interpret them as a sentence: \texttt{a = c(b) = 1}. The cause of this problem is that the interpreter incorrectly parses \texttt{c} and \texttt{(b)} into function calls. To avoid ambiguity, we can add a semicolon at the end of the statement to clearly separate the statement:
\begin{lstlisting}[language=berry, numbers=none]
a = c; (b) = 1;
\end{lstlisting}
A better way is not to use parentheses on the left side of the assignment number. Obviously, there is no reason to use parentheses here. Under normal circumstances, complex expressions should not appear on the left side of the assignment operator, but only simple expressions composed of variable names, domain operation expressions, and subscript operation expressions:
\begin{lstlisting}[language=berry, numbers=none]
a = c b = 1
\end{lstlisting}
Using simple expressions only on the left side of the assignment sign will not cause ambiguity in sentence segmentation. Therefore, in most cases, there is no need to use semicolons to separate expressions, and we do not recommend this way of writing.

\subsection {Block} \label{section::block}

A \textbf{Block} is a collection of several sentences. A block is a scope, so the variables defined in the block can only be accessed inside the block and its sub-blocks. There are many places where blocks are used, such as \texttt{if} statements, \texttt{while} statements, function declarations, etc. These statements will contain a block through a pair of keywords. For example, the block used in the \texttt{if} statement:
\begin{lstlisting}[language=berry]
if isOpen
    close()
    print('the device was closed')
end
\end{lstlisting}
The statements in lines 2 to 3 constitute a block, which is sandwiched between the pair of keywords \texttt{if} and \texttt{end} (the conditional expression of the statement in \texttt{if} is not in the block). The block does not need to contain any statements, which constitutes an empty block, or it can be said to be a block containing an empty statement. Broadly speaking, any number of consecutive sentences can be called a block, but we prefer to expand the scope of the block as much as possible, which can ensure that the area of   the block is consistent with the scope of the scope. In the above example, we tend to think that rows 2 to 3 are a whole block, which is the largest range between \texttt{if} keywords and \texttt{end} keywords.

\subsubsection {\texttt{do} Statement}Sometimes we just want to open up a new scope, but don't want to use any control statements. In this case, we can use the \texttt{do} statement to encapsulate the block. \texttt{do} The statement has no control function. \texttt{do} The sentence has the form
\begin{algorithm}
    \texttt{do}\\
    \qquad $\bm{block}$ \\
    \texttt{end}
\end{algorithm}\vspace{-0.6em}\\
Among them $\bm{block}$ is the block we need. This statement uses a pair of \texttt{do} and \texttt{end} keywords to contain blocks. \texttt{do} The statement has no control function, nor does it generate any runtime instructions.

\section {conditional statement}

Berry provides \texttt{if} statements to realize the function of conditional control execution. This kind of program structure is generally called branch structure. \texttt{if} The statement will determine the branch of execution based on the true (\texttt{true}) or false (\texttt{false}) conditional expression. In some languages, there are other options for implementing conditional control. For example, languages   such as C and C++ provide \texttt{switch} statements, but in order to simplify the design, Berry does not support \texttt{switch} statements.

\subsection{\texttt{if} Statement}

\textbf{\texttt{if} statement} is used to implement the branch structure, which selects the branch of the program according to the true or false of a certain judgment condition. The statement can also include \texttt{else} branch or \texttt{elif} branch. The simple \texttt{if} statement form without branches is
\begin{algorithm}
    \texttt{if} $\bm{condition}$ \ \
    \qquad $\bm{block}$ \ \
    \texttt{end}
\end{algorithm}\vspace{-0.6em}\\
$\bm{condition}$ is a conditional expression. When the value of $\bm{condition}$ is \texttt{true}, $\bm{block}$ in the second line will be executed, otherwise the $\bm{block}$ will be skipped and the statement following \texttt{end} will be executed. In the case of $\bm{block}$ being executed, after the last statement in the block is executed, it will leave the \texttt{if} statement and start executing the statement following \texttt{end}.

Here is an example to illustrate the usage of the \texttt{if} statement:
\begin{lstlisting}[language=berry, numbers=none]
if 8 % 2 == 0
    print('this number is even')
end
\end{lstlisting}
This code is used to judge whether the number \texttt{8} is even, and if it is, it will output \texttt{this number is even}. Although this example is very simple, it is enough to illustrate the basic usage of \texttt{if} sentences.

If you want to have a corresponding branch for execution when the condition is met and not met, use the \texttt{if} statement with the \texttt{else} branch. \texttt{if else} The form of the sentence is
\begin{algorithm}
    \texttt{if} $\bm{condition}$ \ \
        \qquad $\bm{block}$ \\
    \texttt{else} \\
        \qquad $\bm{block}$ \\
    \texttt{end}
\end{algorithm}\vspace{-0.6em}
Different from the simple \texttt{if} statement, the \texttt{if else} statement will execute $\bm{block}$ under the \texttt{else} branch when the value of $\bm{condition}$ is \texttt{false}. No matter which branch is executed under $\bm{block}$, after the last statement in the block is executed, the \texttt{if else} statement will pop out, that is, the statement after \texttt{end} will be executed. In other words, no matter whether the value of $\bm{condition}$ is \texttt{true} or \texttt{false}, one $\bm{block}$ will be executed.

Continue to use the judgment of parity as an example, this time change the demand to output corresponding information according to the parity of the input number. The code to achieve this requirement is:
\begin{lstlisting}[language=berry, numbers=none]
if x % 2 == 0
    print('this number is even')
else
    print('this number is odd')
end
\end{lstlisting}
Before running this code, we must first assign an integer value to the variable \texttt{x}, which is the number we want to check for parity. If \texttt{x} is an even number, the program will output \texttt{this number is even}, otherwise it will output \texttt{this number is odd}.Sometimes we need to nest \texttt{if} statements. One way is to nest a \texttt{if} statement under the \texttt{else} branch. This is a very common requirement because many conditions need to be judged consecutively. For this kind of demand, use the \texttt{if else} statement to write:
\begin{lstlisting}[language=berry, numbers=none]
if expr
    block
else
    if expr
        block
    end
end
\end{lstlisting}
Obviously, this way of writing will increase the indentation level of the code, and it is more cumbersome to use multiple \texttt{end} at the end. As an improvement, Berry provides the \texttt{elif} branch to optimize the above writing. Using the \texttt{elif} branch is equivalent to the above code, in the form
\begin{algorithm}
    \texttt{if} $\bm{condition}$ \\
        \qquad $\bm{block}$ \\
    \texttt{elif} $\bm{condition}$ \\
        \qquad $\bm{block}$ \\
    \texttt{else} \\
    \qquad $\bm{block}$ \\
    \texttt{end}
\end{algorithm}\vspace{-0.6em}

\texttt{elif} The branch must be used after the \texttt{if} branch and before the \text{else} branch, and the \texttt{elif} branch can be used multiple times in succession. If the $\bm{condition}$ corresponding to the \texttt{elif} branch is satisfied, the $\bm{block}$ under the branch will be executed. \texttt{elif} Branching is suitable for situations that require multiple conditions to be judged in sequence.

We use a piece of code that judges positive, negative, and 0 to demonstrate the \texttt{elif} branch:
\begin{lstlisting}[language=berry, numbers=none]
if x> 0
    print('positive')
elif x == 0
    print('zero')
else
    print('negative')
end
\end{lstlisting}
Here too, the variable \texttt{x} must be assigned first. This code is very simple and will not be explained.

Some languages   have a problem called dangling "\texttt{else}", which refers to when a \texttt{if} sentence is nested inside another \texttt{if} sentence, where does the \texttt{else} branch belong? Problem with the sentence \texttt{if}. When using C/C++, we must consider the problem of dangling \texttt{else}. In order to avoid ambiguity on the problem of \texttt{if else}, C/C++ programmers often use curly braces to make a branch into a block. In Berry, the branch of the \texttt{if} statement must be a block, which also determines that Berry does not have the problem of overhanging \texttt{else}.

\section {Iteration Statement}

Iterative statements are also called loop statements, which are used to repeat certain operations until the termination condition is met. Berry provides \texttt{while} statement and \texttt{for} two iteration statements. Many languages   also provide these two statements for iteration. Berry’s \texttt{while} statement is similar to the \texttt{while} statement in C/C++, but Berry’s \texttt{for} statement is only used to traverse the elements in the container, similar to the \texttt{foreach} statement provided by some languages   and the one introduced by C++11 New \texttt{for} sentence style. The C-style \texttt{for} statement is not supported.

\subsection{\texttt{while} Statement}

\textbf{\texttt{while} statement} is a basic iterative statement. \texttt{while} statement uses a judgment condition. When the condition is true, the loop body is executed repeatedly, otherwise the loop is ended. The pattern of the statement is
\begin{algorithm}
    \texttt{while} $\bm{condition}$ \ \
        \qquad $\bm{block}$ \ \
    \texttt{end}
\end{algorithm}\vspace{-0.6em}\\
When the program runs to the \texttt{while} statement, it will check whether the expression $\bm{condition}$ is true or false. If it is true, execute the loop body $\bm{block}$, otherwise end the loop. After executing the last statement in $\bm{block}$, the program will jump to the beginning of the statement \texttt{while} and start the next round of detection. If the $\bm{condition}$ expression is false when it is first evaluated, the loop body $\bm{block}$ will not be executed at all (same as the $\bm{condition}$ expression of the \texttt{if} statement is false).Generally speaking, the value of $\bm{condition}$ expression should be able to change during the loop, rather than a constant or a variable modified outside the loop, which will cause the loop to not execute or fail to terminate. A loop that never ends is called an endless loop. Usually we usually expect the loop to execute a specified number of times and then terminate. For example, when using the \texttt{while} loop to access all elements in the array, we hope that the number of loop executions is the length of the array, for example:
\begin{lstlisting}[language=berry, numbers=none]
i = 0
l = ['a','b','c']
while i <l.size()
    print(l[i])
    i = i + 1
end
\end{lstlisting}
This loop gets the elements from the array \texttt{l} and prints them. We use a variable \texttt{i} as the loop counter and array index. We let the value of \texttt{i} reach the length of the array \texttt{l} to end the loop. In the last line of the loop body, we add \texttt{1} to the value of \texttt{i} to ensure that the next element of the array is accessed in the next loop, and the \texttt{while} loop ends when the number of loops reaches the length of the array.

\subsection{\texttt{for} Statement}

Berry’s \textbf{\texttt{for} statement} is used to traverse the elements in the container, and its form is
\begin{algorithm}
    \texttt{for } $\bm{varaible}$ \texttt{:} $\bm{expression}$ \\
        \qquad $\bm{block}$ \ \
    \texttt{end}
\end{algorithm}\vspace{-0.6em}

$\bm{expression}$ The value of the expression must be an iterable container or function, such as the \texttt{range} class. \texttt{for} The statement obtains an iterator from the container, and obtains an element in the container every time through the call to the iterator.

$\bm{varaible}$ is called an iteration variable, which is always defined in the statement \texttt{for}. Therefore $\bm{varaible}$ must be a variable name and not an expression. The container element obtained from the iterator in each loop will be assigned to the iteration variable. This process occurs before the first statement in $\bm{block}$.

The \texttt{for} statement will check whether there are any unvisited elements in the iterator for iteration. If there are, the next iteration will start, otherwise it will end the \texttt{for} statement and execute the statement following \texttt{end}. Currently, Berry only provides read-only iterators, which means that the elements in the container cannot be modified through the iteration variables in the \texttt{for} statement.

The scope of the iteration variable $\bm{varaible}$ is limited to the loop body $\bm{block}$, and the variable will not have any relationship with the variable with the same name outside the scope. To illustrate this point, let's use an example to illustrate. In this example, we use the \texttt{for} statement to access all the elements in the \texttt{rang} instance and print them out. Of course, we also use this example to demonstrate the scope of loop variables.
\begin{lstlisting}[language=berry]
i = "Hi, I'm fine." # Outer variable
for i: 0 .. 2
    print(i) # Iteration variable
end
print(i)
\end{lstlisting}

In this example, relative to the iteration variable \texttt{i} defined in line 2, the variable \texttt{i} defined in line 1 is an external variable. Running this example will get the following result
\begin{lstlisting}[numbers=none]
0
1
2
Hi, I'm fine
\end{lstlisting}
It can be seen that the iteration variable \texttt{i} and the external variable \texttt{i} are two different variables. They just have the same name but different scopes.

\subsubsection{\texttt{for} Principle of Statement}

Unlike the traditional iterative statement \texttt{while}, the \texttt{for} statement uses iterators to traverse the container. If you need to use the \texttt{for} statement to traverse a custom class, you need to understand its implementation mechanism. When using the \texttt{for} statement, the interpreter hides a lot of implementation details. In fact, for such code:
\begin{lstlisting}[language=berry]
for i: 0 .. 2
    print(i)
end
\end{lstlisting}
Will be translated into the following equivalent code by the interpreter:
\begin{lstlisting}[language=berry]
var it = __iterator__(0 .. 2)
try
    while true
        var i = it()
        print(i)
    end
except'stop_iteration'
    # do nothing
end
\end{lstlisting}To some extent, the \texttt{for} statement is just a syntactic sugar, it is essentially just a simple way of writing a piece of complex code. In this equivalent code, an intermediate variable \texttt{it} is used. The value of the variable is an iterator. In this example, it is an iterator of the \texttt{range} container \texttt{0..2}. When processing the \texttt{for} statement, the interpreter hides the intermediate variable of the iterator, so it cannot be accessed in the code.

% <UNUSED>
\if{false}
The parameter of function \texttt{\_\_iterator\_\_} is a container, and the function returns an iterator of parameters. This function gets the iterator by calling the parameter \text{iter} method. Therefore, if the return value of the \texttt{iter} method is an instance (\texttt{instance}) type, this instance must have a \texttt{next} method and a \texttt{hasnext} method.

The parameter of function \texttt{\_\_hasnext\_\_} is an iterator, which checks whether the iterator has the next element by calling the \texttt{hasnext} method of the iterator. \texttt{hasnext} The return value of the method is of type \texttt{boolean}. The parameter of function \texttt{\_\_next\_\_} is also an iterator, which gets the next element in the iterator by calling the \texttt{next} method of the iterator.

So far, the \texttt{\_\_iterator\_\_}, \texttt{\_\_hasnext\_\_} and \texttt{\_\_next\_\_} functions simply call some methods of the container or iterator and then return the return value of these methods. Therefore, the equivalent writing of the \texttt{for} statement can also be simplified into this form:
\begin{lstlisting}[language=berry]
do
    var it = (0 .. 2).iter()
    while (it.hasnext())
        var i = it.next()
        print(i)
    end
end
\end{lstlisting}
This code is easier to read. It can be seen from the effective code that the scope of the iterator variable \texttt{it} is the entire \texttt{for} statement, but it is not visible outside the \texttt{for} statement, while the scope of the iteration variable \texttt{i} is in the loop body, so every time Iterations will define new iteration variables.
\fi
% !<UNUSED>

\section {Jump Statement}

The jump statement provided by Berry is used to realize the jump of the program flow in the loop process. Jump statements are divided into \texttt{break} statements and \texttt{continue} statements. These two statements must be used inside iterative statements and can only be used inside functions to jump. Some languages   provide \texttt{goto} statements to realize arbitrary jumps within functions, which Berry does not provide, but the effects of \texttt{goto} statements can be replaced by conditional statements and iteration statements.

\subsection{\texttt{break} Statement}

\texttt{break} Used to terminate the iteration statement and jump out. After the execution of the \texttt{break} statement, the nearest level of the iteration statement will be terminated immediately and execution will continue from the position of the first statement after the iteration statement. In order to illustrate the execution flow of the \texttt{break} statement, we use an example to demonstrate:
\begin{lstlisting}[language=berry]
while true
    print('before break')
    break
    print('after break')
end
print('out of the loop')
\end{lstlisting}
In this code, the \texttt{break} statement is in a \texttt{while} loop. Before and after the \texttt{break} statement and after the \texttt{while} statement, we have placed a print statement to test the execution flow of the program. The result of this code is:
\begin{lstlisting}[numbers=none]
before break
out of the loop
\end{lstlisting}
This shows that the \texttt{while} statement ends the loop at the \texttt{break} statement position on the 3rd line and the program continues to execute from the 6th line.

\subsection{\texttt{continue} Statement}\texttt{continue} The statement is also used inside an iteration statement. Its function is to end an iteration and immediately start the next round. Therefore, after the execution of the \texttt{continue} statement, the remaining code in the iteration statement of the nearest layer will no longer be executed, but a new round of iteration will start. Here we use a \texttt{for} statement to demonstrate the function of the \texttt{continue} statement:
\begin{lstlisting}[language=berry]
for i: 0 .. 5
    if i >= 2
        continue
    end
    print('i =', i)
end
print('out of the loop')
\end{lstlisting}
Here, the \texttt{for} statement will iterate 6 times. When the iteration variable \texttt{i} is greater than or equal to \texttt{2}, the \texttt{continue} statement on line 3 will be executed, and the print statement on line 5 will not be executed thereafter. In other words, line 5 will only be executed in the first two iterations (at this time \texttt{i<2}). The running result of this routine is:
\begin{lstlisting}[numbers=none]
i = 0
i = 1
out of the loop
\end{lstlisting}
It can be seen that the value of the variable \texttt{i} is only printed twice, which is in line with expectations. Readers can try to print the value of the variable \texttt{i} before the \texttt{continue} statement. You will find that the \texttt{for} statement does iterate 6 times, indicating that the \texttt{continue} statement does not terminate the iteration.

\section{\texttt{import} Statement}

Berry has some predefined modules, such as the \texttt{math} module for mathematical calculations. These modules cannot be used directly, but must be imported with the \texttt{import} statement. There are two ways to import a module:
\begin{algorithm}
    \texttt{import} $\bm{name}$ \ \
    \texttt{import} $\bm{name}$ \texttt{as} $\bm{varaible}$
\end{algorithm}\vspace{-0.6em}\\
$\bm{name}$ For the name of the module to be imported, when using the first writing method to import the module, the imported module can be called directly by using the module name. The second way of writing is to import a module named $\bm{name}$ and modify the module name when calling it to $\bm{varaible}$. For example, a module named \texttt{math}, we use the first method to import and use:
\begin{lstlisting}[language=berry, numbers=none]
import math
math.sin(0)
\end{lstlisting}
Here directly use \texttt{math} to call the module. If the name of a module is relatively long and it is not convenient to write, you can use the \texttt{import as} statement. Here, assume a module named \texttt{hardware}. We want to call the function \texttt{setled} of the module, we can import the module \texttt{hardware} into the variable named \texttt{hw} and use:
\begin{lstlisting}[language=berry, numbers=none]
import hardware as hw
hw.setled(true)
\end{lstlisting}

\section {Exception Handling}

\textbf{Exception handling} The mechanism allows the program to capture and handle exceptions that occur during runtime. Berry supports an exception capture mechanism, which allows the exception capture and handling process to be separated. That is, part of the program is used to detect and collect exceptions, and the other part of the program is used to handle exceptions.

First of all, the problematic program needs to throw an exception first. When these programs are in an exception handling block, a specific program will catch and handle the exception.

\subsection {Throw an exception}

Using the \texttt{raise} statement raises an exception. \texttt{raise} The statement will pass a value to indicate the type of exception so that it can be identified by a specific exception handler. Here is how to use the \texttt{raise} statement:
\begin{algorithm}
    \texttt{raise }$\bm{exception}$ \\
    \texttt{raise }$\bm{exception}$\texttt{, }$\bm{message}$
\end{algorithm}\vspace{-0.6em}\\
The value of the expression $\bm{exception}$ is the thrown \textbf{Outliers}; the optional $\bm{message}$ expression is usually a string describing the exception information, and this expression is called \textbf{Abnormal parameter}. Berry allows any value to be used as an abnormal value, for example, a string can be used as an abnormal value:
\begin{lstlisting}[language=berry, numbers=none]
raise'my_error','an example of raise'
\end{lstlisting}

After the program executes to the \texttt{raise} statement, it will not continue to execute the statements following it, but will jump to the nearest exception handling block. If the most recent exception handling block is in other functions, the functions along the call chain will exit early. If there is no exception handling block, \textbf{Abnormal exit} will occur, and the interpreter will print the exception error message and the call stack of the error location.When the \texttt{raise} statement is in the \texttt{try} statement block, the exception will be caught by the latter. The caught exception will be handled by the \texttt{except} block associated with the \texttt{try} block. If the thrown exception can be handled by the \texttt{except} block, the execution of this block will continue from the statement after the last \texttt{except} block. If all \texttt{except} statements cannot handle the exception, the exception will be rethrown until it can be handled or the exception exits.

\subsubsection {Outliers}

In Berry, you can use any value as an outlier, but we usually use short strings. Berry may also throw some exceptions internally. We call these exceptions \textbf{Standard exception}. All standard exception values   are of string type. Table \ref{tab::stdexpect_list} lists all standard exceptions.
\begin{table}[htb]
    \centering
    \setlength{\tabcolsep}{3mm}
    \begin{tabular}{cll} \toprule
        \textbf{Outliers} & \textbf{Description} & \textbf{Parameter Description} \\ \midrule
        \texttt{assert\_failed} & Assertion failed & Specific exception information \\
        \texttt{index\_error} & \makecell[l]{Subscript error \\ (usually out of bounds)} & Specific exception information \\
        \texttt{io\_error} & IO Malfunction & Specific exception information \\
        \texttt{key\_error} & Key error & Specific exception information \\
        \texttt{runtime\_error} & VM runtime exception & Specific exception information \\
        \texttt{stop\_iteration} & End of iterator & no \\
        \texttt{syntax\_error} & Syntax error & \makecell[l]{The specific error message given \\ by the compiler} \\
        \texttt{unrealized\_error} & Unrealized function & Specific exception information \\
        \texttt{type\_error} & Type error & Specific exception information \\
        \bottomrule
    \end{tabular}
    \caption{Standard exception list}
    \label{tab::stdexpect_list}
\end{table}

\subsection {Catch exceptions}

Use the \texttt{excpet} statement to catch exceptions. It must be paired with the \texttt{try} statement, that is, a \texttt{try} statement block must be followed by one or more \texttt{except} statement blocks. \texttt{try-except} The basic form of the sentence is
\begin{algorithm}
    \texttt{try} \ \
        \qquad $\bm{block}$ \\
    \texttt{excpet} $\bm{...}$ \ \
        \qquad $\bm{block}$ \ \
    \texttt{end}
\end{algorithm}\vspace{-0.6em}\\
The \texttt{except} branch can have the following forms
\begin{algorithm}
    \texttt{excpet ..} \ \
    \texttt{excpet }$\bm{exceptions}$ \\
    \texttt{excpet }$\bm{exceptions}$\texttt{ as }$\bm{variable}$ \\
    \texttt{excpet }$\bm{exceptions}$\texttt{ as }$\bm{variable}$\texttt{, }$\bm{message}$ \\
    \texttt{excpet .. as }$\bm{variable}$ \\
    \texttt{excpet .. as }$\bm{variable}$\texttt{, }$\bm{message}$ \\
\end{algorithm}\vspace{-0.6em}\\
The most basic \texttt{except} statement does not use parameters, this \texttt{except} branch will catch all exceptions; \textbf{Catch exception list} $\bm{exceptions}$ is a list of outliers that can be matched by the corresponding \texttt{except} branch, used between multiple values   in the list Separate by commas; $\bm{variable}$ is \textbf{Abnormal variable}, if the branch catches an exception, the outlier will be bound to the variable; $\bm{message}$ is \textbf{Abnormal parameter variable}, if the branch catches an exception, the abnormal parameter value will be bound To the variable.

When an exception is caught in the \texttt{try} statement block, the interpreter will check the \texttt{except} branch one by one. If the exception value exists in the capture list of a branch, the code block under the branch will be called to handle the exception, and the entire \texttt{try-except} statement will exit after the code block is executed. If all the \texttt{except} branches do not match, the exception will be re-thrown and caught and handled by the outer exception handler.
    \chapter {Function}

\textbf{function} is a "subroutine" that can be called by external code. As a part of the program, the function itself is also a piece of code. A function can have 0 or more parameters and will return a result, which is called the function's \textbf{return value}.

In Berry, the function is \textbf{first class value}. Therefore, in addition to calling functions, you can also pass functions as values, for example, bind functions to variables, use functions as return values, and so on.

\section {Basic Information}

The use of functions includes two parts: function definition and call. The function definition statement uses the \texttt{def} keyword as the beginning. The function definition is the process of packaging and naming the code of the function body. This process only generates the function structure and does not execute the function. The execution function must use \textbf{call operator}, which is a pair of parentheses. The call operator acts on an expression whose result is a function type. The parameters passed to the function are written in parentheses, and multiple parameters are separated by commas. The result of the calling expression is the return value of the function.

\subsection {Function Definition}

\subsubsection {Named Function}

\textbf{named function} is a function given a name when it is defined. Its definition statement consists of the following parts: \texttt{def} keywords, function names, lists consisting of 0 to multiple parameters, and function bodies ( function body), multiple parameters in the parameter list are separated by commas, and all parameters are written in a pair of parentheses. We call the parameter when the function is defined as \textbf{Formal parameters}, and the parameter when calling the function as \textbf{Arguments}. The general form of the function definition is:
\begin{algorithm}
    \texttt{def} $\bm{name}$ \texttt{(} $\bm{arguments}$ \texttt{)} \\
        \qquad $\bm{block}$ \\
    \texttt{end}
\end{algorithm}\vspace{-0.6em}\\
The function name $\bm{name}$ is an identifier; $\bm{arguments}$ is the formal parameter list; $\bm{block}$ is the function body. If the function body is an empty statement, the function is called an "empty function". The function return value statement is contained in the function body. If there is no return statement in $\bm{block}$, the function will return \texttt{nil} by default. The function name is actually the variable name of the bound function object. If the name already exists in the current scope, defining the function is equivalent to binding the function object to this variable.

The following example defines a function named \texttt{add}. The function of this function is to sum two numbers and return.
\begin{lstlisting}[language=berry, numbers=none]
def add(a, b)
    return a + b
end
\end{lstlisting}
\texttt{add} The function has two parameters \texttt{a} and \texttt{b}, and the two addends are passed into the function through these parameters for calculation. \texttt{return} The statement returns the result of the calculation.

A function as a class attribute is called a method. This part of the content will be explained in the object-oriented chapter.

\subsubsection {Anonymous Function}

Unlike named functions, \textbf{anonymous function} has no name, and its definition expression has the form:
\begin{algorithm}
    \texttt{def} \texttt{(} $\bm{arguments}$ \texttt{)} \\
        \qquad $\bm{block}$ \\
    \texttt{end}
\end{algorithm}\vspace{-0.6em}\\
It can be seen that compared with named functions, there is no function name in the definition of anonymous functions $\bm{name}$. The definition of an anonymous function is essentially an expression, which is called \textbf{Function literal}. In order to use anonymous functions, we can bind the function literal value to a variable:
\begin{lstlisting}[language=berry, numbers=none]
add = def (a, b)
    return a + b
end
\end{lstlisting}
The function of this code is exactly the same as that of the function \texttt{add} in the previous section. An anonymous function can be used to conveniently pass the function value as a literal value. Like other types of literals, function literals are also the smallest unit of expressions. Therefore, between \texttt{def} keywords and \texttt{end} are an indivisible whole.

\subsection {Call function}

Take the \texttt{add} function as an example. To call this function, you need to provide two values, and you can get the sum of the two numbers by calling the function:
\begin{lstlisting}[language=berry, numbers=none]
res = add(5, 3)
print(res) # 8
\end{lstlisting}
We call the called function (the \texttt{add} function in the example) as \textbf{Called function}, and the function that calls the called function (the \texttt{main} function in the example) as \textbf{Key function}. The function call process is as follows: First, the interpreter will (implicitly) initialize the formal parameter list of the called function with the argument list, and at the same time suspend the calling function and save its state, then create an environment for the called function and execute the called function. function.

The function will end its execution when it encounters the \texttt{return} statement and pass the return value to the calling function. The interpreter will destroy the environment of the called function after the called function returns, then restore the environment of the calling function and continue to execute the calling function. The return value of the function is also the result of the function call expression.The following example defines a function \texttt{square} and binds this function to a variable \texttt{f}, and then calls the function \texttt{square} through the variable \texttt{f}. This usage is similar to function pointers in C language.
\begin{lstlisting}[language=berry, numbers=none]
def square(n)
    return n * n
end
f = square
print(f(5)) # 25
\end{lstlisting}
It should be noted that the function object is only bound to these variables (refer to section \ref{section::assign_operator}) and cannot be modified, so reassigning the variable corresponding to the function name will not make the function lose:
\begin{lstlisting}[language=berry, numbers=none]
f = square
square = nil
print(f(5)) # 25
\end{lstlisting}
It can be seen that the function can still be called normally after reassigning \texttt{square}. Only after the function object is no longer bound to any variable will it be lost, and the resources occupied by this type of function object will be recycled by the system.

\subsubsection {forward call}

The call of the function must be in the scope of the function variable, so it usually cannot be called before the function is defined. In order to solve this problem, you can use this method to compromise:
\begin{lstlisting}[language=berry]
var func1
def func2(x)
    return func1(x)
end
def func1(x)
    return x * x
end
print(func2(4)) # 16
\end{lstlisting}
In this example, \texttt{func2} calls \texttt{func1}, but the function \texttt{func1} is defined after \texttt{func2}. After executing this code, the program will output the correct result \texttt{16}. This routine uses the mechanism that the function will not be called when the function is defined. Define the variable \texttt{func1} before defining \texttt{func2} to ensure that the symbol \texttt{func1} will not be found during compilation. Then we define the function \texttt{func1} after \texttt{func2} so that the function will be used to overwrite the value of the variable \texttt{func1}. When the function \texttt{func2} is called in the last line \texttt{print(func2(4))}, the variable \texttt{func1} is already the function we need, so the correct result will be output.

\subsubsection {Recursive call}

\textbf{recursive function} refers to functions that call themselves directly or indirectly. Recursion refers to a strategy that divides the problem into similar sub-problems and then solves them. Taking factorial as an example, the recursive definition of factorial is $0!=1, n!=n\cdot(n-1)!$, we can write the recursive function for calculating factorial according to the definition:
\begin{lstlisting}[language=berry]
def fact(n)
    if n == 0
        return 1
    end
    return n * fact(n-1)
end
\end{lstlisting}
Take the factorial of $5$ as an example, the process of manually calculating the factorial of 5 is:
\begin{equation*}
5! = 5 \times 4 \times 3 \times 2 \times 1 = 120
\end{equation*}
The result of calling the \texttt{fact} function is also $120$:
\begin{lstlisting}[language=berry, numbers=none]
print(fact(5)) # 120
\end{lstlisting}

In order to ensure that the depth of the recursive call is limited (too deep recursion level will exhaust the stack space), the recursive function must have an end condition. \texttt{fact} The \texttt{if} statement in the second line of the function definition is used to detect the end condition, and the recursive process ends when \texttt{n} is calculated as \texttt{0}. The above factorial formula does not apply to non-integer parameters. Executing an expression like \texttt{fact(5.1)} will cause a stack overflow error due to the inability to end the recursion.

There is another situation \texttt{Indirect recursion}, that is, the function is not called by itself but by another function (directly or indirectly) called by it. Indirect recursion usually requires the use of forward function call techniques. Take the functions \texttt{is\_odd} and \texttt{is\_even} for calculating odd and even numbers as examples:
\begin{lstlisting}[language=berry]
var is_odd
def is_even(n)
    if n == 0
        return true
    end
    return is_odd(n-1)
end
def is_odd(n)
    if n == 0
        return false
    end
    return is_even(n-1)
end
\end{lstlisting}
These two functions call each other. In order to ensure that this name is in the scope when calling the function \texttt{is\_odd} on line 6, the variable \texttt{is\_odd} is defined on line 1.

\subsubsection {Anonymous function call}If an anonymous function will only be called once, the easiest way is to call it when it is defined, for example:
\begin{lstlisting}[language=berry, numbers=none]
res = def (a, b) return a + b end (1, 2) # 3
\end{lstlisting}
In this routine, we use the call expression directly after the function literal to call the function. This usage is very suitable for functions that will only be called in one place.

You can also bind an anonymous function to a variable and call it:
\begin{lstlisting}[language=berry, numbers=none]
add = def (a, b) return a + b end
res = add(1, 2) # 3
\end{lstlisting}
This usage is similar to the call of a named function, essentially calling the variable bound to the function value. It should be noted that it is more difficult to make recursive calls to anonymous functions, unless you use forward call techniques.

\subsection {Formal and actual parameters}

The function uses actual parameters to initialize the formal parameters when it is called. Under normal circumstances, the actual parameter and the shape parameter are equal and the positions correspond to each other, but Berry also allows the actual parameter to be unequal to the formal parameter: if the actual parameter is more than the formal parameter, the extra actual parameter will be discarded. Less than the formal parameters will initialize the remaining formal parameters to \texttt{nil}.

The process of parameter passing is similar to assignment operation. For \texttt{nil}, \texttt{boolean} and numeric types, parameter passing is by value, while other types are by reference. For the writable pass-by-reference type such as instance, modifying them in the called function will also modify the object in the calling function. The following example demonstrates this feature:
\begin{lstlisting}[language=berry]
var l = [], i = 0
def func(a, b)
    a.push(1)
    b ='string'
end
func(l, i)
print(l, i) # [1] 0
\end{lstlisting}
It can be seen that the value of variable \texttt{l} has changed after calling function \texttt{func}, but the value of variable \texttt{i} has not changed.

\subsection {Functions and local variables}

The function body itself is a scope, so the variables defined in the function are all local variables (refer to section \ref{section::scope_life}). Unlike directly nested blocks, every time a function is called, space is allocated for local variables. The space for local variables is allocated on the stack, and the allocation information is determined at compile time, so this process is very fast. When multiple levels of scope are nested in a function, the interpreter allocates stack space for the scope nesting chain with the most local variables, rather than the total number of local variables in the function.

\subsection{\texttt{return} Statement}

\texttt{return} The statement is used to return the result of a function, that is, the return value of the function. All functions in Berry have a return value, but you can not use any \texttt{return} statement in the function body. At this time, the interpreter will generate a default \texttt{return} statement to ensure that the function returns. \texttt{return} There are two ways to write sentences:
\begin{algorithm}
    \texttt{return} \\
    \texttt{return }$\bm{expression}$
\end{algorithm}\vspace{-0.6em}\\
The first way of writing is to write only the \texttt{return} keyword and not the expression to be returned. In this case, the default \texttt{nil} value is returned. The second way of writing is to follow the expression $\bm{expression}$ after the \texttt{return} keyword, and the value of the expression will be used as the return value of the function. When the program executes to the \texttt{return} statement, the currently running function will end execution and return to the code that called the function to continue running.

When using a separate keyword \texttt{return} as the return statement of a function, it is easy to cause ambiguity. At this time, it is recommended to add a semicolon after \texttt{return} to prevent errors:
\begin{lstlisting}[language=berry, numbers=none]
def func()
    return;
    x = 1
end
\end{lstlisting}
In this example, the \texttt{x = 1} statement after the \texttt{return} statement will not be executed, so it is redundant. If this kind of redundant code is avoided, the \texttt{return} statement is usually followed by keywords such as \texttt{end}, \texttt{else} or \texttt{elif}. In this case, even if a separate \texttt{return} statement is used, there is no need to worry about ambiguity.

\section{closure}

\subsection {Basic Concepts}

As mentioned earlier, functions are the first type of value in Berry. You can define functions anywhere, and you can also pass functions as parameters or return values. When another function is defined in a function, the nested function can access the local variables of any outer function. We call the "local variables of the outer function" used in the function the function \textbf{Free variable}. The generalized free variables also include global variables, but there is no such rule in Berry.\textbf{Closure} is a technique that binds functions to \textbf{environments}. The environment is a mapping that associates each free variable of a function with a value. In terms of implementation, closures associate the function prototype with its own variables. Function prototypes are generated at compile time, and environment is a runtime concept, so closures are also dynamically generated at runtime. Each closure binds the function prototype to the environment when it is generated, for example, in the following example:
\begin{lstlisting}[language=berry]
def func(i) # The outer function
    def foo() # The inner function (closure)
        print(i)
    end
    foo()
end
\end{lstlisting}
The inner function \texttt{foo} is a closure, which has a free variable \texttt{i}, which is a parameter of the outer function \texttt{func}. When the closure \texttt{foo} is generated, its function prototype is bound to the environment containing the free variable \texttt{i}. When the variable \texttt{foo} leaves the scope, the closure will be destroyed. Usually, the inner function will be the return value of the outer function, for example:
\begin{lstlisting}[language=berry]
def func(i) # The outer function
    return def () # Return a closure (anonymous function)
        print(i)
        i = i + 1
    end
end
\end{lstlisting}
The closure returned here is an anonymous function. When the closure is returned by the outer function, the local variables of the outer function will be destroyed, and the closure will not be able to directly access the variables in the original outer function. The system will copy the value of the free variable to the environment when the free variable is destroyed. The life cycle of these free variables is the same as the closure, and can only be accessed by the closure. The returned function or closure will not be executed automatically, so we need to call the closure returned by the function \texttt{func}:
\begin{lstlisting}[language=berry]
f = func(0)
f()
\end{lstlisting}
This code will output \texttt{0}. If we continue to call the closure \texttt{f}, we will get the output \texttt{1}, \texttt{2}, \texttt{3}\ldots\ This may not be well understood: variable [2.198 ] Is destroyed after the function \texttt{func} returns, and as a free variable of the closure \texttt{f}, \texttt{i} will be stored in the closure environment, so every time \texttt{f} is called, the value of \texttt{i} will be added to $1$ (\texttt{func} function definition line 4).

\subsubsection {Use of closures}

Closures have many uses. Here are a few common uses:

\paragraph{Lazy evaluation}

The closure does not do anything until it is called.

\paragraph{Function private communication}

You can let some closures share free variables, which are only visible to these closures, and communicate between functions by changing the values   of these free variables. This can avoid the use of external variables.

\paragraph{Generate multiple functions}

Sometimes we may need to use multiple functions, these functions may only have different values   of some variables. We can implement a function and then use these different variables as function parameters. A better way is to return the closure through a factory function, and use these possibly different variables as free variables of the closure, so that you don't always have to write those parameters when calling the function, and any number of similar functions can be generated.

\paragraph{Simulate private members}

Some languages   support the use of private members in objects, but Berry's class does not support private members. We can use the free variables of closures to simulate private members. This use is not the original intention of designing closures, but nowadays, this "misuse" of closures is very common.

\paragraph{Cache result}

If there is a function that is very time-consuming to run, it will take a lot of time to call it every time. We can cache the result of this function, look it up in the cache before calling the function, and return the cached value if found, otherwise call the function and update the cached value. We can use closures to save the cached value so that it will not be exposed to the outer scope, and the cached result will be retained (until the closure is destroyed).

\subsection {Binding free variables}If multiple closures bind the same free variable, all closures will always share this free variable. E.g:
\begin{lstlisting}[language=berry]
def func(i) # The outer function
    return [# Return a closure list
        def () # The closure #1
            print("closure 1 log:", i)
            i = i + 1
        end,
        def () # The closure #2
            print("closure 2 log:", i)
            i = i + 1
        end
    ]
end
\end{lstlisting}
The function \texttt{func} in this example returns two closures through a list, and these two closures share free variables \texttt{i}. If we call these closures:
\begin{lstlisting}[language=berry]
f = func(0)
f[0]() # closure 1 log: 0
f[1]() # closure 2 log: 1
\end{lstlisting}
As you can see, we updated the free variable \texttt{i} when we called the closure \texttt{f[0]}, and this change affected the result of calling the closure \texttt{f[1]}. This is because if a free variable is used by multiple closures, there is only one copy of the free variable, and all closures have a reference to the free variable entity. Therefore, any modification to the free variable is visible to all closures that use the free variable.

Similarly, before the local variables of the outer function are destroyed, modifying the value of the free variable will also affect the closure:
\begin{lstlisting}[language=berry]
def func()
    i = 0
    def foo()
        print(i)
    end
    i = 1
    return foo
end
\end{lstlisting}
In this example, we change the value of the variable \texttt{i} (which is the free variable of the closure \texttt{foo}) from \texttt{0} to \texttt{1} before the outer function \texttt{func} returns, then we call the closure afterwards The value of the free variable \texttt{i} when the package \texttt{foo} is also \texttt{1}:
\begin{lstlisting}[language=berry]
func()() # 1
\end{lstlisting}

\subsection {Create closure in loop}When constructing a closure in the loop body, you may not want the free variables of the closure to change with the loop variables. Let's first look at an example of creating a closure in a loop \texttt{while}:
\begin{lstlisting}[language=berry]
def func()
    l = [] i = 0
    while i <= 2
        l.push(def () print(i) end)
        i = i + 1
    end
    return l
end
\end{lstlisting}
In this example, we construct a closure in a loop and put this closure in a \texttt{list}. Obviously, when the loop ends, the value of the variable \texttt{i} will be \texttt{3}, and all the closures in the list \texttt{l} are also references using this variable. If we execute the closure returned by \texttt{func} we will get the same result:
\begin{lstlisting}[language=berry]
res = func()
res[0]() # 3
res[1]() # 3
res[2]() # 3
\end{lstlisting}
If we want each closure to refer to different free variables, we can define another layer of functions, and then bind the current loop variables with the function parameters:
\begin{lstlisting}[language=berry]
def func()
    l = [] i = 0
    while i <= 2
        l.push(def (n)
            return def () print(n) end
        end (i))
        i = i + 1
    end
    return l
end
\end{lstlisting}
To help understand this seemingly incomprehensible code, we focus on the code from lines 4 to 6:
\begin{lstlisting}[language=berry]
def (n)
    return def ()
        print(n)
    end
end (i)
\end{lstlisting}
Here actually defines an anonymous function and calls it immediately. The function of this temporary anonymous function is to bind the value of the loop variable \texttt{i} to its parameter \texttt{n}, and the variable \texttt{n} is also what we need to close The free variables of the package, so that the free variables bound to the closure constructed during each loop are different. Now we will get the desired output:
\begin{lstlisting}[language=berry]
res = func()
res[0]() # 0
res[1]() # 1
res[2]() # 2
\end{lstlisting}
There are some ways to solve the problem of loop variables as free variables. A slightly simpler way is to define a temporary variable in the loop body:
\begin{lstlisting}[language=berry]
def func()
    l = [] i = 0
    while i <= 2
        temp = i
        l.push(def () print(temp) end)
        i = i + 1
    end
    return l
end
\end{lstlisting}
Here \texttt{temp} is a temporary variable. The scope of this variable is in the loop body, so it will be redefined every time it loops. We can also use the \texttt{for} statement to solve the problem:
\begin{lstlisting}[language=berry]
def func()
    l = []
    for i: 0 .. 2
        l.push(def () print(i) end)
    end
    return l
end
\end{lstlisting}
This may be the simplest way. \texttt{for} The iteration variable of the statement will be created in each loop. The principle is similar to the previous method.

\section {Lambda expression}

\textbf{Lambda expression} is a special anonymous function. Lambda expression is composed of parameter list and function body, but the form is different from general function: \vspace{-0.5em}
\begin{gather*}
    \texttt{/}\ args\ \texttt{->}\ expr
\end{gather*}
$\bm{args}$ is the parameter list, the number of parameters can be zero or more, and multiple parameters are separated by commas or spaces (cannot be mixed at the same time); $\bm{expr}$ is the return expression, the lambda expression will return the expression value. Lambda expressions are suitable for implementing functions with very simple functions. For example, the lambda expression for judging the size of two numbers is:
\begin{lstlisting}[language=berry, numbers=none]
/ a b -> a <b
\end{lstlisting}
This is easier than writing a function of the same function. In some general sorting algorithms, this type of size comparison function may need to be used extensively. Using lambda expressions can simplify the code and improve readability.

Like general functions, lambda expressions can form closures. Lambda expressions are called in the same way as ordinary functions. If you use the immediate calling method similar to anonymous functions:
\begin{lstlisting}[language=berry, numbers=none]
lambda = / a b -> a <b
result = lambda(1, 2) # normal calling
result = (/ a b -> a <b)(1, 2) # direct calling
\end{lstlisting}
Since the function call operator has a higher priority, a pair of parentheses should be added to the lambda expression when making a direct call, so that it will be called as a whole.
    \chapter{面向对象功能}

出于优化方面的考虑,Berry没有将简单类型作为对象,这些简单类型包括 \texttt{nil} 类型、数值类型、布尔类型和字符串类型。但是Berry提供了类来实现对象机制,在Berry的基本数据类型中,\texttt{list}、\texttt{map} 和 \texttt{range} 是类对象。一个对象是是包含数据和方法的集合,其中数据由一些变量来构成,而方法则是函数。对象的类型称为类(class),而对象的实体称为实例(instance)。

\section{类和实例}

\subsection{类的声明}

要使用一个类首先要进行声明。类的声明由关键字 \texttt{class} 开始,声明中要指定类的成员变量和方法,这是声明类的一个例子:
\begin{lstlisting}[language=berry, numbers=none]
class person
    var name, age
    def init(name, age)
        self.name = name
        self.age = age
    end
    def tostring()
        return 'name: ' + str(self.name) + ', age: ' + str(self.age)
    end
end
\end{lstlisting}

类的成员变量使用关键字 \texttt{var} 声明,而成员方法使用关键字 \texttt{def} 声明。目前Berry不支持在定义时初始化成员变量,因此成员变量的初始化工作应该由构造函数来完成。类的属性不能再声明完成后再做修改,因此类是一种只读对象\footnote{这种设计是为了保证在实现解释器的时候可以在C语言中静态构造类并使用 \texttt{const} 属性修饰以节省RAM}。

Berry 的类不支持访问限制,类的所有属性都对外部可见。在原生类中可以使用一些技巧使属性对 Berry 代码不可见(通常是让成员名字以``\text{.}''开头)。可以使用一些约定来限制对类中成员的访问,比如约定使用下划线开头的属性是私有属性,这种约定并不会在语法层面上有什么用,但是有利于代码的逻辑结构。

\subsection{实例化}

类本身只是一种抽象的描述。以汽车为例,我知道汽车的概念,而当我们真的要使用汽车的时候则需要真实的汽车。使用类的情况也类似,我们不会仅仅去使用这种抽象的描述,而是需要根据这种描述去生产出一个具体的对象。这个过程叫做\textbf{类的实例化},简称实例化,实例化产生的具体对象称为\textbf{实例}。类本身不具有数据,而实例化根据类所描述的信息生产一个实例并赋予实例具体的数据。

\subsection{方法和 \texttt{self} 参数}

类的方法本质上也是函数,与普通的函数不同,方法会隐式地传入一个 \texttt{self} 参数,且 \text{self} 总是作为第一个参数,该参数存储当前实例的引用。由于 \texttt{self} 参数的存在,方法的参数数量会比声明时定义的参数数量多一个。这里我们用一个简单的例子演示:
\begin{lstlisting}[language=berry, numbers=none]
class Test
    def method()
        return self
    end
end
object = Test()
print(object)
print(object.method())
\end{lstlisting}
这个例子中定义了一个 \texttt{Test} 类,它有一个 \texttt{method} 方法,该方法返回它的 \texttt{self} 参数。例程中的最后两行分别打印了 \texttt{Test} 类的实例 \texttt{object} 的值和使用 \texttt{method} 方法的返回值。该例子的运行结果为\footnote{由于实例对象是动态分配的,它们的内存地址是随机的,读者运行这段代码的结果可能与此处不同。}
\begin{lstlisting}[numbers=none]
<instance: 00E880D4>
<instance: 00E880D4>
\end{lstlisting}
可以看出,方法的 \texttt{self} 参数和使用实例的名字(例子中的 \texttt{object})都是表示同一个对象,它们都是实例的引用。使用 \texttt{self} 可以在方法中访问实例的成员或者属性。

\subsection{构造函数和析构函数}

\subsubsection{构造函数}

类的构造函数为 \texttt{init} 方法,构造函数会在类实例化的时候调用,因此构造函数一般用于成员的初始化工作,例如:
\begin{lstlisting}[language=berry, numbers=none]
class Test
    var a
    def init()
        self.a = 'this is a test'
    end
end
\end{lstlisting}
这个例子中的构造函数将 \texttt{Test} 类的 \texttt{a} 成员初始化为字符串 \texttt{'this is a test'}。如果我们实例化该类,就可以获得成员 \texttt{a} 的值:
\begin{lstlisting}[language=berry, numbers=none]
print(Test().a) # this is a test
\end{lstlisting}

\subsubsection{析构函数}

类的析构函数为 \texttt{deinit} 方法,析构函数会在实例被销毁时调用,析构函数一般用于完成一些清理工作。由于垃圾回收机制会自动释放无用对象的内存,因此不需要在析构函数中释放内存(也没有办法在析构函数中释放内存)。在大部分情况下都不需要使用析构函数,除非某个类要求在销毁时必须进行一定的处理,一个典型的例子是文件对象在销毁时必须关闭文件。

\section{类的继承}

Berry 只支持单继承,也就是类只能有一个基类,基类使用运算符 \texttt{:} 来声明:
\begin{lstlisting}[language=berry, numbers=none]
class Test : Base
    ...
end
\end{lstlisting}
这里 \texttt{Test} 类继承自 \texttt{Base} 类。子类会继承基类的所有方法和属性,同时你可以在子类中覆盖它们,这个机制被称为\textbf{重载}。通常情况下,我们只会重载方法,而不必重载属性。

Berry 类的继承机制比较简单,子类会包含基类的引用,实例对象也是类似。在实例化一个有基类的类时其实会生成多个对象,这些对象会根据继承关系链在一起,最后我们会拿到继承链最末端的实例对象。

\section{方法重载}

\textbf{重载}是指子类和基类使用同名的方法,而子类的方法将会覆盖基类方法的机制。准确地说成员变量也可以重载,但是这种重载没有任何意义。方法的重载分为普通方法重载以及运算符重载。

\subsection{普通方法重载}

\subsection{运算符重载}

可以通过类的运算符重载使实例支持内置运算符的操作,例如重载了加法运算符的类,我们可以对其实例使用加法运算符进行运算。重载的运算符是具有特殊名称的方法,二元运算符的重载函数形式为
\begin{algorithm}
    \texttt{def }$\bm{operator}$\texttt{(}$\bm{other}$\texttt{)}\\
    \qquad $\bm{block}$ \\
    \texttt{end}
\end{algorithm}\vspace{-0.6em}\\
$\bm{operator}$为重载的二元运算符,二元运算符的左操作数是 \texttt{self} 对象,而右操作数则是参数$\bm{other}$的值。一元运算符的重载函数形式为
\begin{algorithm}
    \texttt{def }$\bm{operator}$\texttt{()}\\
    \qquad $\bm{block}$ \\
    \texttt{end}
\end{algorithm}\vspace{-0.6em}\\
$\bm{operator}$为重载的一元运算符,为了和减法运算符相区别,一元负号在重载时写作 \texttt{-*}。运算符重载函数应当具有返回值,因为默认的 \texttt{nil} 返回值通常不是期望的结果。我们先以一个整数类作为例子,说明运算符重载的使用方法。首先要定义 \texttt{integer} 类:
\begin{lstlisting}[language=berry]
class integer
    var value
    def init(v)
        self.value = v
    end
    def +(other)
        return integer(self.value + other.value)
    end
    def *(other)
        return integer(self.value * other.value)
    end
    def -*()
        return integer(-self.value)
    end
    def tostring(other)
        return str(self.value)
    end
end
\end{lstlisting}
\texttt{integer} 类重载了加号、乘号和符号运算符,\texttt{tostring} 方法是为了让实例可以使用 \texttt{print} 函数输出结果。我们可以用一行简单的代码来测试类的运算符重载功能:
\begin{lstlisting}[language=berry, numbers=none]
integer(1) + integer(2) * -integer(3) # -5
\end{lstlisting}
这行代码的结果是一个 \texttt{integer} 实例,该实例的 \texttt{value} 成员的值是 \texttt{-5},这个和整数做同样的四则运算得到的结果一致。

逻辑运算符不能直接重载,如果需要实例支持逻辑运算,则要实现 \texttt{tobool} 方法,该方法没有参数,返回值必须是布尔类型。实例的逻辑运算实际上是通过将实例转化为布尔值来实现的,因此实例的逻辑运算完全符合一般逻辑运算的性质。下标运算符也不是直接重载,而是要通过 \texttt{item} 和 \texttt{setitem} 方法来实现。\texttt{item} 方法用于下标读取,它的第一个参数是下标值,返回值是下标运算的结果;\texttt{setitem} 用于下标写入,它的第一个参数是下标值,第二个参数是待写入的值,该方法不使用返回值。

重载后的运算符可以被赋予任何意义,甚至不满足运算符通常的性质。从代码的通用性和理解难度的方面来考虑,不建议用户对重载运算符赋予与普遍含义相差甚远的功能。

\section{访问基类对象}

    \chapter{库和模块}

%==============================================================================
\section{基础库}

标准库中有一些可以直接使用的函数和类,它们为 Berry 程序提供基础的服务,因此也被称为基础库。基础库中的函数和类都是全局作用域可见的(属于内建作用域),因此可以在任何地方使用。不要定义和基础库中函数或类相同名称的变量,这样做会导致无法引用基础库中的函数和类。

\subsection{内建函数}

%%%%%%%%%%%%%%%%%%%%%%%%%%%%%%%%%%%%%%%%%%%%%%%%%%%%%%%%%%%%%%%%%%%%%%%%%%%%%%%
\libtitle{\texttt{print} 函数}

\paragraph{用法}
\begin{lstlisting}[language=berry, numbers=none]
print(...)
\end{lstlisting}

\paragraph{说明}
该函数会将输入的参数打印到标准输出设备。该函数可以接受任何类型、任何数量的参数。所有的类型都会直接打印它的值,而对于实例,该函数会检查实例是否有 \texttt{tostring()} 方法,如果有则打印该实例调用 \texttt{tostring()} 方法后的返回值,否则打印实例的地址。

\paragraph{例子}
\begin{lstlisting}[language=berry, numbers=none]
print('Hello World!') # Hello World!
print([1, 2, '3'])    # [1, 2, '3']
print(print)          # <function: 0x561092293780>
\end{lstlisting}

%%%%%%%%%%%%%%%%%%%%%%%%%%%%%%%%%%%%%%%%%%%%%%%%%%%%%%%%%%%%%%%%%%%%%%%%%%%%%%%
\libtitle{\texttt{input} 函数}

\paragraph{用法}
\begin{lstlisting}[language=berry, numbers=none]
input()
input(prompt)
\end{lstlisting}

\paragraph{说明}
\texttt{input} 函数用于从标准输入设备输入一行字符串。该函数可以使用 \texttt{prompt} 参数作为输入提示,\texttt{prompt} 参数必须为字符串类型。
调用 \texttt{input} 函数后会从键盘缓冲区中读取字符,直到遇到换行字符为止。

\paragraph{例子}
\begin{lstlisting}[language=berry, numbers=none]
input('please enter a string: ') # please enter a string: 
\end{lstlisting}
\texttt{input} 函数在按下``Enter''键以后才会返回,因此程序``卡住''并不是错误。

%%%%%%%%%%%%%%%%%%%%%%%%%%%%%%%%%%%%%%%%%%%%%%%%%%%%%%%%%%%%%%%%%%%%%%%%%%%%%%%
\libtitle{\texttt{type} 函数} \label{section::baselib_type}

\paragraph{用法}
\begin{lstlisting}[language=berry, numbers=none]
type(value)
\end{lstlisting}

\begin{itemize}
    \item \emph{value}:输入参数(期望获取它的类型)。
    \item \emph{return value}:说明参数类型的字符串。
\end{itemize}

\paragraph{说明}
该函数接收一个任意类型的参数并返回参数的类型。返回值是说明参数类型的字符串。表\ref{tab::type_return_list}给出了主要参数类型对应的返回值:
\begin{table}[htb]
    \centering
    \setlength{\tabcolsep}{6mm}
    \begin{tabular}{cc!{\vrule width 1pt}cc} \Xhline{1pt}
        \textbf{参数类型} & \textbf{返回值} & \textbf{参数类型} & \textbf{返回值} \\ \hline
        Nil & \texttt{'nil'} & Integer & \texttt{'int'} \\
        Real & \texttt{'real'} & Boolean & \texttt{'bool'} \\
        Function & \texttt{'function'} & Class & \texttt{'class'} \\
        String & \texttt{'string'} & Instance & \texttt{'instance'} \\
        \Xhline{1pt}
    \end{tabular}
    \caption{类型名对照表}
    \label{tab::type_return_list}
\end{table}

\paragraph{示例}
\begin{lstlisting}[language=berry, numbers=none]
type(0)         # 'int'
type(0.5)       # 'real'
type('hello')   # 'string'
type(print)     # 'function'
\end{lstlisting}

%%%%%%%%%%%%%%%%%%%%%%%%%%%%%%%%%%%%%%%%%%%%%%%%%%%%%%%%%%%%%%%%%%%%%%%%%%%%%%%
\libtitle{\texttt{classname} 函数}

\paragraph{用法}
\begin{lstlisting}[language=berry, numbers=none]
classname(object)
\end{lstlisting}

\paragraph{说明}
该函数返回参数的类名(字符串)。因此参数必须是一个类或者实例,其他类型的参数将返回 \texttt{nil}。

\paragraph{示例}
\begin{lstlisting}[language=berry, numbers=none]
classname(list)     # 'list'
classname(list())   # 'list'
classname({})       # 'map'
classname(0)        # nil
\end{lstlisting}

%%%%%%%%%%%%%%%%%%%%%%%%%%%%%%%%%%%%%%%%%%%%%%%%%%%%%%%%%%%%%%%%%%%%%%%%%%%%%%%
\libtitle{\texttt{str} 函数}

\paragraph{用法}
\begin{lstlisting}[language=berry, numbers=none]
str(value)
\end{lstlisting}

\paragraph{说明}
该函数将参数转化为字符串并返回。 \texttt{str} 函数可以接受任意类型的参数并转化。当参数类型为实例时将检查该实例是否有 \texttt{tostring()} 方法,如果有将使用该方法的返回值,否则将实例的地址转化为字符串。

\paragraph{示例}
\begin{lstlisting}[language=berry, numbers=none]
str(0)  # '0'
str(nil)  # 'nil'
str(list)  # 'list'
str([0, 1, 2])  # '[0, 1, 2]'
\end{lstlisting}

%%%%%%%%%%%%%%%%%%%%%%%%%%%%%%%%%%%%%%%%%%%%%%%%%%%%%%%%%%%%%%%%%%%%%%%%%%%%%%%
\libtitle{\texttt{number} 函数}

\paragraph{用法}
\begin{lstlisting}[language=berry, numbers=none]
number(value)
\end{lstlisting}

\paragraph{说明}
该函数将输入的字符串或者数字转化为数值类型返回。如果输入参数为整数或者实数则直接返回。如果是字符串则尝试将字符串按十进制格式转化为数值,转化时会自动判断整数或实数。其他类型返回 \texttt{nil}。

\paragraph{示例}
\begin{lstlisting}[language=berry, numbers=none]
number(5)       # 5
number('45.6')  # 45.6
number('50')    # 50
number(list)    # nil
\end{lstlisting}

%%%%%%%%%%%%%%%%%%%%%%%%%%%%%%%%%%%%%%%%%%%%%%%%%%%%%%%%%%%%%%%%%%%%%%%%%%%%%%%
\libtitle{\texttt{int} 函数}

\paragraph{用法}
\begin{lstlisting}[language=berry, numbers=none]
int(value)
\end{lstlisting}

\paragraph{说明}
该函数将输入的字符串或者数字转化为整数并返回。如果输入参数为整数则直接返回,如果是实数则舍弃小数部分。如果是字符串则尝试将字符串按十进制转化为整数。其他类型返回 \texttt{nil}。

\paragraph{示例}
\begin{lstlisting}[language=berry, numbers=none]
int(5)       # 5
int(45.6)    # 45
int('50')    # 50
int(list)    # nil
\end{lstlisting}

%%%%%%%%%%%%%%%%%%%%%%%%%%%%%%%%%%%%%%%%%%%%%%%%%%%%%%%%%%%%%%%%%%%%%%%%%%%%%%%
\libtitle{\texttt{real} 函数}

\paragraph{用法}
\begin{lstlisting}[language=berry, numbers=none]
real(value)
\end{lstlisting}

\paragraph{说明}
该函数将输入的字符串或者数字转化为实数并返回。如果输入参数为实数则直接返回,如果是整数则转化为实数。如果是字符串则尝试将字符串按十进制转化为实数。其他类型返回 \texttt{nil}。

\paragraph{示例}
\begin{lstlisting}[language=berry, numbers=none]
real(5)       # 5, type(real(5)) → 'real'
real(45.6)    # 45.6
real('50.5')  # 50.5
real(list)    # nil
\end{lstlisting}

%%%%%%%%%%%%%%%%%%%%%%%%%%%%%%%%%%%%%%%%%%%%%%%%%%%%%%%%%%%%%%%%%%%%%%%%%%%%%%%
\libtitle{\texttt{length} 函数}

\paragraph{用法}
\begin{lstlisting}[language=berry, numbers=none]
length(value)
\end{lstlisting}

\paragraph{说明}
该函数返回输入字符串的长度。如果输入参数不为字符串则返回0。字符串的长度以字节计算。

\paragraph{示例}
\begin{lstlisting}[language=berry, numbers=none]
length(10)          # 0
length('s')         # 1
length('string')    # 6
\end{lstlisting}

%%%%%%%%%%%%%%%%%%%%%%%%%%%%%%%%%%%%%%%%%%%%%%%%%%%%%%%%%%%%%%%%%%%%%%%%%%%%%%%
\libtitle{\texttt{super} 函数}

\paragraph{用法}
\begin{lstlisting}[language=berry, numbers=none]
super(object)
\end{lstlisting}

\paragraph{说明}
该函数返回实例的父对象。当你将一个派生类实例化时,会同时实例化它的基类。访问基类的实例(也就是父对象)时需要使用 \texttt{super} 函数。

\paragraph{示例}
\begin{lstlisting}[language=berry, numbers=none]
class mylist : list end
l = mylist()    # classname(l) --> 'mylist'
sl = super(l)   # classname(sl) --> 'list'
\end{lstlisting}

%%%%%%%%%%%%%%%%%%%%%%%%%%%%%%%%%%%%%%%%%%%%%%%%%%%%%%%%%%%%%%%%%%%%%%%%%%%%%%%
\libtitle{\texttt{assert} 函数}

\paragraph{用法}
\begin{lstlisting}[language=berry, numbers=none]
assert(expression)
\end{lstlisting}

\paragraph{说明}
该函数用于实现断言功能。\texttt{assert} 函数接受一个参数,当参数的值为 \texttt{false} 或者 \texttt{nil} 时该函数将会触发一个断言错误,否则该函数不会产生任何效果。需要注意的是,即使参数是逻辑运算中等效为 \texttt{false} 的值(例如 \texttt{0})也不会触发断言错误。

\paragraph{示例}
\begin{lstlisting}[language=berry, numbers=none]
assert(false)   # assert failed
assert(nil)     # assert failed
assert()        # assert failed
assert(true)    # pass
assert(0)       # pass
\end{lstlisting}

%%%%%%%%%%%%%%%%%%%%%%%%%%%%%%%%%%%%%%%%%%%%%%%%%%%%%%%%%%%%%%%%%%%%%%%%%%%%%%%
\libtitle{\texttt{compile} 函数}

\paragraph{用法}
\begin{lstlisting}[language=berry, numbers=none]
compile(string)
compile(string, 'string')
compile(filename, 'file')
\end{lstlisting}

\paragraph{说明}
该函数将编译 Berry 源代码编译为一个函数,源代码可以是一个字符串,也可以是一个文本文件。\texttt{compile} 函数的第一个参数为一个字符串,第二个参数为字符串 \texttt{'string'} 或 \texttt{'file'}。当第二个参数为 \texttt{'string'} 或没有第二个参数时,\texttt{compile} 函数会将第一个参数作为源代码进行编译。当第二个参数为 \texttt{'file'} 时,\texttt{compile} 函数将会编译第一个参数对应的文件。如果编译成功,\texttt{compile} 将会返回编译生成的函数,否则返回 \texttt{nil}。

\paragraph{示例}
\begin{lstlisting}[language=berry, numbers=none]
compile('print(\'Hello World!\')')() # Hello World!
compile('test.be', 'file')
\end{lstlisting}

%==============================================================================

\subsection{\texttt{list} 类}

\texttt{list} 是一个内建类类型,这是一种顺序存储容器,支持下标读写。\texttt{list} 类似其它编程语言中的数组(array)。获得一个 \texttt{list} 类的实例可以使用方括号对来构造:\texttt{[]} 将生成一个空的 \texttt{list} 实例,而 \texttt{[expr, expr, ...]} 将生成具有若干元素的 \texttt{list} 实例。还可以通过调用 \texttt{list} 类来实例化:执行 \texttt{list()} 将得到一个空的 \texttt{list} 实例,而 \texttt{list(expr, expr, ...)} 将返回一个具有若干元素的实例。

\libtitle{\texttt{list} 方法(构造函数)} 

初始化 \texttt{list} 容器。该方法可以接受 0 到多个参数,传递多个参数时生成的 \texttt{list} 实例将以这些参数作为元素,元素的排列顺序和参数的排列顺序一致。

\libtitle{\texttt{tostring} 方法}

将 \texttt{list} 实例序列化为字符串并返回。例如执行 \texttt{[1, [], 1.5].tostring()} 的结果是 \texttt{'[1, [], 1.5]'}。如果 \texttt{list} 容器引用了自身,相应为位置将使用省略号来代替具体的值:
\begin{lstlisting}[language=berry, numbers=none]
l = [1, 2]
l[0] = l
print(l)    # [[...], 2]
\end{lstlisting}

\libtitle{\texttt{append} 方法}

在 \texttt{list} 容器尾部追加一个元素。该方法的原型是 \texttt{append(value)},参数 \texttt{value} 是要追加的值,追加的值存储在 \texttt{list} 容器的尾部。追加操作会使 \texttt{list} 容器的元素数量加1。可以向 \texttt{list} 实例中追加任何类型的值。

\libtitle{\texttt{insert} 方法}

在 \texttt{list} 容器的指定位置插入一个元素。该方法的原型是 \texttt{insert(index, value)},参数 \texttt{index} 是待插入的位置,\texttt{value} 是待插入的值。在 \texttt{index} 位置上插入元素以后,原来从该位置开始的所有元素都会向后移动一个元素的位置。插入操作会使 \texttt{list} 容器的元素数量加1。可以向 \texttt{list} 容器中插入任何类型的值。

假设一个 \texttt{list} 实例 \texttt{l} 的值为 \texttt{[0, 1, 2]},我们在位置 1 插入一个字符串 \texttt{'string'},则要调用 \texttt{l.insert(1, 'string')}。最后新的 \texttt{list} 值为 \texttt{[0, 'string', 1, 2]}。

如果一个 \texttt{list} 容器的元素数量为 $S$,则插入位置的取值范围为 $\{i: -S\leqslant i<S, i\in\mathbb{Z}\}$。插入位置为正值时,从 \texttt{list} 容器的头部向后索引,否则从 \texttt{list} 容器的尾部向前索引。

\libtitle{\texttt{remove} 方法}

移除容器中的一个元素。该方法的原型是 \texttt{remove(index)},参数 \texttt{index} 是待移除元素的位置。移除元素后,被移除元素后面的元素将向前移动一个元素的位置,容器的元素数量减1。和 \texttt{insert} 方法一样, \texttt{remove} 方法也可以使用正索引或者负索引。

\libtitle{\texttt{item} 方法}

获取 \texttt{list} 容器中的一个元素。该方法的原型是 \texttt{item(index)},参数 \texttt{index} 是待获取元素的索引,方法的返回值是索引位置的元素。\texttt{list} 容器支持多种索引方式:

\begin{itemize}
    \item 整数索引:索引值可以是正整数或者是负整数(和 \texttt{insert} 的索引方式一样)。此时 \texttt{item} 的返回值是索引位置的元素。如果索引位置超过了容器的元素数量或者在第 0 个元素之前,\texttt{item} 方法将返回 \texttt{nil}。
    \item \texttt{list} 索引:使用一个整数列表作为索引,\texttt{item} 返回一个 \texttt{list},返回值 \texttt{list} 中的每个元素都是参数 \texttt{list} 中的每个整数索引对应的元素。表达式 \texttt{[3, 2, 1].item([0, 2])} 的值为 \texttt{[3, 1]}。如果参数 \texttt{list} 中的某个元素类型不是整数,那么返回值 \texttt{list} 中该位置的值为 \texttt{nil}。
    \item \texttt{range} 索引:使用一个整数范围作为索引,\texttt{item} 返回一个 \texttt{list}。返回值中储存被索引 \texttt{list} 自参数 \texttt{range} 下限到上限被索引的元素。如果索引超出了被索引 \texttt{list} 的索引范围,返回值 \texttt{list} 会使用 \texttt{nil} 填充索引超出的位置。
\end{itemize}

\libtitle{\texttt{setitem} 方法}

设置容器中指定位置的值。该方法的原型是 \texttt{setitem(index, value)},\texttt{index} 是待写入元素的位置,\texttt{value} 是待写入的值。\texttt{index} 就是写入位置的整数索引值。索引位置超出容器的索引范围会导致 \texttt{setitem} 执行失败。

\libtitle{\texttt{size} 方法}

返回 \texttt{list} 容器的元素数量,也就是获取容器的长度。该方法的原型是 \texttt{size()}。

\libtitle{\texttt{resize} 方法}

重新设置 \texttt{list} 容器的长度。该方法的原型是 \texttt{resize(count)},参数 \texttt{count} 是容器的新长度。使用 \texttt{resize} 增加容器的长度时,新增的元素将会被初始化为 \texttt{nil}。使用 \texttt{reszie} 减少容器的长度时会丢弃容器尾部的部分元素。例如:
\begin{lstlisting}[language=berry, numbers=none]
l = [1, 2, 3]
l.resize(5)     # Expansion, l == [1, 2, 3, nil, nil]
l.resize(2)     # Reduce, l == [1, 2]
\end{lstlisting}

\libtitle{\texttt{iter} 方法}

返回一个用于遍历当前 \texttt{list} 容器的迭代器。

%==============================================================================
\subsection{\texttt{map} 类}

%==============================================================================
\subsection{\texttt{range} 类}

%==============================================================================
\section{扩展模块}
%==============================================================================
\subsection{JSON模块}

JSON是一种轻量级的数据交换格式,它是JavaScript的一个子集,它使用完全独立于编程语言的文本格式来表示数据。Berry提供了JSON模块来提供对JSON数据的支持。JSON模块只包含有两个函数 \texttt{load} 和 \texttt{dump},它们分别用于将JSON字符串解析乘Berry对象和将一个Berry对象序列化为JSON文本。

%%%%%%%%%%%%%%%%%%%%%%%%%%%%%%%%%%%%%%%%%%%%%%%%%%%%%%%%%%%%%%%%%%%%%%%%%%%%%%%
\libtitle{\texttt{load} 函数}

\paragraph{用法}
\begin{lstlisting}[language=berry, numbers=none]
load(text)
\end{lstlisting}

\paragraph{说明}
该函数用于将输入的JSON文本转化为Berry对象并返回。转化的规则如表\ref{tab::json2berry_rule}所示。如果JSON文本存在语法错误,该函数将返回 \texttt{nil}。
\begin{table}[htb]
    \centering
    \setlength{\tabcolsep}{18mm}
    \begin{tabular}{cc} \Xhline{1pt}
        \textbf{JSON类型} & \textbf{Berry类型} \\ \hline
        \texttt{null} & \texttt{nil} \\
        \texttt{number} & \texttt{integer} or \texttt{real} \\
        \texttt{string} & \texttt{string} \\
        \texttt{array} & \texttt{list} \\
        \texttt{object} & \texttt{map} \\
        \Xhline{1pt}
    \end{tabular}
    \caption{JSON类型到Berry类型的转换规则}
    \label{tab::json2berry_rule}
\end{table}

\paragraph{示例}
\begin{lstlisting}[language=berry, numbers=none]
import json
json.load('0')    # 0
json.load('[{"name": "liu", "age": 13}, 10.0]') # [{'name': 'liu', 'age': 13}, 10]
\end{lstlisting}

%%%%%%%%%%%%%%%%%%%%%%%%%%%%%%%%%%%%%%%%%%%%%%%%%%%%%%%%%%%%%%%%%%%%%%%%%%%%%%%
\libtitle{\texttt{dump} 函数}

\paragraph{用法}
\begin{lstlisting}[language=berry, numbers=none]
dump(object, ['format'])
\end{lstlisting}

\paragraph{说明}
该函数用于将Berry对象序列化为JSON文本。序列化的转换规则如表\ref{tab::berry2json_rule}所示。
\begin{table}[htb]
    \centering
    \setlength{\tabcolsep}{18mm}
    \begin{tabular}{cc} \Xhline{1pt}
        \textbf{Berry类型} & \textbf{JSON类型} \\ \hline
        \texttt{nil} & \texttt{null} \\
        \texttt{integer} & \texttt{number} \\
        \texttt{real} & \texttt{number} \\
        \texttt{list} & \texttt{array} \\
        \texttt{map} & \texttt{object} \\
        \texttt{map}的键 & \texttt{string} \\
        其他 & \texttt{string} \\
        \Xhline{1pt}
    \end{tabular}
    \caption{Berry类型到JSON类型的转换规则}
    \label{tab::berry2json_rule}
\end{table}

\paragraph{示例}
\begin{lstlisting}[language=berry, numbers=none]
import json
json.dump('string')     # '"string"'
json.dump('string')     # '"string"'
json.dump({0: 'item 0', 'list': [0, 1, 2]}) # '{"0":"item 0","list":[0,1,2]}'
json.dump({0: 'item 0', 'list': [0, 1, 2], 'func': print}, 'format')
#-
{
    "0": "item 0",
    "list": [
        0,
        1,
        2
    ],
    "func": "<function: 00410310>"
}
-#
\end{lstlisting}

%==============================================================================
\subsection{Math 模块}

这个模块用于提供数学函数的支持,如常用的三角函数、开方函数等。使用 math 模块要先使用 \texttt{import math} 语句来导入。本节所有的例子都假设该模块已经被正确导入。

%%%%%%%%%%%%%%%%%%%%%%%%%%%%%%%%%%%%%%%%%%%%%%%%%%%%%%%%%%%%%%%%%%%%%%%%%%%%%%%
\libtitle{\texttt{pi} 常量}

\paragraph{说明}
圆周率 $\pi$ 的近似值,实数类型,约等于 $3.141592654$。

\paragraph{示例}
\begin{lstlisting}[language=berry, numbers=none]
math.pi # 3.14159
\end{lstlisting}

%%%%%%%%%%%%%%%%%%%%%%%%%%%%%%%%%%%%%%%%%%%%%%%%%%%%%%%%%%%%%%%%%%%%%%%%%%%%%%%
\libtitle{\texttt{abs} 函数}

\paragraph{用法}
\begin{lstlisting}[language=berry, numbers=none]
abs(value)
\end{lstlisting}

\paragraph{说明}
该函数返回参数的绝对值,参数可以是整数或实数。如果没有参数,该函数返回 \texttt{0},如果有多个参数则只处理第一个参数。\texttt{abs} 函数的返回类型是实数。

\paragraph{示例}
\begin{lstlisting}[language=berry, numbers=none]
math.abs(-1)    # 1
math.abs(1.5)   # 1.5
\end{lstlisting}

%%%%%%%%%%%%%%%%%%%%%%%%%%%%%%%%%%%%%%%%%%%%%%%%%%%%%%%%%%%%%%%%%%%%%%%%%%%%%%%
\libtitle{\texttt{ceil} 函数}

\paragraph{用法}
\begin{lstlisting}[language=berry, numbers=none]
ceil(value)
\end{lstlisting}

\paragraph{说明}
该函数返回参数的向上取整值,也就是大于或者等参数的最小整数值,参数可以是整数或实数。如果没有参数,该函数返回 \texttt{0},如果有多个参数则只处理第一个参数。\texttt{ceil} 函数的返回类型是实数。

\paragraph{示例}
\begin{lstlisting}[language=berry, numbers=none]
math.ceil(-1.2) # -1
math.ceil(1.5)  # 2
\end{lstlisting}

%%%%%%%%%%%%%%%%%%%%%%%%%%%%%%%%%%%%%%%%%%%%%%%%%%%%%%%%%%%%%%%%%%%%%%%%%%%%%%%
\libtitle{\texttt{floor} 函数}

\paragraph{用法}
\begin{lstlisting}[language=berry, numbers=none]
floor(value)
\end{lstlisting}

\paragraph{说明}
该函数返回参数的向下取整值,也就是不大于参数的最大整数值,参数可以是整数或实数。如果没有参数,该函数返回 \texttt{0},如果有多个参数则只处理第一个参数。\texttt{floor} 函数的返回类型是实数。

\paragraph{示例}
\begin{lstlisting}[language=berry, numbers=none]
math.floor(-1.2) # -2
math.floor(1.5)  # 1
\end{lstlisting}

%%%%%%%%%%%%%%%%%%%%%%%%%%%%%%%%%%%%%%%%%%%%%%%%%%%%%%%%%%%%%%%%%%%%%%%%%%%%%%%
\libtitle{\texttt{sin} 函数}

\paragraph{用法}
\begin{lstlisting}[language=berry, numbers=none]
sin(value)
\end{lstlisting}

\paragraph{说明}
该函数返回参数的正弦函数值,参数可以是整数或实数,单位为弧度。如果没有参数,该函数返回 \texttt{0},如果有多个参数则只处理第一个参数。\texttt{sin} 函数的返回类型是实数。

\paragraph{示例}
\begin{lstlisting}[language=berry, numbers=none]
math.sin(1)             # 0.841471
math.sin(math.pi * 0.5) # 1
\end{lstlisting}

%%%%%%%%%%%%%%%%%%%%%%%%%%%%%%%%%%%%%%%%%%%%%%%%%%%%%%%%%%%%%%%%%%%%%%%%%%%%%%%
\libtitle{\texttt{cos} 函数}

\paragraph{用法}
\begin{lstlisting}[language=berry, numbers=none]
cos(value)
\end{lstlisting}

\paragraph{说明}
该函数返回参数的余弦函数值,参数可以是整数或实数,单位为弧度。如果没有参数,该函数返回 \texttt{0},如果有多个参数则只处理第一个参数。\texttt{cos} 函数的返回类型是实数。

\paragraph{示例}
\begin{lstlisting}[language=berry, numbers=none]
math.cos(1)         # 0.540302
math.cos(math.pi)   # -1
\end{lstlisting}

%%%%%%%%%%%%%%%%%%%%%%%%%%%%%%%%%%%%%%%%%%%%%%%%%%%%%%%%%%%%%%%%%%%%%%%%%%%%%%%
\libtitle{\texttt{tan} 函数}

\paragraph{用法}
\begin{lstlisting}[language=berry, numbers=none]
tan(value)
\end{lstlisting}

\paragraph{说明}
该函数返回参数的正切函数值,参数可以是整数或实数,单位为弧度。如果没有参数,该函数返回 \texttt{0},如果有多个参数则只处理第一个参数。\texttt{tan} 函数的返回类型是实数。

\paragraph{示例}
\begin{lstlisting}[language=berry, numbers=none]
math.tan(1)             # 1.55741
math.tan(math.pi / 4)   # 1
\end{lstlisting}

%%%%%%%%%%%%%%%%%%%%%%%%%%%%%%%%%%%%%%%%%%%%%%%%%%%%%%%%%%%%%%%%%%%%%%%%%%%%%%%
\libtitle{\texttt{asin} 函数}

\paragraph{用法}
\begin{lstlisting}[language=berry, numbers=none]
asin(value)
\end{lstlisting}

\paragraph{说明}
该函数返回参数的反正弦函数值,参数可以是整数或实数,取值范围为 $[-1,1]$。如果没有参数,该函数返回 \texttt{0},如果有多个参数则只处理第一个参数。\texttt{asin} 函数的返回类型是实数,单位为弧度。

\paragraph{示例}
\begin{lstlisting}[language=berry, numbers=none]
math.asin(1)                    # 1.5708
math.asin(0.5) * 180 / math.pi  # 30
\end{lstlisting}

%%%%%%%%%%%%%%%%%%%%%%%%%%%%%%%%%%%%%%%%%%%%%%%%%%%%%%%%%%%%%%%%%%%%%%%%%%%%%%%
\libtitle{\texttt{acos} 函数}

\paragraph{用法}
\begin{lstlisting}[language=berry, numbers=none]
acos(value)
\end{lstlisting}

\paragraph{说明}
该函数返回参数的反余弦函数值,参数可以是整数或实数,取值范围为 $[-1,1]$。如果没有参数,该函数返回 \texttt{0},如果有多个参数则只处理第一个参数。\texttt{acos} 函数的返回类型是实数,单位为弧度。

\paragraph{示例}
\begin{lstlisting}[language=berry, numbers=none]
math.acos(1)    # 0
math.acos(0)    # 1.5708
\end{lstlisting}

%%%%%%%%%%%%%%%%%%%%%%%%%%%%%%%%%%%%%%%%%%%%%%%%%%%%%%%%%%%%%%%%%%%%%%%%%%%%%%%
\libtitle{\texttt{atan} 函数}

\paragraph{用法}
\begin{lstlisting}[language=berry, numbers=none]
atan(value)
\end{lstlisting}

\paragraph{说明}
该函数返回参数的反正切函数值,参数可以是整数或实数,取值范围为 $[-\infty,+\infty]$。如果没有参数,该函数返回 \texttt{0},如果有多个参数则只处理第一个参数。\texttt{atan} 函数的返回类型是实数,单位为弧度。

\paragraph{示例}
\begin{lstlisting}[language=berry, numbers=none]
math.atan(1) * 180 / math.pi    # 45
\end{lstlisting}

%%%%%%%%%%%%%%%%%%%%%%%%%%%%%%%%%%%%%%%%%%%%%%%%%%%%%%%%%%%%%%%%%%%%%%%%%%%%%%%
\libtitle{\texttt{sinh} 函数}

\paragraph{用法}
\begin{lstlisting}[language=berry, numbers=none]
sinh(value)
\end{lstlisting}

\paragraph{说明}
该函数返回参数的双曲正弦函数值。如果没有参数,该函数返回 \texttt{0},如果有多个参数则只处理第一个参数。\texttt{sinh} 函数的返回类型是实数。

\paragraph{示例}
\begin{lstlisting}[language=berry, numbers=none]
math.sinh(1) # 1.1752
\end{lstlisting}

%%%%%%%%%%%%%%%%%%%%%%%%%%%%%%%%%%%%%%%%%%%%%%%%%%%%%%%%%%%%%%%%%%%%%%%%%%%%%%%
\libtitle{\texttt{cosh} 函数}

\paragraph{用法}
\begin{lstlisting}[language=berry, numbers=none]
cosh(value)
\end{lstlisting}

\paragraph{说明}
该函数返回参数的双曲余弦函数值。如果没有参数,该函数返回 \texttt{0},如果有多个参数则只处理第一个参数。\texttt{cosh} 函数的返回类型是实数。

\paragraph{示例}
\begin{lstlisting}[language=berry, numbers=none]
math.cosh(1) # 1.54308
\end{lstlisting}

%%%%%%%%%%%%%%%%%%%%%%%%%%%%%%%%%%%%%%%%%%%%%%%%%%%%%%%%%%%%%%%%%%%%%%%%%%%%%%%
\libtitle{\texttt{tanh} 函数}

\paragraph{用法}
\begin{lstlisting}[language=berry, numbers=none]
tanh(value)
\end{lstlisting}

\paragraph{说明}
该函数返回参数的双曲正切函数值。如果没有参数,该函数返回 \texttt{0},如果有多个参数则只处理第一个参数。\texttt{tanh} 函数的返回类型是实数。

\paragraph{示例}
\begin{lstlisting}[language=berry, numbers=none]
math.tanh(1) # 0.761594
\end{lstlisting}

%%%%%%%%%%%%%%%%%%%%%%%%%%%%%%%%%%%%%%%%%%%%%%%%%%%%%%%%%%%%%%%%%%%%%%%%%%%%%%%
\libtitle{\texttt{sqrt} 函数}

\paragraph{用法}
\begin{lstlisting}[language=berry, numbers=none]
sqrt(value)
\end{lstlisting}

\paragraph{说明}
该函数返回参数的平方根。该函数的参数不能是负数。如果没有参数,该函数返回 \texttt{0},如果有多个参数则只处理第一个参数。\texttt{sqrt} 函数的返回类型是实数。

\paragraph{示例}
\begin{lstlisting}[language=berry, numbers=none]
math.sqrt(2) # 1.41421
\end{lstlisting}

%%%%%%%%%%%%%%%%%%%%%%%%%%%%%%%%%%%%%%%%%%%%%%%%%%%%%%%%%%%%%%%%%%%%%%%%%%%%%%%
\libtitle{\texttt{exp} 函数}

\paragraph{用法}
\begin{lstlisting}[language=berry, numbers=none]
exp(value)
\end{lstlisting}

\paragraph{说明}
该函数返回参数的以自然常数 $e$ 为底的指数函数值。如果没有参数,该函数返回 \texttt{0},如果有多个参数则只处理第一个参数。\texttt{exp} 函数的返回类型是实数。

\paragraph{示例}
\begin{lstlisting}[language=berry, numbers=none]
math.exp(1) # 2.71828
\end{lstlisting}

%%%%%%%%%%%%%%%%%%%%%%%%%%%%%%%%%%%%%%%%%%%%%%%%%%%%%%%%%%%%%%%%%%%%%%%%%%%%%%%
\libtitle{\texttt{log} 函数}

\paragraph{用法}
\begin{lstlisting}[language=berry, numbers=none]
log(value)
\end{lstlisting}

\paragraph{说明}
该函数返回参数的自然对数。参数必须是一个正数。如果没有参数,该函数返回 \texttt{0},如果有多个参数则只处理第一个参数。\texttt{log} 函数的返回类型是实数。

\paragraph{示例}
\begin{lstlisting}[language=berry, numbers=none]
math.log(2.718282) # 1
\end{lstlisting}

%%%%%%%%%%%%%%%%%%%%%%%%%%%%%%%%%%%%%%%%%%%%%%%%%%%%%%%%%%%%%%%%%%%%%%%%%%%%%%%
\libtitle{\texttt{log10} 函数}

\paragraph{用法}
\begin{lstlisting}[language=berry, numbers=none]
log10(value)
\end{lstlisting}

\paragraph{说明}
该函数返回参数的以 $10$ 为底的对数。参数必须是一个正数。如果没有参数,该函数返回 \texttt{0},如果有多个参数则只处理第一个参数。\texttt{log10} 函数的返回类型是实数。

\paragraph{示例}
\begin{lstlisting}[language=berry, numbers=none]
math.log10(10) # 1
\end{lstlisting}

%%%%%%%%%%%%%%%%%%%%%%%%%%%%%%%%%%%%%%%%%%%%%%%%%%%%%%%%%%%%%%%%%%%%%%%%%%%%%%%
\libtitle{\texttt{deg} 函数}

\paragraph{用法}
\begin{lstlisting}[language=berry, numbers=none]
deg(value)
\end{lstlisting}

\paragraph{说明}
该函数用于将弧度转换为角度。参数的单位为弧度。如果没有参数,该函数返回 \texttt{0},如果有多个参数则只处理第一个参数。\texttt{deg} 函数的返回类型是实数,单位为角度。

\paragraph{示例}
\begin{lstlisting}[language=berry, numbers=none]
math.deg(math.pi) # 180
\end{lstlisting}

%%%%%%%%%%%%%%%%%%%%%%%%%%%%%%%%%%%%%%%%%%%%%%%%%%%%%%%%%%%%%%%%%%%%%%%%%%%%%%%
\libtitle{\texttt{rad} 函数}

\paragraph{用法}
\begin{lstlisting}[language=berry, numbers=none]
rad(value)
\end{lstlisting}

\paragraph{说明}
该函数用于将角度转换为弧度。参数的单位为角度。如果没有参数,该函数返回 \texttt{0},如果有多个参数则只处理第一个参数。\texttt{rad} 函数的返回类型是实数,单位为弧度。

\paragraph{示例}
\begin{lstlisting}[language=berry, numbers=none]
math.rad(180) # 3.14159
\end{lstlisting}

%%%%%%%%%%%%%%%%%%%%%%%%%%%%%%%%%%%%%%%%%%%%%%%%%%%%%%%%%%%%%%%%%%%%%%%%%%%%%%%
\libtitle{\texttt{pow} 函数}

\paragraph{用法}
\begin{lstlisting}[language=berry, numbers=none]
pow(x, y)
\end{lstlisting}

\paragraph{说明}
该函数的返回值为表达式 $x^y$ 的结果,即参数 \texttt{x} 的 \texttt{y} 次方。如果参数不全,该函数返回 \texttt{0},如果有多余的参数则只处理前两个参数。\texttt{pow} 函数的返回类型是实数。

\paragraph{示例}
\begin{lstlisting}[language=berry, numbers=none]
math.pow(2, 3) # 8
\end{lstlisting}

%%%%%%%%%%%%%%%%%%%%%%%%%%%%%%%%%%%%%%%%%%%%%%%%%%%%%%%%%%%%%%%%%%%%%%%%%%%%%%%
\libtitle{\texttt{srand} 函数}

\paragraph{用法}
\begin{lstlisting}[language=berry, numbers=none]
srand(value)
\end{lstlisting}

\paragraph{说明}
该函数用于设置随机数发生器的种子。参数的类型应当是一个整数。

\paragraph{示例}
\begin{lstlisting}[language=berry, numbers=none]
math.srand(2)
\end{lstlisting}

%%%%%%%%%%%%%%%%%%%%%%%%%%%%%%%%%%%%%%%%%%%%%%%%%%%%%%%%%%%%%%%%%%%%%%%%%%%%%%%
\libtitle{\texttt{rand} 函数}

\paragraph{用法}
\begin{lstlisting}[language=berry, numbers=none]
rand()
\end{lstlisting}

\paragraph{说明}
该函数用于获取一个随机整数。

\paragraph{示例}
\begin{lstlisting}[language=berry, numbers=none]
math.rand()
\end{lstlisting}

\subsection{Time 模块}

该模块用于提供与时间相关的功能。

%%%%%%%%%%%%%%%%%%%%%%%%%%%%%%%%%%%%%%%%%%%%%%%%%%%%%%%%%%%%%%%%%%%%%%%%%%%%%%%
\libtitle{\texttt{time} 函数}

\paragraph{原型}
\begin{lstlisting}[language=berry, numbers=none]
time()
\end{lstlisting}

\paragraph{说明}
返回当前时间戳,时间戳是自 UNIX 纪元时间(1970年1月1日00:00:00,UTC)起经过的时间,以秒为单位。

\libtitle{\title{dump} 函数}

\paragraph{原型}
\begin{lstlisting}[language=berry, numbers=none]
dump(ts)
\end{lstlisting}

\paragraph{说明}
将输入的时间戳 \texttt{ts} 转化为一个表示时间的 \texttt{map},其键值对应关系如表\ref{tab::time_dump_map}所示。
\begin{table}[htb]
    \centering
    \setlength{\tabcolsep}{2mm}
    \begin{tabular}{cccccc} \Xhline{1pt}
        \textbf{键} & \textbf{值} & \textbf{键} & \textbf{值} & \textbf{键} & \textbf{值} \\ \hline
        \texttt{'year'} & 年(从1900起) & \texttt{'month'} & 月(1-12) & \texttt{'day'} & 日(1-31) \\
        \texttt{'hour'} & 时(0-23) & \texttt{'min'} & 分(0-59) & \texttt{'sec'} & 秒(0-59) \\
        \texttt{'weekday'} & 星期(1-7) \\
        \Xhline{1pt}
    \end{tabular}
    \caption{\texttt{time.dump} 函数返回值的键值关系}
    \label{tab::time_dump_map}
\end{table}

\subsection{String 模块}

\subsection{OS 模块}

    \chapter{API}

\section{API接口概述}


\end{document}
