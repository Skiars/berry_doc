\chapter{FFI}

\textbf{语言交互接口}(Foreign Function Interface,FFI)是不同语言之间交互的接口。Berry提供了一套FFI来实现与C语言之间的交互,这套接口也非常容易在C++中使用。大部分的FFI接口是一些函数,它们的声明都放在\textit{berry.h}文件中。为了降低RAM的使用量,FFI还提供一套C编译期生成固定哈希表的机制,该机制必须要用到外部工具来生成C代码。

\section{基础}

FFI中最为重要的交互功能的应该是Berry代码与C函数相互调用的功能。为了实现两种语言互相调用对方的函数,我们要先了解Berry函数的参数传递机制。

\subsection{虚拟栈}

Berry使用一个虚拟栈(virtual stack)和C语言编写的原生函数传递值,栈中每个元素都是一个Berry值。Berry代码调用原生函数时总会创建一个新的栈,并把所有的参数压入栈中。在C代码中也可以使用这个虚拟栈来存储数据,存储在栈中的值不会被垃圾回收器回收。

\begin{wrapfigure}{r}{0.5\textwidth}
\centering
\begin{tikzpicture}
    \begin{scope}[
    start chain=1 going right,start chain=2 going below,node distance=-0.15mm,minimum width=0.5cm,minimum height=0.6cm,font=\small\ttfamily
  ]
  \node [on chain=1] at (-1.5, -0.4) {\ldots};
  \foreach \x in {-1,...,-8} {
      \x, \node at (3.5+\x*0.5, 0.2) {\x};
  }
  \node [draw,on chain=1] {};
  \foreach \x in {1,...,8} {
      \x, \node (\x) [draw,on chain=1] {\x};
  }
  \foreach \x in {9,...,10} {
      \x, \node [draw,on chain=1] {};
  }
  \node [on chain=1] {\ldots};
  \node (base) at (-0.5, -1.2) {base};
  \node (top)  at (3, -1.2) {top};

  \draw[->] (base) -- (1);
  \draw[->] (top) -- (8);
  \end{scope}
\end{tikzpicture}
\caption{虚拟堆栈}
\label{fig::virtual_stack}
\end{wrapfigure}

Berry使用的堆栈如图\ref{fig::virtual_stack}所示,该堆栈从左往右增长。Berry代码调用一个原生函数时会得到一个初始的堆栈,该堆栈第一个值的位置称为\textbf{栈底}(\texttt{base}),而最后一个值称为\textbf{栈顶}(\texttt{top}),函数只能访问堆栈中栈底到栈顶之间的值。栈底的位置是固定的,而栈顶的位置可以根据需要上涨或者下降。堆栈为空时,栈底指针\texttt{base}将会大于栈顶指针\texttt{top}。虚拟堆栈不严格遵循栈的操作规则:除了push和pop以外,虚拟栈还可以通过索引来访问,甚至可以在任意位置插入或者删除值。索引栈中元素的方法有两种:一种是以栈底为参考的\textbf{绝对索引},绝对索引值是从$1$开始的正整数;另一种是以栈顶为参考的\textbf{相对索引},相对索引值是从$-1$开始的负整数。以图\ref{fig::virtual_stack}为例,索引值$1,2\ldots 8$是绝对索引,元素的绝对索引就是该元素到栈底的距离。索引值$-1,-2\ldots -8$是相对索引,元素的相对索引值为该元素到栈顶距离的负数。如果一个索引值$index$有效,那么它所指的元素应该处于栈底到栈顶之间,也就是满足表达式$1\leq \mathrm{abs}(index)\leq top-base+1$。
