\chapter{FFI}

\textbf{语言交互接口}(Foreign Function Interface,FFI)是不同语言之间交互的接口。Berry提供了一套FFI来实现与C语言之间的交互,这套接口也非常容易在C++中使用。大部分的FFI接口是一些函数,它们的声明都放在\textit{berry.h}文件中。为了降低RAM的使用量,FFI还提供一套C编译期生成固定哈希表的机制,该机制必须要用到外部工具来生成C代码。

\section{基础}

FFI中最为重要的交互功能的应该是Berry代码与C函数相互调用的功能。为了实现两种语言互相调用对方的函数,我们要先了解Berry函数的参数传递机制。

\subsection{虚拟栈}

Berry使用一个虚拟栈(stack)和C语言编写的原生函数传递值,栈中每个元素都是一个Berry值。Berry代码调用原生函数时总会创建一个新的栈,而所有参数会压入栈中,在C代码中也可以直接使用这个虚拟栈来存储数据。
