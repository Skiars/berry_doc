\chapter{基本信息}

\section{简介}

Berry是一款超轻量级的动态类型嵌入式脚本语言,该语言主要支持过程式编程,同时也支持面向对象编程和函数式编程。Berry的一个重要的设计目标是能够在内存非常小的嵌入式设备上运行,因此该语言十分精简。尽管如此,Berry依然是一款功能丰富的脚本语言,

\section{开始使用}

\subsection{获取解释器}

读者可以到项目的GitHub页面(\url{https://github.com/gztss/berry})上获取Berry解释器的源代码。读者需要自行编译Berry解释器。具体的编译方法可以查阅源码根目录下的README.md文档,在项目的GitHub页面上也可以查看这份文档。

首先要要安装好GCC和git等工具,对于Linux和macOS系统,还要安装readline库。准备工作完成后使用 \texttt{git} 命令把解释器源代码从远程仓库克隆到本地:
\begin{lstlisting}[language=bash, numbers=none]
git clone https://github.com/gztss/berry.git
\end{lstlisting}
进入berry目录,使用 \texttt{make} 命令编译解释器:
\begin{lstlisting}[language=bash, numbers=none]
cd berry
make
\end{lstlisting}

现在应该能在berry目录下找到解释器的可执行文件(在Windows系统中,解释器程序的文件名是``berry.exe'',在Linux和macOS系统中的文件名是``berry''),你可以直接运行该可执行文件\footnote{在Windows中你可以直接双击运行可执行文件,在Linux或者macOS中通常要使用``终端''(Terminal)来运行。你也可以在Windows的``命令提示符''(cmd)窗口中运行解释器。具体的使用方法请参考README.md文档。}来启动解释器。在Linux或者macOS系统下可以使用命令 \texttt{sudo make install} 安装解释器,此后你可以在终端中输入 \texttt{berry} 来启动解释器。

\subsection{REPL环境}

直接启动解释器(在终端或命令窗口里输入 \texttt{berry} 而不带参数,或者在Windows中双击berry.exe)时会看到以下界面:
\begin{lstlisting}[language=berry, numbers=none]
Berry 0.0.4 (build in Feb  1 2019, 13:14:04)
[GCC 8.1.0] on Windows (default)
>
\end{lstlisting}
界面的前两行显示了Berry解释器的版本、编译时间、编译器和操作系统等信息,第三行的符号``\texttt{>}''叫做提示符,光标显示在提示符后面。

不带参数地启动解释器会进入REPL(Read Eval Print Loop)模式,也就是交互模式。在REPL模式下,我们可以输入源代码然后按下``Enter''键之后执行。

\subsubsection{Hello World程序}

以经典的``Hello World''程序为例,在REPL中输入 \texttt{print('Hello World')} 并执行,运行结果如下:
\begin{lstlisting}[language=berry, numbers=none]
Berry 0.0.4 (build in Feb  1 2019, 13:14:04)
[GCC 8.1.0] on Windows (default)
> print('Hello World')
Hello World
>
\end{lstlisting}
解释器输出了``\texttt{Hello World}''文本。这行代码通过调用 \texttt{print} 函数来实现对字符串 \texttt{'Hello World'} 的输出。在REPL中,如果表达式的返回值不是 \texttt{nil} 则会显示该返回值。例如输入表达式 \texttt{1 + 2} 会显示计算结果 \texttt{3}:
\begin{lstlisting}[language=berry, numbers=none]
> 1 + 2
3
\end{lstlisting}

因此REPL下最简单的``Hello World''程序是直接输入字符串 \texttt{'Hello World'} 并运行:
\begin{lstlisting}[language=berry, numbers=none]
> 'Hello World'
Hello World
\end{lstlisting}

\subsubsection{REPL的更多用法}

你还可以把Berry解释器的交互模式当成一个科学计算器来使用,不过,一些数学函数不能直接使用,而要先使用 \texttt{import math} 语句来导入数学库,然后在使用数学库中的函数时要使用``\texttt{math.}''作为前缀,例如 \texttt{sin} 函数要写成 \texttt{math.sin}:
\begin{lstlisting}[language=berry, numbers=none]
> import math
> math.pi
3.14159
> math.sin(math.pi / 2)
1
> math.sqrt(2)
1.41421
\end{lstlisting}

\subsection{脚本文件}

Berry的脚本文件是存储Berry代码的文件,脚本文件可以由解释器执行。通常情况下,脚本文件是扩展名为``.be''的文本文件。使用解释器执行脚本的命令是:
\begin{lstlisting}[language=bash, numbers=none]
berry script_file
\end{lstlisting}
\texttt{script\_file} 是脚本文件的文件名。使用该命令会运行解释器执行 \texttt{script\_file} 脚本文件中的Berry代码,执行完毕后解释器会退出。

如果希望执行完脚本文件后解释器进入REPL模式可以在调用解释器的命令中加入 \texttt{-i} 参数:
\begin{lstlisting}[language=bash, numbers=none]
berry -i script_file
\end{lstlisting}
这条命令会先执行 \texttt{script\_file} 文件中的代码,然后进入REPL模式。

\section{词法}

在介绍Berry的语法之前,我们先来看一段简单的代码(你可以在REPL模式中运行这段代码):
\begin{lstlisting}[language=berry]
def func(x) # a function example
    return x + 1.5
end
print('func(10) =', func(10))
\end{lstlisting}

这段代码中定义了一个函数 \texttt{func} 并在后面调用了它。在了解这段代码怎样工作之前,我们先介绍Berry语言的语法元素。

以上代码中,语法元素的具体分类为: \texttt{def} 、 \texttt{return} 和 \texttt{end} 是Berry语言的关键字;而第1行中的``\texttt{\# a function example}''被称为注释; \texttt{print} 、 \texttt{func} 和 \texttt{x} 是一些标识符,它们通常用于表示一个变量; \texttt{1.5} 和 \texttt{10} 这些数字被称为数值字面量,它们相当于日常生活中使用的数字; \texttt{'func(10) ='} 是一个字符串字面量,他们大量用于需要表示文本的地方; \texttt{+} 是一个加法运算符,这里使用加法运算符可以将变量 \texttt{x} 和数值 \texttt{1.5} 相加。

以上的分类实际上是从词法分析器的角度来做的。词法分析是Berry源代码解析的第一步,为了写出正确的源代码,我们先从最基础的词法开始介绍。

\subsection{注释}

注释是不会生成任何代码的一些文本,它们用于在源代码中做批注并给人们阅读,而编译器则不会解释它们的内容。Berry支持单行注释和跨行的块注释。单行注释从字符``\texttt{\#}'开始,直到换行字符结束。快注释从文本``\texttt{\#-}''开始,直到文本``\texttt{-\#}''结束。以下是使用注释的例子:
\begin{lstlisting}[language=berry, numbers=none]
# This is a line comment
#- This is a
   block comment
-#
\end{lstlisting}

和C语言类似,快注释不支持嵌套,以下代码将在第一个``\texttt{-\#}''文本处终止对注释的解析:
\begin{lstlisting}[language=berry, numbers=none]
#- Some comments -# ... -#
\end{lstlisting}

\subsection{字面值}

字面值是编程时在源代码中直接写出的固定值。Berry的字面量有整数、实数、布尔量、字符串和nil。例如,数值 \texttt{34} 是一个整数字面值。

\subsubsection{数值字面值}

数值字面值包括\textbf{整数}(integer)字面值和\textbf{实数}(real)字面值。
\begin{lstlisting}[language=berry, numbers=none]
40      # Integer literal
0x80    # Hexadecimal literal (integer)
3.14    # Real literal
1.1e-6  # Real literal
\end{lstlisting}

数值字面值的写法和日常写法类似。Berry支持16进制的整数字面值,16进制字面值使用前缀 \texttt{0x} 或者 \texttt{0X} 开头,后面是一个16进制数。

\subsubsection{布尔字面值}

布尔值(boolean)用来表示逻辑状态中的真和假。你可以使用 \texttt{true} 和 \texttt{false} 这两个关键字来表示布尔字面值。

\subsubsection{字符串字面值}

字符串(string)是一段文本,它的字面值写法是使用一对 \texttt{'} 或 \texttt{"} 包围字符串文本:
\begin{lstlisting}[language=berry, numbers=none]
'this is a string'
"this is a string"
\end{lstlisting}

字符串字面值提供一些转义序列来表示不能直接输入的字符,转义序列以字符 \texttt{'\textbackslash'} 开始,然后紧跟一个特定的字符序列实现转义。Berry规定的转义序列有
\begin{table}[htb]
    \centering
    \setlength{\tabcolsep}{4mm}
    \begin{tabular}{clclcl} \Xhline{1pt}
        \makecell[c]{\textbf{转义字符}} & \makecell[l]{\textbf{意义}} & \makecell[c]{\textbf{转义字符}} & \makecell[l]{\textbf{意义}} & \makecell[c]{\textbf{转义字符}} & \makecell[l]{\textbf{意义}} \\ \Xhline{1pt}
        \texttt{\textbackslash a} & 响铃 & \texttt{\textbackslash b} & 退格符 & \texttt{\textbackslash f} & 换页符 \\
        \texttt{\textbackslash n} & 换行符 & \texttt{\textbackslash r} & 回车符 & \texttt{\textbackslash t} & 水平制表符 \\
        \texttt{\textbackslash v} & 垂直制表符 & \texttt{\textbackslash \textbackslash} & 反斜线 & \texttt{\textbackslash '} & 单引号 \\
        \texttt{\textbackslash "} & 双引号 & \texttt{\textbackslash ?} & 问号 & \texttt{\textbackslash 0} & 空字符 \\
        \Xhline{1pt}
    \end{tabular}
    \caption{转义字符序列}
    \label{tab::escape_character}
\end{table}

转义序列可以在字符串中使用,例如
\begin{lstlisting}[language=berry, numbers=none]
print('escape character LF\n\tnew line')
\end{lstlisting}
的运行结果是
\begin{lstlisting}[numbers=none]
escape character LF
        new line
\end{lstlisting}

还可以使用泛化的转义序列,其形式是 \texttt{\textbackslash x} 后跟2个十六进制数,或者是 \texttt{\textbackslash} 3个八进制数,使用这种转义序列可以表示任意字符。下面是使用ASCII字符集的一些例子:
\begin{lstlisting}[language=berry, numbers=none]
'\115' #- 'M' -#    '\x34' #- '4' -#    '\064' #- '4' -#
\end{lstlisting}

\subsubsection{Nil字面值}

Nil表示空值,其字面值使用关键字 \texttt{nil} 来表示。

\subsection{标识符} \label{section:identifier}

标识符(identifier)是由用户定义的名字,它由下划线或者字母作为开头,再由若干个下划线、字母或者数字的组合构成。和大多数语言类似,Berry是大小写敏感的,因此标识符 \texttt{A} 和标识符 \texttt{a} 会解析为两种标识符。
\begin{lstlisting}[language=berry, numbers=none]
a
TestVariable
Test_Var
_init
baseCass
_
\end{lstlisting}

\subsection{关键字}

Berry保留以下记号作为语言的关键字:
\begin{lstlisting}[language=berry, numbers=none]
    if          elif        else        while       for         def
    end         class       break       continue    return      true
    false       nil         var         do          import      as
\end{lstlisting}

关键字的具体使用方法会在相关的章节中介绍。注意,不能将关键字作为标识符使用,由于Berry是大小写敏感的,因此 \texttt{If} 可以用于标识符。
